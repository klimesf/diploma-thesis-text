\documentclass[11pt,twoside,a4paper]{book}
% definice dokumentu
\usepackage[czech, english]{babel}
\usepackage[T1]{fontenc}                % pouzije EC fonty
\usepackage[utf8]{inputenc}             % utf8 kódování vstupu
\usepackage[square, numbers]{natbib}    % sazba pouzite literatury
\usepackage{indentfirst}                % 1. odstavec jako v cestine, pro práci v aj možno zakomentovat
\usepackage{fancyhdr}                   % tisk hlaviček a patiček stránek
\usepackage{nomencl}                    % umožňuje snadno definovat zkratky a jejich seznam

%%%%%%%%%%%%%%%%%%%%%%%%%%%%%%%%%%%%%%%%%%%%%%%%%%%%%%%%%%%%%%%
% informace o práci
\newcommand\WorkTitle{Centrální správa a automatická integrace byznys pravidel v architektuře orientované na služby} % název
\newcommand\FirstandFamilyName{Bc. Filip Klimeš}                                                                     % autor
\newcommand\Supervisor{Ing. Karel Čemus}                                                                             % vedoucí

\newcommand\TypeOfWork{Diplomová práce} % typ práce [Diplomová práce | Bakalářská práce | Bachelor's Project | Master's Thesis ]

% Nastavte následují podle vašeho oboru a programu (pomoc hledejte na http://www.fel.cvut.cz/cz/education/bk/prehled.html)								
\newcommand\StudProgram{Otevřená informatika, Magisterský}    % program
\newcommand\StudBranch{Softwarové inženýrství}                % obor

%%%%%%%%%%%%%%%%%%%%%%%%%%%%%%%%%%%%%%%%%%%%%%%%%%%%%%%%%%%%%%%
% minimální importy
\usepackage{graphicx}               % pro vkládání obrázků
\usepackage{k336_thesis_macros}     % specialni makra pro formatovani DP a BP
\usepackage[
pdftitle={\WorkTitle},              % nastaví v informacích o pdf název
pdfauthor={\FirstandFamilyName},    % nastaví v informacích o pdf autora
colorlinks=true,                    % před tiskem doporučujeme nastavit na false, aby odkazy a url nebyly šedé při ČB tisku
breaklinks=true,
urlcolor=red,
citecolor=blue,
linkcolor=blue,
unicode=true,
]
{hyperref}                          % pro zobrazování "prokliknutelných" linků

% rozšiřující importy
\usepackage{listings}               % slouží pro tisk zdrojových kódů se syntax higlighting
\usepackage{algorithmicx}           % slouží pro zápis algoritmů
\usepackage{algpseudocode}          % slouží pro výpis pseudokódu
\usepackage{enumitem}
\usepackage{scrextend}
\usepackage{blindtext}
\usepackage{multirow}
\usepackage{makecell}
\usepackage{wrapfig}
\usepackage{lscape}
\usepackage{rotating}
\usepackage{epstopdf}
\usepackage{sourcecode}
\usepackage{protobuf/lang}  % include language definition for protobuf
\usepackage{protobuf/style} % include custom style for proto declarations.

\usepackage[style=long,acronym,nomain,numberedsection]{glossaries}
\makeglossaries

\newglossaryentry{ADDA}{name=ADDA,description={Aspect-Driven Design Approach}}
\newglossaryentry{API}{name=API,description={Application Programming Interface}}
\newglossaryentry{AST}{name=AST,description={Abstract Syntax Tree}}
\newglossaryentry{AOP}{name=AOP,description={Aspect Oriented Programming}}
\newglossaryentry{BDD}{name=BDD,description={Behaviour Driven Development}}
\newglossaryentry{CI}{name=CI,description={Continuous Integration}}
\newglossaryentry{CORBA}{name=CORBA,description={Common Object Request Broker Architecture}}
\newglossaryentry{CRUD}{name=CRUD,description={Create, Read, Update, Delete}}
\newglossaryentry{CSS}{name=CSS,description={Cascading Style Sheets}}
\newglossaryentry{DOM}{name=DOM,description={Document Object Model}}
\newglossaryentry{DSL}{name=DSL,description={Domain-Specific Language}}
\newglossaryentry{EIS}{name=EIS,description={Enterprise Information System}}
\newglossaryentry{EL}{name=EL,description={Expression Language}}
\newglossaryentry{ESB}{name=ESB,description={Enterprise Service Bus}}
\newglossaryentry{HATEOAS}{name=HATEOAS,description={Hypermedia as the engine ot application state}}
\newglossaryentry{HTML}{name=HTML,description={Hypertext Markup Language}}
\newglossaryentry{HTTP}{name=HTTP,description={Hypertext Transfer Protocol}}
\newglossaryentry{IDL}{name=IDL,description={Interface Description Language}}
\newglossaryentry{IS}{name=IS,description={Informačn\'{\i} systém}}
\newglossaryentry{JSON}{name=JSON,description={JavaScript Object Notation}}
\newglossaryentry{MDA}{name=MDA,description={Model-Driver Architecture}}
\newglossaryentry{MQ}{name=MQ,description={Message Queue}}
\newglossaryentry{OMG}{name=OMG,description={Object Modeling Group}}
\newglossaryentry{ORB}{name=ORB,description={Object Request Broker}}
\newglossaryentry{OOP}{name=OOP,description={Object Oriented Programming}}
\newglossaryentry{P2P}{name=P2P,description={Peer-to-peer}}
\newglossaryentry{PIM}{name=PIM,description={Platform Independent Model}}
\newglossaryentry{PSM}{name=PSM,description={Platform Specific Model}}
\newglossaryentry{REST}{name=REST,description={Representational State Transfer}}
\newglossaryentry{RMI}{name=RMI,description={Remote Method Invocation}}
\newglossaryentry{RPC}{name=RPC,description={Remote Procedure Call}}
\newglossaryentry{SLOC}{name=SLOC,description={Source Lines of Code}}
\newglossaryentry{SOA}{name=SOA,description={Service Oriented Architecture}}
\newglossaryentry{SOAP}{name=SOAP,description={Simple Object Access Protocol}}
\newglossaryentry{TCP}{name=TCP,description={Transmission Control Protocol}}
\newglossaryentry{UC}{name=UC,description={Use Case}}
\newglossaryentry{UI}{name=UI,description={User Interface}}
\newglossaryentry{UML}{name=UML,description={Unified Modeling Language}}
\newglossaryentry{URL}{name=URL,description={Uniform Resource Locator}}
\newglossaryentry{URI}{name=URI,description={Uniform Resource Identifier}}
\newglossaryentry{WSDL}{name=WSDL,description={Web Service Description Language}}
\newglossaryentry{XML}{name=XML,description={Extensible Markup Language}}
\newglossaryentry{XSD}{name=XSD,description={XML Schema Definition}}
\newglossaryentry{YAML}{name=YAML,description={YAML Ain't Markup Language}}


% custom macros
\usepackage{custom}

%%%%%%%%%%%%%%%%%%%%%%%%%%%%%%%%%%%%%%%%%%%%%%%%%%%%%%%%%%%%%%%
% příkazy šablony
\makenomenclature                                % při překladu zajistí vytvoření pracovního souboru se seznamem zkratek

\let\oldUrl\url                                  % url adresy budou zobrazeny: <url>
\renewcommand\url[1]{<\texttt{\oldUrl{#1}}>}

%%%%%%%%%%%%%%%%%%%%%%%%%%%%%%%%%%%%%%%%%%%%%%%%%%%%%%%%%%%%%%%
% vaše vlastní příkazy
\newcommand*{\nomExpl}[2]{#2 (#1)\nomenclature{#1}{#2}}   % usnadňuje zápis zkratek : Slova ke Zkrácení (SZ)
\newcommand*{\nom}[2]{#1\nomenclature{#1}{#2}}            % usnadňuje zápis zkratek : SZ
\linespread{1.2}


%%%%%%%%%%%%%%%%%%%%%%%%%%%%%%%%%%%%%%%%%%%%%%%%%%%%%%%%%%%%%%%
% vlastní dokument
%%%%%%%%%%%%%%%%%%%%%%%%%%%%%%%%%%%%%%%%%%%%%%%%%%%%%%%%%%%%%%%
\begin{document}

%%%%%%%%%%%%%%%%%%%%%%%%%%
% nastavení jazyka, kterým je práce psána
\selectlanguage{czech}    % podle jazyka práce nastavte na [czech | english]
\dptranslate                % nastaví české nebo anglické popisy (např. katedra -> department); viz k336_thesis_macros

%%%%%%%%%%%%%%%%%%%%%%%%%%
% Poznamky ke kompletaci prace
% Nasledujici pasaz uzavrenou v {} ve sve praci samozrejme
% zakomentujte nebo odstrante.
% Ve vysledne svazane praci bude nahrazena skutecnym
% oficialnim zadanim vasi prace.
{
\pagenumbering{roman} \cleardoublepage \thispagestyle{empty}
\chapter*{Na tomto místě bude oficiální zadání vaší práce}
\begin{itemize}
\item Toto zadání je podepsané děkanem a vedoucím katedry,
\item musíte si ho vyzvednout na studijním oddělení Katedry počítačů na Karlově náměstí,
\item v jedné odevzdané práci bude originál tohoto zadání (originál zůstává po obhajobě na katedře),
\item ve druhé bude na stejném místě neověřená kopie tohoto dokumentu (tato se vám vrátí po obhajobě).
\end{itemize}
\newpage
}

%%%%%%%%%%%%%%%%%%%%%%%%%%
% Titulni stranka / Title page
\coverpagestarts

%%%%%%%%%%%%%%%%%%%%%%%%%%%
% Poděkovani / Acknowledgements

\acknowledgements
\noindent
Zde můžete napsat své poděkování, pokud chcete a máte komu děkovat.


%%%%%%%%%%%%%%%%%%%%%%%%%%%
% Prohlášení / Declaration

\declaration{V~Praze dne 20.\,5.\,2018}
%\declaration{In Kořenovice nad Bečvárkou on May 15, 2008}


%%%%%%%%%%%%%%%%%%%%%%%%%%%%
% Abstrakt / Abstract

\abstractpage

Translation of Czech abstract into English.

% Prace v cestine musi krome abstraktu v anglictine obsahovat i
% abstrakt v cestine.
\vglue60mm

\noindent{\Huge \textbf{Abstrakt}}
\vskip 2.75\baselineskip

\noindent
Abstrakt práce by měl velmi stručně vystihovat její obsah. Tedy čím se práce zabývá a co je jejím výsledkem/přínosem.

\noindent
Očekávají se cca 1 -- 2 odstavce, maximálně půl stránky.

%%%%%%%%%%%%%%%%%%%%%%%%%%
% obsahy a seznamy
\tableofcontents        % Obsah / Table of Contents

% pokud v práci nejsou obrázky nebo tabulky - odstraňte jejich seznam
\listoffigures            % Obsah / Table of Contents
%\listoftables            % Seznam tabulek / List of Tables

\renewcommand{\lstlistingname}{Zdrojový kód}
\renewcommand{\lstlistlistingname}{Seznam zdrojových kódů}
\lstset{
basicstyle=\fontsize{10}{12}\selectfont\ttfamily
}
\lstlistoflistings

%%%%%%%%%%%%%%%%%%%%%%%%%%
% začátek textu
\mainbodystarts

%\usepackage[T1]{fontenc}
%\usepackage[utf8]{inputenc}

%!TEX ROOT=../diploma-thesis.tex

\chapter{Úvod}\label{ch:uvod}

\todo{
\begin{itemize}
    \item popis tématu a jeho důležitost
    \item motivace práce
    \item co bude c\'{\i}lem a požadovan\'ym v\'ystupem této práce
    \item popis struktury DP
\end{itemize}
}

\section{Jak č\'{\i}st tuto práci}

\goal{Popis struktury DP a obsah kapitol}
Kapitola~\ref{ch:analyza} se věnuje detailn\'{\i} anal\'yze problematiky bynysov\'ych pravidel a
architektury orientované na služby, včetně jej\'{\i}ho historického v\'yvoje až po nejnovějš\'{\i} trendy
a v závěru identifikuje požadavky kladené na implementaci knihovny pro centráln\'{\i} správu a
automatickou distribuci byznysov\'ych pravidel v této architektuře. Kapitola~\ref{ch:reserse}
se zab\'yvá rešerš\'{\i} stávaj\'{\i}c\'{\i}ch pr\'{\i}stupu k v\'yvoji informacn\'{\i}ch systému a speciálně se zaměřuje
na koncepty aspektově orientovaného programován\'{\i} a modern\'{\i}ho aspekty ř\'{\i}zeného př\'{\i}stupu k návrhu
systémů. Dále se kapitola věnuje průzkumu existuj\'{\i}c\'{\i}ch nástrojů pro správu byznysov\'ych pravidel.
Kapitola~\ref{ch:navrh} formalizuje prostřed\'{\i} architektury orientované na služby do terminologie
aspektově orientovaného programován\'{\i} a na základě této formalizace navrhuje koncept frameworku,
kter\'y realizuje centráln\'{\i} správu a automatickou distribuci byznysov\'ych pravidel.
V kapitole~\ref{ch:implementace} je detailně probrána implementace knihoven pro navržen\'y framework
pro platformy jazyků Java a Python a frameworku Node.js. Následuj\'{\i}c\'{\i} kapitola~\ref{ch:verifikace}
popisuje, jak\'ym způsobem byly tyto knihovny otestovány a jak byla prokázána jejich funkčnost. Zároveň
je zde popsána validace a vyhodnocen\'{\i} konceptu frameworku jeho nasazen\'{\i}m při v\'yvoji
jednoduchého ukázkového e-commerce systému. V posledn\'{\i} kapitole~\ref{ch:zaver} je shrnuto, jak\'ych
c\'{\i}lů bylo v práci dosáhnuto a jak\'ym dalš\'{\i}m směrem se může v\'yzkum v této oblasti ub\'{\i}rat.

\goal{Zkratky}
Zkratky použ\'{\i}vané v textu jsou vždy nejdř\'{\i}ve zavedeny a vysvětleny. Pro přehlednost byl ale zároveň
souhrn použ\'{\i}van\'ych zkratek přidán jako př\'{\i}loha práce~\ref{ch:shortcuts}.

%!TEX ROOT=../diploma-thesis.tex

\chapter{Anal\'yza}\label{ch:analyza}

\section{Byznys pravidla}

\section{Architektura orientovaná na služby}

\section{Problémy}

\section{Identifikace požadavků na implementaci frameworku}

\section{Shrnutí}

%!TEX ROOT=../diploma-thesis.tex

\chapter{Rešerše}\label{ch:reserse}

Tato kapitola se věnuje rešerši existujících řešení
a výzkumu relevantnímu k tématu této práce. Díky tomu bude umožněno
dosáhnout kvalitního a efektivního návrhu frameworku pro centrální správu a automatickou
distribuci byznysových pravidel.
Kapitola zkoumá modelem řízenou architekturu, generativní
programování a BPEL, jejich výhody, nevýhody a vhodnost použití.
Dále se zaměřuje na aspektově orientované programování
a na něm založený inovativní přístup k návrhu informačních systémů \gls{ADDA}.
Kapitola také zkoumá existující nástroje a specializované jazyky pro vyjádření
byznysových pravidel.
Nakonec se věnuje i shrnutí síťových architektur, které mohou být využity
pro automatickou distribuci byznysových pravidel v \gls{SOA}.

\section{Modelem ř\'{\i}zená architektura}

Modelem řízená architektura (\gls{MDA} z anglického \textit{Model-Driven
Architecture}) se zaměřuje na návrh \gls{IS} s využitím modelů a jejich
následnou transformaci do spustitelného kódu pomocí generativních nástrojů~\cite{soley2000model}.
Hlavní výhodou \gls{MDA} je vysoká úroveň abstrakce, která zbavuje vývojáře nutnosti
manuálně duplikovat informace. Další výhodou je nezávislost na platformě a zvýšení
kvality kódu díky jeho automatickému generování.

%Autorem specifikací \gls{MDA} je konsorcium Object Modeling Group (\gls{OMG}),
%které se zaměřuje na standardizaci modelovacích standardů pro software
%a stojí za modelovacím jazykem \gls{UML}, který je de facto globálním standardem
%pro vizualizaci statických i dynamických aspektů softwarových systémů.

\gls{MDA} využívá Computation Independent Model (\gls{CIM}), který reprezentuje
řešení nezávislé na použitých výpočetních metodách a algoritmech. Z \gls{CIM} je
model převeden do Platform Independent Model (\gls{PIM}),
který popisuje koncepci systému bez ohledu na implementační detaily, typicky k popisu
využívá jazyk \gls{UML}. \gls{PIM} je následně převeden do
Platform Specific Model (\gls{PSM}), tedy modelu využívajícího
specifických aspektů platformy, pro kterou má být systém postaven.
\gls{PIM} může být převeden na jeden či více \gls{PSM}.
Nakonec je \gls{PSM} transformován do spustitelného kódu.

Hlavní nevýhodou \gls{MDA}, která zabraňuje jejímu využití
pro účel této práce, je jednosměrný dopředný proces, kterým je výsledný kód generován.
Pokud dojde ke změně požadavků, která se promítne do modelu, je potřeba přegenerovat
kód celého systému. Kód, který bylo nutno doplnit ručně, může snadno zastarat a je tak
potřeba ho manuálně projít a opravit.
%Změna byznysových pravidel v našem případě by
%navíc při využití \gls{MDA} vyžadovala znovu-nasazení celého
%systému, což je v přímém rozporu s požadavkem na možnost dynamicky
%upravovat či přidávat byznysová pravidla za běhu systému.
Další nevýhodou tohoto přístupu je jeho závislost na \gls{OOP},
které samotné není schopné se efektivně vypořádat s průřezovými
problémy~\cite{cemus2014aspect}, jak si popíšeme v sekci~\ref{sec:aop}.
%Ačkoliv je \gls{MDA} známá již od roku 2000, generativní nástroje pro
%její podporu jsou stále nevyspělé a pro některé platformy úplně chybí.

\section{Generativní programování}

Generativní programování (\gls{GP}) je dalším příkladem paradigmatu, který
využívá vyšší úroveň abstrakce a díky tomu zvýšuje znovupoužitelnost
kódu. \gls{GP} se zaměřuje na maximalizaci automatizace vývoje systému
skrz generování a syntézu vysoce přizpusobitelných komponent. Vývojář
popíše komponentu v abstraktním jazyce přizpůsobeném doméně řešeného
problému a generátor se postará o její automatické vytvoření~\cite{czarnecki2000generative}.
Díky tomu je možné oddělit popis jednotlivých vlastností systému a dosáhnout tak
jejich vysoké znovupoužitelnosti.

\gls{GP} by mohlo být využito pro abstrakci bynysových pravidel a jejich automatickému
začleňování do kódu služeb v systému stavějícím na \gls{SOA}.
Statické generování komponent však nesplňuje požadavek na dynamickou správu
byznysových pravidel za běhu systému.

\section{Business Process Execution Language}

\gls{BPEL}

\todo{
\begin{itemize}
    \item Co to je
    \item Jak to funguje
    \item Výhody a nevýhody
\end{itemize}
}

\section{Objektově orientované programování}\label{sec:oop}

\goal{OOP a jeho popis}
Jedn\'{\i}m z nejpoužívanějších paradigmat používaných k
v\'yvoji modern\'{\i}ch \gls{IS} je objektově orientované programován\'{\i} (\gls{OOP}).
To používá koncept tzv. objektů, které zapouzdřují data a funkcionalitu do
malých funkčních celků odpov\'{\i}daj\'{\i}c\'{\i} struktuře reálného světa~\cite{rentsch1982object}.
Objekty se rozum\'{\i} jak konkrétn\'{\i} koncepty, např\'{\i}klad auto nebo člověk, tak i
abstraktn\'{\i} koncepty, jako je bankovn\'{\i} transakce nebo objednávka v obchodě.
Podoba objektů se pak prom\'{\i}tá do kódu programu i do reprezentace struktur v paměti
poč\'{\i}tače. Tento př\'{\i}stup je velmi snadn\'y pro pochopen\'{\i},
vede k lepš\'{\i}mu návrhu a organizaci programu a snižuje
tak náklady na jeho v\'yvoj a údržbu.

Vlastnosti \gls{OOP} jako je zapouzdření, dědičnost a polymorfismus přináší
vysokou znovupoužitelnost kódu, nižší riziko lidské chyby, zjednodušení
návrhu systému a nižší náklady na vývoj a údržbu software.

\section{Aspektově orientované programován\'{\i}}\label{sec:aop}



\subsection{Motivace}

%\goal{Co je paradigma}
%Programován\'{\i} je komplexn\'{\i} discipl\'{\i}na s teoreticky
%neomezen\'ym počtem možnost\'{\i}, jak\'ym programátor může
%řešit zadan\'y problém. Ačkoliv každá úloha má své specifické
%požadavky, za relativně krátkou historii programován\'{\i} se
%stihlo ustálit několik ideologi\'{\i}, tzv. programovac\'{\i}ch
%paradigmat, které programátorovi poskytuj\'{\i} sadu abstrakc\'{\i}
%a základn\'{\i}ch principů~\cite{van2009programming}.
%D\'{\i}ky znalosti paradigmatu může programátor nejen zlepšit
%svou produktivitu, ale zároveň může snáze pochopit myšlenky
%jiného programátora a t\'{\i}m zlepšit kvalitu t\'ymové spolupráce.

\goal{Nedostatky OOP}
Ačkoliv je \gls{OOP} velmi siln\'ym nástrojem, existuj\'{\i} problémy,
které nelze v jeho rámci efektivně řešit.
Příkladem takového problému jsou obecné požadavky na systém,
které musej\'{\i} b\'yt konzistentně dodržovány na v\'{\i}ce
m\'{\i}stech systému, které spolu zdánlivě nesouvis\'{\i},
tzv. \textit{průřezové problémy} (z anglického \textit{cross-cutting concerns}).
V rámci \gls{OOP} je programátor nucen v ojektech manuálně opakovat
kód, kter\'y zodpov\'{\i}dá za jejich realizaci. Duplikace kódu
vede k větš\'{\i} náchylnosti na lidskou chybu a k vyšš\'{\i}m nárokům na v\'yvoj
a údržbu daného softwarového systému~\cite{fowler1999refactoring}.
Obrázek~\ref{fig:cross-cutting} znázorňuje vzájemné postavení průřezových
problémů a komponent informačního systému.

\begin{figure}[t]
    \centering
    \includegraphics[keepaspectratio=true, width=0.35\linewidth]{figures/cross-cutting.pdf}
    \caption{Průřezové problémy v informačních systémech}
    \label{fig:cross-cutting}
\end{figure}

\goal{Konkrétní příklad nedostatku OOP}
Př\'{\i}kladem průřezového problému může b\'yt logován\'{\i}
systémov\'ych akc\'{\i}, optimalizace správy paměti
nebo jednotné zpracován\'{\i} v\'yjimek~\cite{kiczales1997aspect},
ale i aplikace byznysových pravidel~\cite{cemus2014aspect}.
%Uvažme e-commerce systém a zpracování transakcí. To musí
%být zohledněno jak při vytváření objednávek, tak při registraci
%nového uživatele \textendash\xspace dvě byznysové akce, které by měly
%být podle \gls{OOP} implementovány v naprosto odlišných objektech
%a striktně odděleny, ale část jejich funkcionality je identická,
%a tudíž dochází k duplikaci kódu.
Ve zdrojovém kódu~\ref{lst:tangling} je znázorněno, jak průřezové
problémy zasahují do kódu imaginární třídy implementované v
jazyce Java, která slouží pro vytváření objednávek v e-commerce
systému popsaném v sekci~\ref{sec:shortcomings}.
Aspekt logování je zohledněn na třech místech, stejně jako aspekt transakcí.
Navíc jsou zde zohledněna i byznysová pravidla pro validaci doručovací
a fakturační adresy objednávky.

\lstinputlisting[
caption={Příklad průřezových problémů zohledněných při vytváření objednávky},
label={lst:tangling},
language=Java,
%frame=single,
float,
floatplacement=t
]
{code/tangling.java}

\subsection{Vlastnosti}
Aspektově orientované programován\'{\i} (\gls{AOP}) přináš\'{\i} řešen\'{\i}
v\'yše zmiňovaných problémů. Využívá k tomu \textit{separation of concerns} \textendash\xspace
extrahuje kód zachycující průřezové problémy, tzv. \textit{aspekty}, do jednoho bodu, tzv. (\textit{single focal point}).
Pomoc\'{\i} procesu zvaného \textit{weaving} je poté tento kód automaticky distribuován.
Weaving může proběhnout staticky při kompilaci programu nebo dynamicky
při jeho běhu. V obou př\'{\i}padech ale programátorovi ulehčuje práci,
protože k definici i změně aspektu docház\'{\i} centrálně, a t\'{\i}m je eliminována
potřeba manuáln\'{\i} duplikace kódu. \gls{AOP} nen\'{\i} paradigmatem poskytuj\'{\i}c\'{\i}m
kompletn\'{\i} framework pro návrh programu. V ideáln\'{\i}m př\'{\i}padě je tedy k návrhu
systému využita kombinace \gls{AOP} s jin\'ym paradigmatem.
%Tato práce se zaměř\'{\i}uje na kombinaci \gls{AOP} a \gls{OOP}.

\subsection{Názvosloví}

\paragraph{Aspekt}
Základním pojmem v rámci \gls{AOP} je \textit{aspekt},
který zapozdřuje průřezovou funkcionalitu a zároveň adresuje místa, kde má být
funkcionalita aplikována. Aspekt vždy obsahuje alespoň jeden \textit{advice}
a jeden \textit{pointcut}.

\paragraph{Join-point}
Místo v kódu, na které může být aplikována funkcionalita aspektu, se nazývá
\textit{join-point}. Typů join-pointů je více a závisí na použitém paradigmatu,
na který je \gls{AOP} aplikováno, a také na programovacím jazyce. V případě
kombinace s \gls{OOP} a klasickým víceúčelovým jazykem jako je například Java,
mohou jako join-pointy sloužit konstruktory tříd, volání metod, zápis a čtení
z atributu objektu, inicializace třídy nebo objektu a mnoho dalších.

\paragraph{Pointcut}
Ne každý aspekt je aplikován na každý join-point. Množina join-pointů,
na které je jeden konkrétní aspekt aplikován, se nazývá \textit{pointcut}.
Tato množina může být určena staticky, a být tak známá při kompilaci programu, nebo
dynamicky za běhu programu, což přináší výpočetní složitost navíc.
Dynamické určení pointcutu ale umožňuje vývojářům postihnout i případy,
kdy nelze předem jasně určit místa, kde má být aspekt začleněn.
Příkladem může být zpracování transakcí v zanořených byznysových operacích,
kdy transakce má být započata pouze při vstupu do vnější operace
a dokončena pouze při výstupu z vnější operace.

\paragraph{Advice}
Funkcionalita, kterou aspekt přidává v jeho pointcutu, se nazývá
\textit{advice}. Existuje více typů advice, podle toho, kam je
daná funkcionalita přidána. Například při volání metody může
být funkcionalita přidána před, za, nebo kolem metody.

\paragraph{Weaving}
Proces, kterým jsou advice začleňovány podle pointcutu do
jednotlivých join-pointů se nazývá \textit{weaving}. Ten může
probíhat již při kompilaci nebo dynamicky za běhu programu,
tzv. \textit{run-time weaving}. Proces weavingu je ilustrován
na obrázku~\ref{fig:aspect-weaving}. Komponenta zodpovědná za
weaving se nazývá \textit{aspect weaver}.

\begin{figure}[t]
    \centering
    \includegraphics[keepaspectratio=true, width=0.7\linewidth]{figures/aspect-weaving.pdf}
    \caption{Proces weavingu aspektů}
    \label{fig:aspect-weaving}
\end{figure}

\section{Aspect-driven Design Approach}

Alternativním způsobem návrhu informačních systémů, který staví na principech \gls{AOP},
je Aspect-driven Design Approach\footnote{Autoři nejprve používali termín \textit{Aspect-Oriented
Design Approach} (AODA), který byl později změněn. Oba tyto pojmy jsou vzájemně zaměnitelné.}
(\gls{ADDA})~\cite{cemus2014aspect}, představený v roce 2014.
Tento přístup se zaměřuje na formalizaci jednotlivých komponent informačních systémů identifikování aspektů
v informačních systémech a jejich separaci do \textit{single focal point}.
Následně přístup využívá weaving pro automatickou distribuci aspektů do systému.
K popisu aspektu doporučuje využití doménově specifického jazyka, který bude navržen na
míru danému průřezovému problému.

\subsection{Možnosti aplikace}

Autoři \gls{ADDA} aplikovali tento koncept v několika oblastech \gls{IS}.
Mezi tyto oblasti patří automatické začleňování byznysových pravidel
do datové vrstvy informačních systémů~\cite{cemus2015automated}, automatické
generování uživatelských rozhraní citlivých na kontext uživatele~\cite{cemus2017separation},
validaci vstupů formulářů v uživatelském rozhraní vůči byznysovým pravidlům~\cite{cemus2016context}\cite{cemus2017separation}
a automatické extrakci dokumentace~\cite{cemus2017automated}.

\paragraph{Automatické začleňování byznysových pravidel do datové vrstvy}

Jednou z možných aplikací přístupu \gls{ADDA} je automatické začleňování
byznysových pravidel do datové vrstvy \gls{IS}\footnote{Předpokládáme standardní
třívrstvou architekturu informačních systémů~\cite{fowler2002patterns}}.
Byznysová pravidla jsou nejprve vhodně popsána pomocí \gls{DSL} a následně jsou
extrahována do jednoho bodu, ze kterého jsou automaticky distribuována.
Pomocí specializovaného weaveru jsou pravidla překládána do podmínek
jazyka \gls{JPQL}, potažmo \gls{SQL}, který je využíván k získávání dat
z databázových systémů. To vede ke snížení manuální duplikace byznysových
pravidel.

\paragraph{Automatické generování uživatelského rozhraní}

Uživatelská rozhraní tvoří až 48 \% kódu informačních systému
a zabírají až 50 \% jejich vývojového času~\cite{kennard2009separation}.
Do \gls{UI} se přitom typicky promítá mnoho aspektů, které jsou
již v systému obsaženy, a vývojáři je musí manuálně duplikovat.
Typickým příkladem je struktura datového modelu, která se promítá
zejmána do struktury formulářů sloužících pro manipulaci s daty systému.
Byznysová pravidla jsou promítána do \gls{UI} při validaci vstupních
dat formulářů na straně klienta~\cite{cemus2017separation}.
Dalšími příklady může být lokalizace \gls{UI} do různých jazyků
nebo rozložení a stylizace jednolivých ovládacích prvků, která
je zpravidla uniformní v celém systému.

Autoři přístupu \gls{ADDA} přicházejí s řešením v podobě
využití několika \gls{DSL} pro popis jednotlivých aspektů
a run-time weavingu, který aspekty při běhu aplikace
dynamicky začlení do \gls{UI} s ohledem na aktuální kontext
uživatele, například na jeho geolokační polohu či velikost
displeje, na kterém je rozhraní zobrazováno.
Díky tomu je dosaženo významné redukce kódu~\cite{cemus2016context}
potřebného pro popis adaptibilního uživatelského rozhraní
a tím jsou ušetřeny náklady na vývoj a údržbu informačního
systému využívajícího tento přístup.

\paragraph{Automatická extrakce dokumentace}

Další oblastí informačních systému, do které se promítají jeho aspekty,
je jeho dokumentace~\cite{cemus2017automated}. Autoři \gls{ADDA}
využívají data mining pro získání metainformací o byznysových operacích,
datovém modelu systému a o byznysových pravidlech. Díky tomu mohou
automaticky vygenerovat seznam byznysových operací, potažmo implementovaných
use-cases, strukturu doménového modelu a formální popis byznysových pravidel,
který může sloužit pro verifikaci jejich správnosti.

\subsection{Výhody a nevýhody}

\gls{ADDA} poskytuje vývojářům způsob jakým výrazně snížit náklady na vývoj a údržbu
systému díky deduplikaci, která je dosažena extrakcí aspektů
do \textit{single focal point} a jejich automatickou distribucí do
příslušných komponent systému. Tento přístup však nese vysokou počáteční investici v
podobě vývoje specializovaných \gls{DSL} a dynamických aspect weaverů.
Ačkoliv autoři tohoto přístupu implementovali prototypy knihoven umožňující
požadovanou funkcionalitu, pro nasazení do reálného systému
nejsou tyto knihovny připraveny.

Přístup \gls{ADDA} splňuje požadavky identifikované v sekci~\ref{sec:implementation-requirements},
zejména využití speciálních \gls{DSL} pro popis aspektů a
jejich automatickou distribuci za běhu systému. Pro popis byznysových pravidel
využívá \gls{ADDA} nástroj \textit{Drools}, který je popsán v následující sekci.

\section{Stávaj\'{\i}c\'{\i} řešen\'{\i} reprezentace business pravidel}\label{sec:business-rule-dsl}

Tato kapitola se zaměřuje i na současné možnosti zachycení
byznysových pravidel ve specializovaných jazycích a vhodnost jejich použití
pro účel frameworku, který bude výstupem této práce.
Ačkoliv existuje relativně velké množství knihoven poskytujících
\gls{DSL} pro popis byznysových pravidel a umožňující automatickou
distribuci byznys pravidel, žádný z nich neposkytuje podporu velkého
množství programovacích jazyků, resp. platforem, ve kterých by mohl
být jazyk použit. Příkladem může být projekt \textit{business-ruless}
pro jazyk Python\footnote{https://pypi.org/project/business-rules/}, projekt FlexRule\footnote{http://www.flexrule.com/archives/business-rule-language/} pro platformy .NET a
JavaScript nebo \gls{BRMS} JRules\footnote{https://www.ibm.com/support/knowledgecenter/en/SSZJPZ\_11.3.0/com.ibm.swg.im.iis.conn.jrules.use.doc/topics/c\_jrules\_enginemode.html} od společnosti IBM pro platformu \gls{Java EE}.
Tato sekce se tedy zaměřuje zejména na framework Drools, který
používají autoři přístupu \gls{ADDA}, a také na moderní nástroj
JetBrains MPS, který umožňuje vytvářet vlastní \gls{DSL} a transformovat
ho do libovolných víceúčelových jazyků.

\subsection{Drools DSL}

Framework Drools\footnote{https://www.drools.org/} vyvijený společností JBoss\footnote{http://www.jboss.org/}
je open-source projekt realizující \textit{business rule management engine} (\gls{BRMS}),
tedy nástroj pro vývoj a správu byznysových pravidel. Framework umožňuje realizovat
tzv. \textit{produkční systémy}, tedy systémy tvořené sadou \textit{produkčních pravidel}
určujících chování programu. Tato pravidla obsahují popis situace a její řešení v případě,
že nastane. Tyto systémy tedy poskytují určitou formu umělé inteligence, která simuluje
rozhodování experta na danou doménu.

Produkční pravidlo se skládá z levé strany (\gls{LHS} z anglického \textit{left-hand side}),
a z pravé strany (\gls{RHS} z anglického \textit{right-hand side}),
\gls{LHS} popisuje situaci, při které má být pravidlo aplikováno. \gls{RHS} popisuje akci,
která má být vykonána.

Pro správnou funkci systému je nutno při vyhodnocování správně určit, která produkční pravidla
mají být aplikována. Pro tento účel využívá framework Drools algoritmus RETE~\cite{forgy1988rete},
vynalezený Charlesem Forgym v roce 1983, který je přímo navržený pro párování pravidel produkčních systémů.
Využívá stromové paměťové struktury, která minimalizuje výpočetní složitost na úkor paměťové složitosti. Framework
Drools navíc implementuje některá vylepšení algoritmu optimalizující jeho paměťovou složitost.

Součástí frameworku Drools je speciální doménově specifický jazyk vyvinutý přímo
pro modelování produkčních pravidel. Tento jazyk umožňuje popsat \gls{LHS} i \gls{RHS}
daného pravidla a k tomu využívá několik užitečných konstruktů. Pro popis situací i důsledků
využívá dialekt MVEL umožňující komfortní zápis logických výrazů. V rámci Drools \gls{DSL}
lze využívat lokální i globální proměnné s plnou typovou podporu pramenící z jazyka Java a také
podporu regulárních výrazů. Navíc je možno importovat i pomocné funkce, které lze
využít v podmínkách pravidla.

Ve zdrojovém kódu~\ref{lst:drools-example} je znázorněn příklad zápisu
byznysového pravidla v jazyce Drools DSL. Kromě názvu pravidla je v hlavičce
uvedeno, které dialekty jsou v pravidle využity. Dialekt \code{mvel} je popsán v přecchozím textu.
Dialekt \code{java} umožňuje pro \gls{RHS} využít přímo jazyk Java. Popsané produkční pravidlo
vypíše uživatelovo jméno a email, pokud má uživatel email vyplněný.

\lstinputlisting[
caption={Ukázka zápisu byznysového pravidla v jazyce Drools DSL},
label={lst:drools-example},
language=Drools,
%frame=single,
float,
floatplacement=t
]
{code/drools_example.drl}

Ačkoliv je jazyk Drools \gls{DSL} vymodelovaný přímo pro zápis pravidel doménovými experty,
produkčních pravidla se liší od byznysových pravidel zavedených v sekci~\ref{sec:business-rules},
Využít tak lze pouze \gls{LHS}. Zároveň jazyk Drools \gls{DSL} postrádá
nástroje pro kvalitní popis byznysového kontextu držícího byznysová pravidla,
zejména pak rozšiřování jiných kontextů a popis typu jednotlivých pravidel~\cite{cemus2017automated}.
V neposlední řadě nejsou ze strany frameworku Drools podporovány jiné platformy než
Java a .NET, což nevyhovuje požadavkům na platformovou nezávislost.

\subsection{JetBrains MPS}

Moderním nástrojem z dílny společnosti JetBrains\footnote{https://www.jetbrains.com/}
je tzv. \textit{MPS \textendash\xspace Meta Programming System}\footnote{https://www.jetbrains.com/mps/}, který si klade za úkol být univerzálním
nástrojem pro tvorbu doménově specifických jazyků. Staví na konceptu \textit{language-oriented
programming} (\gls{LOP})~\cite{ward1994language} zaměřujícího se na vývoj velmi specifického jazyka,
který je následně použit pro implementaci programu namísto obecného mnohoúčelového jazyka. Pro překlad
ze specifického jazyka do spustitelného kódu je použit automatický překladač. Příkladem jazyka, který využívá koncept \gls{LOP}
je \LaTeX\xspace, který byl využit pro sazbu této diplomové práce. Ten totiž pomocí maker jazyka \TeX\xspace
sestavuje abstraktnější jazyk, který umožňuje autorovi soustředit se hlavně na strukturu textu, aniž by
se musel příliš detailně zaobírat samotnou sazbou.

MPS umožňuje uživateli nadefinovat gramatiku speciálního \gls{DSL} a následně poskytuje
editor pro tento jazyk včetně automatického validátoru. MPS podporuje také transformování kódu napsaného
v nadefinovaném jazyce do jiných, nízkoúrovňovějších jazyků, zejména pak do jazyka Java.
Díky tomu lze nejen vytvářet libovolné \gls{DSL}, ale také rozšiřovat existující
jazyky \textendash\xspace například přidat do jazyka Java podporu pro jednoduchý vizuální
zápis matic, jak můžeme vidět na obrázku~\ref{fig:java-matrix}.

\begin{figure}[t]
    \centering
    \includegraphics[keepaspectratio=true, width=0.7\linewidth]{figures/java-matrix.png}
    \caption{Rozšíření jazyka java o zápis matic pomocí JetBrains MPS}
    \label{fig:java-matrix}
\end{figure}

Výhoda tohoto přístupu je podobně jako u \gls{MDA} vysoká úroveň abstrakce, která
velmi dobře odděluje problém návrhu systému od implementačních problémů.
Navíc lze díky použití \gls{DSL} zapojit do vývoje doménové experty a snížit tak
zátěž na programátory, kteří se mohou více věnovat právě implementačním problémům.
\gls{DSL} typicky zvyšuje expresivitu kódu a díky tomu se zmenšuje jeho objem.
Nižší objem kódu vede ke snížení nákladů na jeho údržbu a vývoj~\cite{littman1987mental}\cite{soloway1986empirical}.
Významnou výhodou MPS, potažmo \gls{LOP}, je vysoká portabilita vyvinutého jazyka.
Pro migraci na jinou platformu stačí doprogramovat překladač jazyka, ale
samotný jazyk může zůstat zachovaný.

Nástroj MPS lze využít k popisu byznysových pravidel, resp. byznysových kontextů.
To by umožnilo snadné znovupoužití pravidel a jejich transformaci do neomezeného počtu jazyků pro
využití na mnoha platformách. Podobně jako u \gls{MDA} je však problém v dopředném
generování \textendash\xspace editor MPS totiž neumožňuje načíst víceúčelový jazyk zpět do \gls{DSL}.


\section{Síťové architektury}

Závěrem se tato kapitola věnuje přehledu síťových architektur, které mohou být využity pro
distribuci byznysových pravidel v systému stavějícímu na \gls{SOA}.

\subsection{Architektura klient-server}\label{sec:client-server}

Model klient-server popisuje vztah mezi komponentami systému, klienty a serverem.
Klient zašle požadavek serveru a ten mu vrátí odpověď~\cite{berson1992client}.
Schéma komunikace je znázorněno na obrázku ~\ref{fig:client-server}.
Tento model může být použit obecně i v rámci jednoho počítače,
nejčastěji je však využíván v síťové komunikaci mezi více počítači.
%V tomto případě klient sestaví síťové spojení k serveru a po získání odpovědi
%od serveru spojení zase uzavře.

\begin{figure}[t]
    \centering
    \includegraphics[keepaspectratio=true, width=0.6\linewidth]{figures/client-server.pdf}
    \caption{Architektura klient-server}
    \label{fig:client-server}
\end{figure}

%Tato architektura je jednou ze základních stavebních kamenů
%internetových protokolů. Využívá ji zejména protokol
%\gls{TCP}~\cite{postel1981transmission}, který je hlavním
%komunikačním protokolem v síti Internet. Jako příklad si můžeme představit
%prohlížení internetových stránek. Uživatel zadá \gls{URL} adresu
%stránky, kterou chce navštívit, a internetový prohlížeč, potažmo uživatelův osobní počítač,
%v roli klienta odešle požadavek na server nacházející se na dané adrese.
%Server požadavek přijme, zpracuje, a odešle odpověď obsahující
%tělo webové stránky. Klient stránku přijme a zobrazí pro koncového
%uživatele.

Tento přístup má několik zásadních výhod, díky kterým se stal
široce využívaným. Díky svojí velmi obecné myšlence je nezávislý
na jakékoliv platformě.
%a jako klient i server mohou sloužit
%jak vysoce výkonné počítače, tak i osobní počítače nebo chytré telefony,
%z nichž každý může využívat odlišné operační systémy –
%stačí aby klient i server uměl komunikovat stejným protokolem.
Zároveň tato architektura přesouvá byznysovou logiku a
ukládání dat na server a díky tomu umožňuje
snadnější kontrolu nad systémem a jeho centrální administraci. S tím
je spojena i snažší škálovatelnost systému. V neposlední řadě
přináší model klient-server díky centralizaci i lepší zabezpečení,
kdy server může jasně definovat a vynucovat přístupová pravidla.

Hlavní nevýhodu této architektury je vytvoření jednoho centrálního bodu,
jehož výpadek ochromí funkci celého systému (v angličtině
\textit{single point of failure}) \textendash\xspace tímto bodem je server.
Pokud na serveru nastane chyba či výpadek, žádný z klientů není schopen využívat
jeho služeb.

%Pro sdílení byznysových pravidel se tato architektura jeví
%jako vhodná. Klient, který potřebuje byznysové pravidlo ke
%své funkci, by zažádal server o dané pravidlo a po jeho získání by
%se postaral o jeho spuštění. Tím by bylo dosaženo automatické
%distribuce a integrace byznysových pravidel. Vzhledem k tomu,
%že server je v této architetkuře centrální autoritou, usnadnila
%by se tak centrální správa byznysových pravidel.

\subsection{Architektura Peer-to-peer}\label{sec:p2p}

Opakem modelu klient-server je síťová architektura zvaná \textit{Peer-to-peer (\gls{P2P})}.
Jednotlivé počítače v síti spolu komunikují přímo, bez centrální autority.
Všechny počítače v síti jsou si vzájemně rovnocenné.~\cite{fox2001peer}
Na obrázku~\ref{fig:peer-to-peer} je tato architektura znázorněna.
Hlavním cílem \gls{P2P} sítí je distribuce dat nebo výpočetních operací.

\begin{figure}[t]
    \centering
    \includegraphics[keepaspectratio=true, width=0.4\linewidth]{figures/peer-to-peer.pdf}
    \caption{Architektura peer-to-peer}
    \label{fig:peer-to-peer}
\end{figure}

%Jednotliví klienti mezi sebou zpravidla vytvářejí virtuální síť, tzv. \textit{overlay},
%která je postavená nad fyzickou sítí, přes kterou jsou reálně fyzicky zasílány zprávy mezi klienty.
%Typicky je tato virtuální síť podmnožinou existující fyzické sítě. Výhodou tohoto přístupu je,
%že klienti jsou abstrahováni od fyzického uspořádání počítačů a mohou spolu komunikovat napřímo,
%i když mezi nimi mohou reálně být v síti zapojeny jiné počítače.

Výhodou architektury \gls{P2P} rostoucí kapacita a výkon sítě s rostoucím počtem klientů,
narozdíl od modelu klient-server, kdy se klienti musí dělit o výkon serveru.
Navíc v takové síti neexistuje \textit{single point of failure} a tak se zvyšuje její robustnost.
Mezi nevýhody této architektury patří zvýšená bezpečnostní rizika způsobená tím,
že klienti jsou otevřeni komunikaci s jakýmkoliv jiným, potenciálně nebezpečným, klientem.
Další nevýhodou může být absence jakékoliv centrální správy sdílených dat.

Díky vysoké datové propustnosti a robustnosti se \gls{P2P} může jevit jako vhodný přístup pro
sdílení byznysových pravidel. Absence centrální správy by však mohla způsobit nekonzistentní stavy
systému při úpravě či přidání byznysového pravidla. To je způsobeno distribucí informací mezi více uzlů sítě.
Změna pravidla by se musela šířit postupně napříč systémem, přičemž některé uzly by stále využívaly starou
verzi pravidla. Nad samotným šířením nelze získat přímou kontrolu.

\subsection{Representational state transfer}\label{sec:rest}

Representational state transfer~\cite{fielding2000rest} (\gls{REST}) je architektura
webových služeb, která staví na protokolu \gls{HTTP}, a klade na systém
několik architektonických omezení, díky kterým může systém dosáhnout
lepšího výkonu, vyšší škálovatelnosti, jednoduchému používání díky jednotnému rozhraní
a lepší odolnosti vůči chybám.

\gls{REST} chápe data systému jako množinu zdrojů (z anglického \textit{resources}),
nad kterými jsou prováděny operace pomocí \gls{HTTP} požadavků. K odlišení operací
nad jedním zdrojem jsou využívána slovesa protokolu \gls{HTTP}, zejména pak \code{GET} pro čtení,
\code{POST} pro vytváření, \code{PUT} pro úpravu a \code{DELETE} pro mazání. Tím jsou zastřešeny
všechny \gls{CRUD} operace.

\begin{figure}[t]
    \centering
    \includegraphics[keepaspectratio=true, width=0.9\linewidth]{figures/rest-statelessness.pdf}
    \caption{Znázornění architektury \gls{REST}}
    \label{fig:rest-statelessness}
\end{figure}

Principy architektury \gls{REST} jsou:

\begin{description}
    \item [Klient-server] Systém by měl využívat model klient-server. Díky tomu může jasně oddělit
    zodpovědnost za uživatelské rozhraní na klienta a zodpovědnost za ukládání dat na server. To zvyšuje
    škálovatelnost systému díky nižším nárokům na server.
    \item [Bezstavovost] Každý požadavek na server by měl obsahovat všechny informace potřebné k jeho vykonání.
    Kromě těla \gls{HTTP} požadavku se k tomuto účelu často využívají i hlavičky požadavku,
    například hlavička \code{Authorization} pro autentizaci uživatele kvůli přístupu k zabezpečeným zdrojům.
    Tím se zjednodušuje komplexita serveru, který plně přesouvá zodpovědnost za uchování stavu uživatelského rozhraní
    na klienta, jak je znázorněno na obrázku~\ref{fig:rest-statelessness}. Nutno poznamenat, že stav dat v systému
    je stále uchováván na serveru.
    \item [Kešování] Odpovědi serveru musí obsahovat explicitní informaci o tom, zda lze odpověď uložit do cache.
    Díky tomu je možné znovupoužívat data, která již server jednou vrátil, a jejich životnost má dlouhodobější charakter.
    Tím se zvyšuje výkon celého systému.
    \item [Vrstvený systém] Klient by neměl mít možnost rozeznat, zda komunikuje přímo se serverem, nebo s prostředníkem,
    např. s proxy serverem, load balancerem nebo cache.
    \item [Code-on-demand] Volitelným požadavkem na systém je tzv. \textit{code-on-demand}, který umožňuje serveru vracet
    spustitelný kód jako odpověď. Klient kód poté spustí na své straně. Typickým příkladem jsou klientské JavaScriptové aplikace spouštěné ve webových prohlížečích.
    \item [Jednotné rozhraní] Zdroje systému musejí mít unikátní identifikátor, např. \gls{URI}. Zdroje jsou při komunikaci reprezentovány libovolným formátem, který
    se může lišit od interní reprezentace zdroje v programu, např. \gls{JSON} či \gls{XML}. Reprezentace zdroje musí být
    dostatečná k tomu, aby šlo na zdroji provést úpravu či smazání. Zprávy mezi klientem a serverem musejí obsahovat veškerá potřebná metadata,
    aby druhá strana mohla zprávu plně zpracovat. K tomu se používají například \gls{HTTP} hlavičky \code{Content-type}
    či \code{Accept}, ve kterých je popsán typ reprezentace zdroje.
    Rozhraní by také mělo dodržovat koncept \textit{Hypermedia as the engine ot application state (\gls{HATEOAS})},
    který vyžaduje, aby server v odpovědích vracel metainformace o struktuře jeho \gls{API}.
    Klient je tak schopen dynamicky navigovat v rozhraní serveru aniž by bylo vyžadováno předem znát přesné adresy zdrojů.
    Princip \gls{HATEOAS} se však ve většině reálných \gls{API} zanedbává či implementuje pouze částečně.
\end{description}

Nevýhodou architektury \gls{REST} je náročná implementace transakcí, které zahrnují více
zdrojů najednou. Protokol \gls{HTTP} nepodporuje uzavření více požadavků do jedné atomické
transakce. To může být problém v \gls{SOA} zejména pokud je vyžadována kooperace více služeb
najednou při vykonávání byznysové operace. Existují však koncepty, které využívají model
Try-Cancel/Confirm~\cite{pardon2011towards}, umožňující zajistit atomické transakce nad
\gls{REST} architekturou. Další nevýhodou je relativně náročná implementace samotné architektury
kvůli neexistujícímu obecnému middleware, který by poskytoval

\subsection{Remote procedure call}\label{sec:rpc}

Remote procedure call (\gls{RPC}) je podstatně starší architekturou než \gls{REST}.
Tento termín byl použit již v roce 1981 Brucem Nelsonem~\cite{nelson1981remote}.
Architektura staví na modelu klient-server a umožňuje jednomu procesu (klientovi)
zavolat proceduru na druhém, vzdáleném procesu (serveru).
Klient zašle serveru zprávu vyžadující zavolání specifické procedury. Server
proceduru provede a po jejím dokončení zašle klientovi odpověď s návratovou hodnotou.
Klient poté může pokračovat ve své práci.

\gls{RPC} zapouzdřuje síťovou komunikaci a v programu samotném
je vzdálená procedura volána stejným způsobem jako lokální procedury. Základním
prvkem architektury na klientovi i na serveru je tzv. \textit{stub}. Tato komponenta
umožňuje volat, resp. obsloužit, vzdálenou proceduru lokálně a zapozdřuje veškerou
síťovou komunikaci a serializaci či deserializaci argumentů, resp. návratových hodnot.
Schéma komunikace je znázorněno na obrázku~\ref{fig:rpc}.

\begin{figure}[t]
    \centering
    \includegraphics[keepaspectratio=true, width=0.7\linewidth]{figures/rpc.pdf}
    \caption{Schéma komunikace \gls{RPC}}
    \label{fig:rpc}
\end{figure}

%Představitelem architektury \gls{RPC} je například technologie \gls{CORBA},
%která již byla popsána v sekci~\ref{sec:corba}, nebo technologie Remote
%Method Invocation (\gls{RMI}) v jazyce Java. Modernějším pojetím je technologie gRPC~\cite{grpcio}
%od společnosti Google\footnote{https://www.google.com/},
%která je v dnešní době hromadně využívána úspěšnými technologickými společnostmi.

Nevýhodou abstrakce lokálních a vzdálených volání jsou
negativní vlastnosti síťové komunikace, její zvýšená latence
a nižší robustnost. Pokud programátor nemá možnost zjistit, zda volá
lokální či vzdálenou proceduru, výsledný kód může být těžké optimalizovat
a správně ošetřit výjimky, které mohou při jeho běhu nastat.
Pro \gls{RPC} však není potřeba implementovat komplexní middleware obstarávající
síťovou komunikaci, serializaci a zpracování chyb. Middleware je zpravidla dodáván v
podobě knihoven dané technologie. \gls{RPC} stejně jako \gls{REST}
ani \gls{RPC} nedefinuje, jakým způsobem by měly být obslouženy transakce.

\section{Shrnut\'{\i}}

Tato kapitola popisuje rešerši \textit{modelem řízené architektury},
\textit{generativního programování}, \textit{BPEL},
jej\'{\i}ch v\'yhody a nev\'yhody. Shrnuje existující síťové architektury,
které mohou být využity pro komunikaci služeb v architektuře \gls{SOA}
a zvažuje vhodnost jejich použití pro účely této práce.
Kapitola dále shrnuje jsme paradigma \textit{aspektově orientovaného programován\'{\i}} a
věnuje se inovativnímu přístupu k návrhu softwarov\'ych systémů \textit{ADDA}.
Nakonec se kapitola věnuje rešerši stávaj\'{\i}c\'{\i}ch řešen\'{\i} reprezentace byznys pravidel
včetně komplexn\'{\i}ho frameworku \textit{Drools} a hodnotí jejich vhodnost pro použití v této práci.

\usepackage[T1]{fontenc}
\usepackage[utf8]{inputenc}

%!TEX ROOT=../diploma-thesis.tex

\chapter{Návrh}\label{ch:navrh}

\section{Formalizace architektury orientované na služby}

\subsection{Join-points}

\subsection{Advices}

\subsection{Pointcuts}

\subsection{Weaving}

\section{Architektura frameworku}

\section{Zachycení byznysového konextu}

Přístup ADDA doporučuje popsat byznysová pravidla pomocí
vlastního, na míru šitého, doménově specifického jazyka~\cite{cemus2015automated}.
V našem případě můžeme jazykem DSL popsat kompletně i celý
byznysový kontext.

\section{Metamodel byznys kontextu}\label{sec:metamodel}

\section{Expression}

\section{Registr byznys kontextů}

\section{Byznys kontext weaver}

\section{Centrální správa byznys kontextů}

\subsection{Uložení rozšířeného pravidla}

\goal{Diskutovat chaining vs. direct update}
% TODO: napiš mě

\section{Service discovery}

\goal{Popsat nezávislost na service discovery}

\usepackage[T1]{fontenc}
\usepackage[utf8]{inputenc}

%!TEX ROOT=../diploma-thesis.tex

\chapter{Implementace prototypů knihoven}\label{ch:implementace}

Součástí zadání této práce je implementace prototypů
knihoven pro framework navržený v kapitole~\ref{ch:navrh}
pro tři rozdílné platformy, z nichž jedna musí být \textit{Java}.
V této kapitole si popíšeme jaké plaformy jsme vybraly a jakým
způsobem byly prototypy knihoven implementovány. Součástí
kapitoly je i stručná rešerše technologií, které byly použity
pro dosažení vytyčených cílů.

% Technické implementační problémy
Pro splnění cílů bylo potřeba vyřešit také několik technických otázek.
Těmi je přenos byznys kontextů mezi jednotlivými službami, podpora
aspektově orientovaného programování v daném programovacím jazyce
a využití principu \textit{runtime weavingu} a integrace knihoven
do služeb, které je budou využívat.

\section{Výběr použitých platforem}

Mimo jazyk Java, který byl určen zadáním, byla pro
implementaci vybrána platforma jazyka \textit{Python}
a platforma \textit{Node.js}, který slouží jako
běhové prostředí pro jazyk \textit{JavaScript}.
Výběr byl proveden na základě aktuálních trendů
ve světě softwarového inženýrství. Projekt GitHut\footnote{http://githut.info/}
z roku 2014, který shrnuje statistiky repozitářů
populární služby pro hosting a sdílení kódu
GitHub\footnote{https://github.com/}, určil
jazyky JavaScript, Java a Python jako tři nejaktivnější.
Služba GitHub následně sama zveřejnila statistiky za rok 2017
v rámci projektu Octoverse\footnote{https://octoverse.github.com/}
a dospěla ke stejnému závěru, ačkoliv Python se umístil na druhé
pozici na úkor jazyka Java. Podle průzkumu oblíbeného
programátorského webového portálu Stack
Overflow\footnote{https://insights.stackoverflow.com/survey/2017\#technology}
se umístily tyto jazyky v první čtveřici nejpopulárnějších jazyků pro obecné použití.

\section{Sdílení byznys kontextů mezi službami}

% Přenášení business contextů

Pro sdílení byznys kontextů mezi jednotlivými službami
je potřeba je přenášet po síti ve formátu, který bude
nezávislý na platformách jednotlivých služeb.

\subsection{Síťová komunikace}

% Formát pro přenos pravidel po síti a jeho výhody
Abychom mohli přenášet byznysové kontexty a jejich pravidla
po síti, musíme zvolit protokol a jednotný formát, ve kterém
spolu budou jednotlivé služby komunikovat.
Tento formát tedy musí být nezávislý na platformě a ideálně
by měl být co nejefektivnější v rychlosti přenosu.

\subsection{Použité technologie}

\subsubsection{Protocol Buffers}


Pro přenos byznysových kontextů byl zvolen open-source formát
\textit{Protocol Buffers}\footnote{
https://developers.google.com/protocol-buffers/
}
vyvinutý společností Google\footnote{
https://www.google.com/
}. Umožňuje explicitně definovat a vynucovat schéma dat,
která jsou přenášena po síti, bez vazby na konkrétní programovací
jazyk. Zároveň je díky binární reprezentaci v přenosu velmi efektivní,
oproti formátům jako je JSON nebo XML~\cite{varda2008protocol}.
Zdrojový kód~\ref{lst:protobuf-example} znázorňuje zápis schématu
zasílaných zpráv ve formátu Protobuffer.

\lstinputlisting[caption={Příklad definice schématu zpráv v jazyce Protobuffer},label={lst:protobuf-example},language=Protobuf, frame=single]{code/protobuffer_example.proto}


\subsubsection{gRPC}

Pro komunikaci byznys kontextů nám nestačí pouze přenosový formát,
je potřeba také popsat schéma samotné komunikace. K tomu byl zvolen
open-source framework gRPC\footnote{
https://grpc.io/
}, který staví na technologii Protocol Buffers a poskytuje vývojáři
možnost definovat komunikaci pomocí protokolu \textit{RPC}. % TODO: citovat článek o RPC?

\lstinputlisting[caption={gRPC},label={lst:grpc-example},language=Protobuf, frame=single]{code/grpc_example.proto}

\section{Knihovna pro platformu Java}

\subsection{Použité technologie}

\subsubsection{Apache Maven}

Pro správu závislostí a automatickou kompilaci a sestavování
knihovny napsané v jazyce java byl zvolen projekt \textit{Maven}. %TODO: footnote
Tento nástroj umožňuje vývojáři komfortně a centrálně
spravovat závislosti jeho projektu včetně detailního
popisu jejich verze. Dále také umožňuje definovat jakým
způsobem bude projekt kompilován.

\subsubsection{AspectJ}

Knihovna AspectJ přináší pro jazyk Java sadu nástrojů,
díky kterým lze snadno implementovat koncepty aspektově orientovaného
programování.

% TOOO: ukázka AspectJ v akci

\section{Knihovna pro platformu Python}

\subsection{Použité technologie}

\subsection{}

\section{Knihovna pro platformu Node.js}


\subsection{Použité technologie}

\subsection{NPM a Yarn}

% TODO: fuj, opravit
Podobně jako byl použit nástroj Maven pro knihovnu v jazyce Java byl
využit balíčkovací nástroj \textit{NPM}, který je předinstalován
v běhovém prostředí \textit{Node.js}. Dále byl využit nástroj
\textit{Yarn}\footnote{http://yarnpkg.com} % TODO: opravit footnote

\section{Doménově specifický jazyk pro popis byznys kontextů}

\goal{Popsat proč a jak jsme tvořili DSL}
Ačkoliv není specifikace a vytvoření doménově specifického jazyka (DSL)
hlavním úkolem této práce, pro ověření konceptu bylo nutné nadefinovat
alespoň jeho zjednodušenou verzi a implementovat část knihovny, která
bude umět jazyk zpracovat a sestavit z něj byznysový kontext v paměti programu.

\goal{Důvody pro výběr XML}
Pro popis kontextů byl jako kompromis mezi jednoduchostí implementace
a přívětivostí pro koncového uživatele zvolen univerzální formát Extensible
Markup Language (XML)~\cite{bray1997extensible}. Tento
jazyk umožňuje serializaci libovolných dat, přímočarý a formální
zápis jejich struktury a také jejich snadné aplikační zpracování.
Zároveň poskytuje relativně dobrou čitelnost pro člověka, ačkoliv
speciálně vytvořené DSL by bylo jistě čitelnější.

\goal{Popis jak XML funguje}
Dokumenty XML se skládají z tzv. \textit{entit}, které obsahují
buď parsovaná nebo neparsovaná data. Parsovaná data se skládají
z jednoduchých znaků reprezentujících jednoduchý text a nebo
speciálních značek, neboli \textit{markup}, které slouží k popisu
struktury dat. Naopak neparsovaná data mohou obsahovat libovolné
znaky, které nenesou žádnou informaci o struktuře dat.

\goal{Popis jaký formát jsme zvolili pro formální zápis schématu XML dokumentu}
Vzhledem k tomu, že XML je volně rozšiřitelný jazyk a neklade
meze v možnostech struktury dat, bylo potřeba jasně definovat
a dokumentovat očekávanou strukturu dokumentu popisujícího
byznys kontext. Pro jazyk XML existuje vícero možností jak schéma
definovat~\cite{lee2000comparative}, od jednoduchého formátu
\textit{DTD} až po komplexní formáty jako je \textit{Schematron}, či
\textit{XML Schema Definition} (XSD), který byl nakonec zvolen.
Díky formálně definovanému schématu můžeme popis byznys kontextu
automaticky validovat a vyvarovat se tak případných chyb.

\goal{Popis formátu}
Ve zdrojovém kódu~\ref{lst:business-context-xml} můžeme vidět
příklad zápisu jednoduchého byznys kontextu s jednou precondition.
Samotný zápis byznys kontextu je obsažen v kořenovém elementu
\code{<businessContext>} a jeho název je popsán atributy
\code{prefix} a \code{name}. Rozšířené kontexty jsou vyčteny
v entitě \code{<includedContexts>}. Preconditions jsou
definovány uvnitř entity \code{<preconditions>} a podobně
jsou definovány \code{<postconditions>}. Obsažená data odpovídají
navrženému metamodelu byznysového kontextu z kapitoly~\ref{ch:navrh}.
Pro zápis podmínek jednotlivých preconditions a post-conditions byl zvolen
opis Expression AST. Toto rozhodnutí vychází z předpokladu,
že lze vzhledem k povaze prototypu relaxovat podmínku
na čitelnost zápisu pravidel ve prospěch jednoduššího zpracování.

\goal{Shrnutí DSL}
Podařilo se nám navrhnout přijatelný formát zápisu byznys kontextu
a implementovat části knihoven, které umějí formát číst a zároveň vytvářet.
Tím jsme dosáhli možnosti zapisovat kontexty bez ohledu na platformu
služby, která je bude využívat. Zároveň tomuto formátu mohou
snáze porozumět doménoví experti a mohou se tak zapojit do
vývojového procesu.

\lstinputlisting[caption={Příklad zápisu byznys kontextu v jazyce XML},label={lst:business-context-xml},language=XML, frame=single, float, floatplacement=H]{code/business_context.xml}

\section{Systém pro centrální správu byznys pravidel}

\subsection{Použité technologie}

\subsubsection{Spring Boot}

\subsection{Detekce cyklů}

Při úpravě nebo vytváření nového byznysového kontextu je
potřeba detekovat případné chyby, abychom změnou neuvedli
systém do nekonzistentního stavu. Kromě syntaktických chyb,
které jsou detekovány automaticky pomocí definovaných pravidel
a schemat ... % TODO: dopsat

\section{Shrnutí}

\goal{Hostování na GitHubu + licence}
Veškerý kód je hostován v centrálním repozitáři
ve službě GitHub\footnote{
https://github.com/klimesf/diploma-thesis
} a je zpřístupněn pod open-source licencí MIT\footnote{
http://www.linfo.org/mitlicense.html
}. Knihovny pro jednotlivé platformy tedy lze libovolně
využívat, modifikovat a šířit.

%!TEX ROOT=../diploma-thesis.tex

\chapter{Verifikace a validace}\label{ch:verifikace}

\section{Testování naprogramované knihovny}

\subsection{Platforma Java}

\subsection{Platforma Python}

\subsection{Platforma Node.js}

\section{Případová studie: e-commerce systém}

\subsection{Model systému}

\subsection{Use-cases}

\subsection{Byznys kontexty}

\subsection{Service discovery}

\subsection{Order service}

\subsection{Product service}

\subsection{User service}

\subsection{Nasazení systému pro centrální správu byznys kontextů}

\section{Shrnutí}

\usepackage[T1]{fontenc}
\usepackage[utf8]{inputenc}

%!TEX ROOT=../diploma-thesis.tex

\chapter{Závěr}\label{ch:zaver}

\section{Analýza dopadu použití frameworku}

\section{Budoucí rozšiřitelnost frameworku}

\section{Možností uplatnění navrženého frameworku}

\section{Další možnosti uplatnění \gls{AOP} v \gls{SOA}}

\todo{
\begin{itemize}
    \item{Extrakce dokumentace}
    \item{Extrakce byznysového modelu}
    \item{Konfigurace prostředí}
\end{itemize}
}

\section{Shrnutí}

\todo{
\begin{itemize}
    \item{Dosáhli jsme cílů práce}
    \item{Stručné shrnutí co všechno a jak jsme udělali}
\end{itemize}
}


\bibliographystyle{csplainnat}

%bibliographystyle{plain}
%\bibliographystyle{psc}
{
%JZ: 11.12.2008 Kdo chce mit v techto ukazkovych odkazech take odkaz na CSTeX:
\def\CS{$\cal C\kern-0.1667em\lower.5ex\hbox{$\cal S$}\kern-0.075em $}
\bibliography{reference}
}

\appendix

%!TEX ROOT=../diploma-thesis.tex

\chapter{Snímky ukázkové aplikace a centrální administrace}

\begin{figure}[H]
    \centering
    \includegraphics[width=0.9\linewidth]{figures/business-context-detail.png}
    \caption{Detail byznysového kontextu v centrální administraci}
    \label{fig:screenshot-context-detail}
\end{figure}

\begin{figure}[H]
    \centering
    \includegraphics[width=0.9\linewidth]{figures/business-context-edit.png}
    \caption{Formulář pro vytvoření nebo úpravu byznysového kontextu v centrální administraci}
    \label{fig:screenshot-context-edit}
\end{figure}

%%\usepackage[T1]{fontenc}
%\usepackage[utf8]{inputenc}

%!TEX ROOT=../diploma-thesis.tex

%\chapter{Zdrojové kódy}
%
%\lstinputlisting[
%caption={Příklad definice služeb v gRPC},
%label={lst:grpc-example},
%language=Protobuf,
%frame=single,
%]
%{code/business_context.proto}

%!TEX ROOT=../diploma-thesis.tex

\renewcommand*{\glsnamefont}[1]{\textbf{#1}}
\setlength{\glsdescwidth}{0.95\linewidth}
\printglossary[type=\acronymtype,title={Seznam použitých zkratek}]\label{ch:shortcuts}

\usepackage[T1]{fontenc}
\usepackage[utf8]{inputenc}

%!TEX ROOT=../diploma-thesis.tex

\chapter{Obsah přiloženého CD}

% TODO: přepsat na DP

\begin{verbatim}
    |-- nutfroms-example/       Ukázkový systém využívající knihovnu
    |  |-- dist/                Zkompilované zdrojové soubory pro distribuci
    |  |-- docs/                Dokumentace
    |  |-- src/                 Zdrojový kód aplikace
    |
    |-- nutforms-ios-client/    Klientská část knihovny pro platformu iOS
    |  |-- client/              Zdrojové soubory knihovny
    |  |-- clientTests/         Zdrojové soubory testů knihovny
    |  |-- dist/                Zkompilované zdrojové soubory pro distribuci
    |  |-- docs/                Dokumentace
    |
    |-- nutfroms-server/        Serverová část knihovny
    |  |-- dist/                Zkompilované zdrojové soubory pro distribuci
    |  |-- docs/                Dokumentace
    |  |-- layout/              Layout servlet
    |  |-- localization/        Localization servlet
    |  |-- meta/                Metadata servlet
    |  |-- widget/              Widget servlet
    |
    |-- nutforms-web-client/    Klientská část knihovny pro webové aplikace
    |  |-- dist/                Zkompilované zdrojové soubory pro distribuci
    |  |-- docs/                Dokumentace
    |  |-- src/                 Zdrojové soubory knihovny
    |  |-- test/                Zdrojové soubory testů knihovny
    |
    |-- text/                   Text bakalářské práce
\end{verbatim}


\end{document}
