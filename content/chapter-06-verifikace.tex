\usepackage[T1]{fontenc}
\usepackage[utf8]{inputenc}

%!TEX ROOT=../diploma-thesis.tex

\chapter{Verifikace a validace}\label{ch:verifikace}

V této kapitole %TODO: co tady?

\section{Testování prototypů knihoven}

Prototypy knihoven, jejichž implementaci jsme popsali
v kapitole~\ref{ch:implementace}, byly také důkladně
otestovány pomocí sady jednotkových a integračních testů.

V rámci konceptu \textit{continous integration}~\cite{fowler2006continuous}
byl kód po celou dobu vývoje zasílán do centrálního repozitáře
a s pomocí nástroje Travis CI\footnote{https://travis-ci.org/}
bylo automaticky spouštěno jeho sestavení a otestování. Systém
zároveň okamžitě informoval vývojáře o jejich výsledcích. To
umožnilo v krátkém časovém horizontu identifikovat konkrétní změny
v kódu, které do programu vnesly chybu. Tím byla snížena
pravděpodobnost regrese a dlouhodobě se zvýšila celková kvalita kódu.

\subsection{Platforma Java}

Prototyp knihovny pro platformu byl testován pomocí
nástroje JUnit\footnote{https://junit.org/junit4/},
který poskytuje všechny potřebné funkce.

\subsection{Platforma Python}

\subsection{Platforma Node.js}

\section{Případová studie: e-commerce systém}

\subsection{Model systému}

\subsection{Use-cases}

\subsection{Byznys kontexty}

\subsection{Service discovery}

\subsection{Order service}

\subsection{Product service}

\subsection{User service}

\subsection{Nasazení systému pro centrální správu byznys kontextů}

\section{Shrnutí}
