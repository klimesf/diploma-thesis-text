\usepackage[T1]{fontenc}
\usepackage[utf8]{inputenc}

%!TEX ROOT=../diploma-thesis.tex

\chapter{Verifikace a validace}\label{ch:verifikace}

\section{Testování naprogramovaných knihoven}

\subsection{Platforma Java}

\subsection{Platforma Python}

\subsection{Platforma Node.js}

\subsection{Continuous Integration}

Po celou dobu vývoje jednotlivých knihoven byl využíván nástroj
Travis CI\footnote{
https://travis-ci.org/
}, který po každém přidáním kódu do repozitáře spouštěl automatizované
testy a informoval vývojáře o jejich výsledcích. To umožnilo okamžitě
identifikovat změny, které do kódu vnesly chybu.
% TODO: co jsem tim vlastně chtěl říct?

\section{Případová studie: e-commerce systém}

\subsection{Model systému}

\subsection{Use-cases}

\subsection{Byznys kontexty}

\subsection{Service discovery}

\subsection{Order service}

\subsection{Product service}

\subsection{User service}

\subsection{Nasazení systému pro centrální správu byznys kontextů}

\section{Shrnutí}
