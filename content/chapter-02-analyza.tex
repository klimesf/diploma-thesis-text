\usepackage[T1]{fontenc}
\usepackage[utf8]{inputenc}

%!TEX ROOT=../diploma-thesis.tex

\chapter{Analýza}\label{ch:analyza}

\section{Byznysový kontext}

\paragraph{Precondition} % TODO: popiš mě

\paragraph{Post-condition} % TODO: popiš mě

\section{Architektura orientovaná na služby}

\goal{Úvod do SOA, proč je potřeba}
V posledních dekádách můžeme sledovat trend nárůstu komplexity
moderních informačních systémů, který je způsoben stále náročnějšími
požadavky na jejich funkcionalitu, výkon a spolehlivost. To nutí
vývojáře těchto systémů přizpůsobovat architekturu systému tak,
aby uměla splnit všechny očekávané funkční i nefunkční požadavky,
zejména pak škálovatelnost systému a jeho schopnost zvládat vysoký
objem dat a uživatelů. \textit{Architektura orientovaná na služby} (SOA) je
důsledkem této evoluce. Podle známého pravidla \uv{rozděl a panuj}
dělí systém na samostatné celky, zvané \textit{služby}, které jsou zodpovědné
za dílčí část funkcionality.

\goal{Výhody SOA}
To přináší výhody v podobě ... % TODO: dopišmě

\goal{Ambiguita termínu SOA v historii}
Historicky byl termín SOA vykládán různými způsoby a vývojáři si
pod nám představovali několik rozdílných, nekompatibilních
konceptů~\cite{fowler2005serviceorientedambiguity}.
To vedlo k vzájemnému nedorozumění a ... této architektury %TODO: dopišmě

\goal{Web services}
SOAP~\cite{curbera2002unraveling}, REST~\cite{fielding2000rest}

~\cite{perrey2003service}
~\cite{cerny2017disambiguation}

\goal{Definice služby}
\paragraph{Služba} je ucelenou systémovou komponentou,
kterou lze nasadit a spustit jako samostatný proces, a
komunikuje s ostatními službami pomocí zpráv.

\section{Problémy}

\section{Identifikace požadavků na implementaci frameworku}

\begin{itemize}
    \item{Definice byznys kontextů pomocí doménově specifického jazyka srozumitelného pro doménové experty}
    \item{Zápis preconditions a post-conditions pravidla jednotlivých byznys kontextů}
    \item{Automatická distribuce kontextů, vyhodnocování jejich preconditions a aplikace post-conditions}
    \item{Možnost jednoho kontextu rozšiřovat jiné kontexty}
    \item{Možnost centrálně spravovat byznysové kontexty, včetně úpravy stávajících a vytváření nových kontextů}
\end{itemize}

\section{Shrnutí}
