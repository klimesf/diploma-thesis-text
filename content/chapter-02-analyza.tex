%!TEX ROOT=../diploma-thesis.tex

\chapter{Anal\'yza problematiky}\label{ch:analyza}

Tato kapitola analyzuje problematiku byznysov\'ych pravidel v informačn\'{\i}ch systémech
a detailně popisuje architekturu orientovanou na služby, včetně jej\'{\i}ho historického
v\'yvoje a modern\'{\i}ch trendů. Na základě toho kapitola identifikuje nedostatky
současn\'ych př\'{\i}stupů při sdílení byznysových pravidel. V závěru kapitoly jsou
specifikovány požadavky, které by měl splňovat framework, jež bude v\'ystupem této diplomové práce.

\section{Byznysová pravidla}\label{sec:business-rules}

Podnikové informačn\'{\i} systémy (\gls{EIS} z anglického \textit{Enterprise Information System})
maj\'{\i} za úkol ulehčit, automatizovat či poskytovat podporu pro byznysové procesy organizace,
která je využívá~\cite{dumas2005process}. Byznysové procesy jsou souborem dílčích úkolů
a aktivit, které naplňují cíle přinášející organizaci byznysovou hodnotu~\cite{weske2012business}.
Jednotlivé kroky byznysového procesu jsou nazývány byznysové operace. Byznysové procesy a operace dohromady
tvoří tzv. byznysovou logiku, která podléhá byznysov\'ym pravidlům. Ta zajišťuj\'{\i} konzistenci dat
systému a zabraňuj\'{\i} nepovolen\'ym operac\'{\i}m~\cite{cemus2015automated}.

\begin{definition}
    Byznysové pravidlo je logická rozhodovací podmínka, která definuje či omezuje byznysové chování informačního systému,
    je specifické pro konkrétní byznysovou doménu, a je atomické, tj. nelze dále rozdělit na dílčí pravidla~\cite{cemus2015automated, morgan2002business}.
\end{definition}

\gls{EIS} obsahují mnoho byznysových pravidel, typicky stovky či tisíce~\cite{morgan2002business}.
Samotné pravidlo však bývá relativně krátké a lze shrnout do jedné věty. Díky tomu je pochopitelné
pro všechny zainteresované strany, které se podílejí na návrhu a vývoji systému.
Byznysová pravidla jsou podle jedné z klasifikací dělena do tř\'{\i} skupin~\cite{cemus2014aspect}:

\begin{description}
    \item [Bezkontextová pravidla] jsou validačn\'{\i} pravidla, která musej\'{\i} b\'yt obecně platná
    v každé byznysové operaci, jinak by mohlo doj\'{\i}t k porušen\'{\i} integrity dat systému.
    Př\'{\i}kladem může b\'yt pravidlo \uv{\textit{Adresa uživatele je platnou e-mailovou adresou}}.
    \item [Kontextová pravidla] jsou pravidla, která musej\'{\i} b\'yt zohledněna v daném kontextu
    byznysové operace, např\'{\i}klad \uv{\textit{Při přidán\'{\i} produktu do koš\'{\i}ku nesm\'{\i} součet položek
    v koš\'{\i}ku přesahovat částku milion korun}}
    \item [Průřezová pravidla] jsou parametrizována stavem systému nebo uživatelského účtu a maj\'{\i}
    dopad na velkou část byznysov\'ych operac\'{\i}, například pravidlo \uv{\textit{V systému nesm\'{\i} prob\'{\i}hat
    žádné změny po dobu účetn\'{\i} uzávěrky}}.
\end{description}

Pravidla lze také rozdělit do dvou skupin podle jejich vztahu k byznysové operaci, a těmi jsou
\textit{preconditions} a \textit{post-conditions}~\cite{cemus2015automated, morgan2002business}.

\subsection{Precondition}

Aby mohla b\'yt byznysová operace vykonána, musej\'{\i}
b\'yt splněny předem definované podm\'{\i}nky, neboli předpoklady,
které naz\'yváme \textit{preconditions}. Pokud alespoň jedna z podm\'{\i}nek
nen\'{\i} splněna, byznysová operace nemůže proběhnout~\cite{larman2001patterns}.
Například při registraci uživatele mus\'{\i} b\'yt splněna podm\'{\i}nka,
že uživatel vyplnil svoj\'{\i} emailovou adresu, a zároveň
dosud v systému neexistuje žádn\'y uživatel se stejnou emailovou adresou.

\subsection{Post-condition}

Na byznysovou operaci mohou b\'yt kladeny požadavky, které
musej\'{\i} b\'yt splněny po jej\'{\i}m úspěšném vykonán\'{\i}~\cite{cemus2015automated}.
Př\'{\i}kladem může b\'yt anonymizace uživatelů při vytvářen\'{\i} statistického
reportu e-commerce společnosti, který nesmí obsahovat citlivé údaje
zákazníků. Dalš\'{\i}m př\'{\i}padem může b\'yt filtrován\'{\i}
v\'ystupu byznysové operace, např\'{\i}klad při v\'ypisu objednávek pro zákazn\'{\i}ka
musí všechny vypsané objednávky patřit danému zákazn\'{\i}kovi.

\subsection{Byznysov\'y kontext}

Jak již bylo uvedeno, vykonávání byznysové operace je podmíněno byznysovými pravidly, která se k ní vztahují. Před spuštěním operace musejí
být splněny všechny preconditions, jinak není možné operaci vykonat. Po jejím dokončení musejí být splněny všechny post-conditions.
Aby \gls{EIS} mohl tyto podmínky ověřit dynamicky za běhu systému, využívá tzv. \textit{exekuční kontext} (z anglického
\textit{execution context}), který se skládá z několika dílčích kontextů~\cite{cemus2017separation}:

\begin{description}
    \item[Aplikační kontext] drží stav globáln\'{\i}ch proměnn\'ych systému, jako například stav databáze,
    nastaven\'{\i} produkčn\'{\i}ho režimu, nebo př\'{\i}znak o tom, zda právě prob\'{\i}há obchodn\'{\i} uzávěrka.
    \item[Uživatelský kontext] obsahuje informace o aktuálně přihlášeném uživateli.
    \item[Kontext požadavku] (z anglického \textit{Request context}) se váže zejména na webové služby a obsahuje
    informace o aktuáln\'{\i}m požadavku, jako je IP adresa či geolokace uživatele.
    \item[Byznysov\'y kontext] obsahuje informace o probíhající byznysové operaci včetně byznysových pravidel.
\end{description}

Byznysový kontext je tedy důležitým prvkem při výkonu byznysové operace, který umožňuje vyhodnocování byznysových pravidel.
Všechny proměnné, které jsou dostupné v exekučním kontextu, jsou dostupné při vyhodnocování pravidel. Díky tomu je možné
definovat široké spektrum logických podmínek, které se mohou přizpůsobit aktuálnímu stavu systému.

\begin{definition}
    Byznysový kontext je množina preconditions a post-conditions, která se váže na konkrétn\'{\i} byznysovou
    operaci~\cite{cemus2015automated}.
\end{definition}

\subsection{Reprezentace byznysového pravidla}\label{sec:business-rule-representation}

Existuje několik možnost\'{\i}, jak v rámci \gls{EIS} zachytit a reprezentovat byznysová
pravidla~\cite{cemus2015automated}. Nejběžnější metodou zachycení byznysových pravidel je zápis
v obecném programovacím jazyce. Ačkoliv většina současných obecných programovacích
jazyků poskytuje reflexi, nepřináší dostatečné možnosti inspekce a extrakce pravidel pro jejich další
využití. Tento způsob však umožňuje snadno definovat i velmi komplikovaná pravidla.

Pokročilejš\'{\i} metodou je pro zachycení pravidel pomocí meta-instrukcí.
Těmi mohou být např\'{\i}klad anotace, případně speciální jazyk, tzv. \textit{Expression
Language}~\cite{nemuraite2008representation}. Tento způsob poskytuje dostatečnou možnost inspekce, ale nen\'{\i}
typově bezpečný a kvůli tomu je náchylnější na chybu. Anotace je navíc potřeba vázat na existující komponenty
systému, což může být pro některá byznysová pravidla nevhodné~\cite{cemus2015automated}.
Příkladem je průřezové pravidlo, které zamyká systém pro úpravy po dobu účetní uzávěrky. Takové pravidlo se
nevztahuje k žádné konkrétní komponentě systému a musí tak být zachyceno v každé komponentě zvlášť.

Nejpokročilejš\'{\i} metodou zachycení pravidla je využití \gls{DSL}, které jsou snadno srozumitelné a
poskytují snadnou inspekci a typovou bezpečnost. Mezi jejich nev\'yhody patř\'{\i} vysoká počátečn\'{\i} investice
v podobě návrhu jazyka, nutnost jeho kompilace nebo interpretace a také proškolení personálu, který s ním bude pracovat.
Kvůli tomu může být vhodné využít existující řešení~\cite{cemus2015automated}, jako například formální jazyk
\textit{Object Constraint Language}~\cite{warmer1998object}. Ten umožňuje snadné automatické zpracování a transformaci,
ale i formální matematické operace nad jeho výrazy. Jeho nevýhodou je pracnost zápisu. Využít lze i některé z průmyslových
řešení, jako je framework Drools\footnote{\url{https://www.drools.org/}}, který je popsaný v sekci~\ref{sec:drools}.

\section{Architektura orientovaná na služby}\label{sec:soa}

\textit{Architektura orientovaná na služby} (\gls{SOA}) je odpovědí na stále se zvyšující
nároky na informační systémy a jejich rostoucí velikost a komplexitu. Na rozd\'{\i}l od \textit{monolitické architektury},
děl\'{\i} \gls{SOA} systém na samostatné nezávislé celky, zvané \textit{služby}, které
poskytují d\'{\i}lč\'{\i} části požadované funkcionality.
Historicky byl term\'{\i}n \gls{SOA} vykládán několika způsoby a představoval
několik rozd\'{\i}ln\'ych, nekompatibiln\'{\i}ch konceptů~\cite{fowler2005serviceorientedambiguity}.
Absence kvalitn\'{\i}ch definic služby a obecně \gls{SOA} vedla v posledn\'{\i} době i ke snahám
o opuštěn\'{\i} tohoto konceptu~\cite{cerny2017disambiguation}.
Pro lepší porozumění se tato kapitola věnuje stručnému historickému přehledu \gls{SOA}
a shrnuje vlastnosti a z nich vyplývající výhody a nevýhody jednotlivých přístupů.

\begin{definition}
    Služba je znovupoužitelný, soudržný, spravovatelný, nasaditelný a nezávislý proces komunikující
    s ostatními službami pomocí zpráv~\cite{dragoni2017microservices, papazoglou2003service}.
\end{definition}

\subsection{Common Object Request Broker Architecture}\label{sec:corba}

Prvn\'{\i}m historick\'ym předchůdcem architektury orientované na služby
byla tzv. \textit{Common Object Request Broker Architecture}
(\gls{CORBA})~\cite{siegel2000corba}. Ta umožňuje vzájemnou komunikaci aplikací
implementovan\'ych v různ\'ych technologi\'{\i}ch. Její základní komponentou
je \textit{Object Request Broker} (\gls{ORB}), kter\'y simuluje vzdálené objekty,
na kter\'ych může klient volat jejich metody. Při zavolán\'{\i} metody
na objektu, kter\'y se fyzicky nacház\'{\i} na vzdáleném stroji,
zprostředkovává \gls{ORB} veškerou komunikaci a poskytuje kompletn\'{\i} rozhran\'{\i}
volaného objektu. Stejný způsob komunikace s lokálním jako se vzdáleným objektem je však
problematický kvůli nižší robusnosti a vyšší latenci síťového přenosu.

\subsection{Web Services}

Nedostatky architektury \gls{CORBA} vedly k vývoji jednodušš\'{\i}ho
a kvalitnějšího formátu pro popis komunikace služeb. Volán\'{\i} metod na vzdálen\'ych objektech
bylo nahrazeno explicitn\'{\i}m pos\'{\i}lán\'{\i}m zpráv mezi službami pomocí protokolu \gls{HTTP}.
Pro popis schématu zpráv vznikl formát \textit{Simple Object Access
Protocol}~\cite{box2000simple}, kter\'y v kombinaci s
\textit{Web Service Description Language}~\cite{christensen2001web}
umožňuje kompletn\'{\i} definici rozhran\'{\i} pro komunikaci mezi službami.

\subsection{Message Queue}

Dalš\'{\i}m z konceptů, kter\'y v rámci \gls{SOA} vznikl, je tzv. \textit{Message Queue} (\gls{MQ}).
Jeho základn\'{\i} myšlenkou je asynchronn\'{\i} komunikace služeb pomoc\'{\i} zpráv nezávisl\'ych
na platformě. Komunikaci zprostředkovává fronta, která přij\'{\i}má a rozes\'{\i}lá
zprávy mezi službami. To přináš\'{\i} vyšš\'{\i} škálovatelnost a menš\'{\i} provázanost
mezi službami. Všechny služby ale mus\'{\i} použ\'{\i}vat jednotn\'y formát zpráv.

\subsection{Enterprise Service Bus}

Ačkoliv zm\'{\i}něné modely usnadňuj\'{\i} komunikaci služeb a zvyšuj\'{\i} jejich
spolehlivost, integrace služeb může b\'yt obt\'{\i}žná, pokud služby použ\'{\i}vaj\'{\i}
navzájem různé komunikačn\'{\i} protokoly a formáty. Tento problém řeší \textit{Enterprise Service
Bus} (\gls{ESB})~\cite{chappell2004enterprise}, znázorněn\'y na obrázku~\ref{fig:enterprise-service-bus},
kter\'y má za úkol propojit heterogenn\'{\i} služby využívající různé technologie a sestavit mezi nimi
komunikačn\'{\i} kanály. T\'{\i}m na sebe \gls{ESB} přeb\'{\i}rá zodpovědnost za překlad jednotliv\'ych zpráv
a centralizuje veškerou komunikaci v systému. Kvůli tomu musí poskytovat routování síťových zpráv a překlad mezi
jednotlivými formáty. \gls{ESB} navíc umožňuje zabezpečení komunikace a může sloužit i pro orchestraci služeb,
jak je popsáno v sekci~\ref{sec:orchestration-vs-choreography}.

\begin{figure}
    \centering
    \includegraphics[keepaspectratio=true, width=0.65\linewidth]{figures/enterprise-service-bus.pdf}
    \caption{Komunikace služeb pomocí Enterprise Service Bus}
    \label{fig:enterprise-service-bus}
\end{figure}

\subsection{Microservices}\label{sec:microservices}

%\goal{Microservices a budoucnost SOA}
Microservices je moderní architekturou, která podobně jako \gls{SOA} přináší řešení
problémů pramenících z vysoké komplexity současných \gls{EIS}. Tato architektura se dá
chápat jako podmnožina \gls{SOA}~\cite{cerny2017disambiguation, richards2015microservices},
ačkoliv existují i názory, že jde o odlišné architektury. V poslední době postupně nahrazuje standardní
\gls{SOA}~\cite{lewis2014microservices, xiao2016reflections} a vzhledem k její vzrůstající
adopci je nutno ji v rámci této práce zohlednit.

Její základn\'{\i} myšlenkou je v\'yvoj informačn\'{\i}ho systému jako množiny mal\'ych oddělen\'ych služeb,
které jsou spouštěny v samostatn\'ych procesech a komunikuj\'{\i} spolu pomoc\'{\i} jednoduch\'ych
protokolů nezávislých na platformě~\cite{lewis2014microservices}. Microservices preferuje decentralizaci a samostatnost služeb
a zaměřuje se na jejich organizaci kolem byznysov\'ych schopnost\'{\i} systému, nam\'{\i}sto horizontáln\'{\i}ho
dělen\'{\i} systému podle jeho vrstev. Hlavní výhodou tohoto přístupu je flexibilita nasazení a škálování, která je vhodná
pro stále populárnější nasazení v cloudu~\cite{cerny2018contextual, kratzke2017understanding, xiao2016reflections}.

\subsection{Orchestrace a choreografie služeb}\label{sec:orchestration-vs-choreography}

Základní podmínkou pro funkci systému využívajícího \gls{SOA} je komunikace a spolupráce jednotlivých služeb.
K tomu slouží principy \textit{orchestrace služeb} a \textit{choreografie služeb}. Porovnán\'{\i} obou př\'{\i}stupů
je graficky znázorněno na obrázku~\ref{fig:choreography-orchestration}.

\paragraph{Orchestrace}
Orchestrace služeb má za úkol zajistit, že komunikace mezi službami
proběhne úspěšně a ve správném časovém sledu~\cite{orchestration},
za použití centráln\'{\i} komponenty -- tzv. \textit{dirigenta}.
Typicky je dirigent implementován jako součást \gls{ESB}.
Orchestrace implikuje centrální řízení a koordinaci služeb při
vykonávání byznysových procesů.

\paragraph{Choreografie}
Opakem orchestrace je tzv. \textit{choreografie služeb} a znamená
vykonáván\'{\i} byznysov\'ych operac\'{\i} autonomně a asynchronně, bez centráln\'{\i}
autority. Choreografie popisuje zejména pravidla a podobu komunikace jednotlivých služeb.
K tomu zpravidla využívá \gls{MQ} a zasílání zpráv o událostech v systému. Každá služba
při vykonávání byznysových operací plní svoji roli bez centrálního řízení~\cite{cerny2017disambiguation}.
Tento př\'{\i}stup je využíván zejména v rámci Microservices~\cite{dragoni2017microservices},
protože umožňuje nižší provázán\'{\i} služeb a rovnoměrnější rozložen\'{\i} zodpovědnost\'{\i} v systémů.

\begin{figure}
    \centering
    \includegraphics[keepaspectratio=true, width=0.8\linewidth]{figures/choreography-orchestration.pdf}
    \caption{Porovnán\'{\i} orchestrace a choreografie služeb}
    \label{fig:choreography-orchestration}
\end{figure}

\subsection{Shrnutí}

Z předchozího textu vyplývá, že přístupy k realizaci \gls{SOA} vychází ze společné
myšlenky členění systémů do dílčích izolovaných služeb poskytujících byznysovou funkcionalitu.
Přístupy se liší zejména v řešení komunikace služeb a v centralizaci jejich správy.
Historické přístupy využívají komplexní komunikační technologie a umožňují centrální
správu a orchestraci systému, zatímco moderní přístup Microservices se zaměřuje zejména
na nízké provázání služeb a využití jednoduchých komunikačních kanálů.
To přináší výzvu při sdílení byznysových pravidel, která je rozebrána v následující sekci.

\begin{definition}
    \gls{SOA} je soubor návrhových principů, který organizuje komponenty software
    kolem byznysové funkcionality a spojuje je pomocí rozhraní a komunikačních protokolů.
    Každá komponenta je soběstačná a izolovaná, okolnímu světu poskytuje pouze
    své komunikační rozhraní~\cite{lewis2014microservices, papazoglou2003service}.
\end{definition}

\section{Nedostatky současného př\'{\i}stupu}\label{sec:shortcomings}

Některá složitější byznysová funkcionalita vyžaduje spolupráci více služeb
najednou~\cite{papazoglou2003service}. K tomuto účelu slouží tzv. kompozitní
služby, které využívají funkcionalitu ostatních služeb. Při kompozici služeb je nutné
zohlednit jejich byznysová pravidla, aby bylo zabráněno nekonzistentním stavům v systému
a zbytečným spouštěním byznysových operací, jejichž preconditions nejsou
splněny~\cite{cerny2017disambiguation}. To je však v přímém rozporu s požadavkem na nízkou
provázanost služeb, které by neměly vzájemně znát svoji interní strukturu. Tato skutečnost
vede k nutnosti duplikace byznysových pravidel v kompozitních službách~\cite{cerny2016survey}.

\begin{definition}
    Kompozitní služba získává a kombinuje informace a funkce z ostatních služeb~\cite{papazoglou2003service}.
\end{definition}

%\goal{Nast\'{\i}něn\'{\i} konkrétn\'{\i}ho př\'{\i}kladu}
Pro lepší představu je problém znázorněn na obrázku~\ref{fig:service-cutting}. Uvažme e-commerce
systém skládaj\'{\i}c\'{\i} se z několika služeb naprogramovan\'ych v různ\'ych technologi\'{\i}ch, a
procesy vytváření faktury a vytváření objednávky. Při dokončení objednávky je potřeba vytvořit fakturu,
objednávková služba k tomu využívá fakturační službu. Podmínkou pro vytvořen\'{\i} faktury je platná fakturačn\'{\i} adresa,
která je zadána již při vytváření objednávky. Nevalidn\'{\i} adresa by však při vytváření faktury
mohla způsobit řadu problémů vyžadujících manuální řešení. Proto musí být tato adresa validována
již při vytvářen\'{\i} objednávky. V ideáln\'{\i}m př\'{\i}padě by navíc měl zákazn\'{\i}k být upozorněn
na nevalidn\'{\i} fakturačn\'{\i} adresu co nejdříve, ještě před odeslán\'{\i}m objednávkového formuláře,
př\'{\i}mo v uživatelském rozhran\'{\i}~\cite{cemus2017separation}. To bývá často implementováno samostatnou
kompozitní službou, která agreguje funkcionalitu celého systému.

%\goal{Náročná údržba a reakce na změnu požadavku}
Na příkladu lze pozorovat, že jedno byznysové pravidlo se prom\'{\i}tá
do tř\'{\i} služeb, z nichž každá má zodpovědnost za jiné byznysové operace.
Stejná logika, která realizuje validaci fakturačn\'{\i} adresy,
mus\'{\i} b\'yt implementována v každé ze zmiňovaných služeb, navíc v různých technologiích.
Pokud by vzešel změnový požadavek na validaci fakturačn\'{\i} adresy, změnu by bylo nutno
provést konzistentně na třech různ\'ych m\'{\i}stech, všechny tři služby znovu
sestavit a nasadit ve správném pořad\'{\i} tak, aby nedošlo k nekonzistentní validaci adresy
při provádění jednotlivých byznysových operací. Změny byznysových pravidel se dějí častěji,
než změny kódu a struktury samotných služeb v \gls{SOA}~\cite{rosenberg2005business}.
Pokud je potřeba s každou změnou byznysového pravidla sestavit a nasadit jednu či více služeb,
dramaticky se zvyšuje náročnost na údržbu takového systému.

\begin{figure}
    \centering
    \includegraphics[keepaspectratio=true, width=0.7\linewidth]{figures/service-cutting.pdf}
    \caption{Př\'{\i}klad funkcionality zasahující do v\'{\i}ce služeb}
    \label{fig:service-cutting}
\end{figure}

\section{Identifikace požadavků na implementaci frameworku}\label{sec:implementation-requirements}

Pro usnadnění vývoje a údržby systému stavějícího na \gls{SOA}, který
obsahuje kompozitní služby, je nutné umožnit sdílení byznysových pravidel.
Ta by měla být zachycena mimo samotnou implementaci služby, ideálně ve formátu,
který bude nezávislý na konkrétní platformě, bude poskytovat možnost automatické
inspekce a bude srozumitelný doménovým expertům. Framework nesmí vynucovat
manuální duplikaci byznysových pravidel. Díky tomu framework sníží náklady
na vývoj a údržbu systému a riziko lidské chyby. Úprava pravidel navíc nesmí
vyžadovat změnu kódu služby a její opětovné nasazování.
Administrátoři systému by měli mít možnost byznysová pravidla spravovat centrálně a bez
přerušení provozu systému, aby mohli co nejrychleji a flexibilně reagovat na změnové požadavky.
Vzhledem k různému chápání \gls{SOA} a postupné adopci Microservices framework nesmí klást
nároky na způsob organizace služeb, tj. měl by umožňovat orchestraci i choreografii.

Framework, který bude výstupem této práce, musí splňovat následující vlastnosti:

\begin{itemize}
    \item{Možnost definovat byznysová pravidla pomoc\'{\i} platformově nezávislého \gls{DSL} srozumitelného pro doménové experty}
    \item{Možnost centrálně spravovat byznysová pravidla, včetně úpravy stávaj\'{\i}c\'{\i}ch a vytvářen\'{\i} nov\'ych dynamicky za běhu systému}
    \item{Automatická distribuce a integrace byznysových pravidel včetně vyhodnocován\'{\i} preconditions a aplikace post-conditions}
    \item{Možnost využ\'{\i}vat framework na v\'{\i}ce plaformách}
    \item{Nezávislost na organizaci služeb}
\end{itemize}

\section{Shrnut\'{\i}}

V této kapitole byla provedena analýza byznysov\'ych pravidel a architektury orientované na služby.
Na základě toho byly popsány nedostatky \gls{SOA} při kompozici služeb a sdílení byznysových pravidel.
Nakonec byly identifikovány požadavky, které by měl splňovat framework, který bude v\'ystupem této práce.
