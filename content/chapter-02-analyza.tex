%!TEX ROOT=../diploma-thesis.tex

\chapter{Anal\'yza}\label{ch:analyza}

Tato kapitola analyzuje problematiku byznysov\'ych pravidel v informačn\'{\i}ch systémech
a detailně popisuje architekturu orientovanou na služby, včetně jej\'{\i}ho historického
v\'yvoje a modern\'{\i}ho trendu v podobě microservices. Na základě toho kapitola popisuje nedostatky
současn\'ych př\'{\i}stupů při řešen\'{\i} průřezov\'ych problémů v těchto architekturách, s důrazem na byznysová pravidla.
V závěru kapitoly jsou identifikovány požadavky, které by měl splňovat framework,
jež bude v\'ystupem této diplomové práce.

\section{Byznysová pravidla}\label{sec:business-rules}

Informačn\'{\i} systémy (\gls{IS}) maj\'{\i} za úkol ulehčit, automatizovat či poskytovat podporu pro
byznysové procesy společnost\'{\i}, které je využ\'{\i}vaj\'{\i}. Tyto procesy jsou tedy stěžejn\'{\i}m
prvkem \gls{IS}. Systém má také za úkol uchovávat a spravovat data společnosti
a měl by zaručit, že nedojde k jejich poškozen\'{\i} či narušen\'{\i} jejich integrity.
Byznysové procesy, potažmo byznysové operace, proto musej\'{\i}
podléhat jasně definovan\'ym byznysov\'ym pravidlům, která zajišťuj\'{\i} konzistenci dat informačn\'{\i}ho
systému a také zabraňuj\'{\i} nepovolen\'ym operac\'{\i}m~\cite{cemus2015automated}.

Byznysová pravidla děl\'{\i}me do tř\'{\i} skupin~\cite{cemus2014aspect}:
\begin{description}
    \item [Bezkontextová pravidla] jsou validačn\'{\i} pravidla, která musej\'{\i} b\'yt obecně platná
    v každé operaci, jinak by mohlo doj\'{\i}t k porušen\'{\i} integrity dat systému. Př\'{\i}kladem může
    b\'yt pravidlo \uv{\textit{Adresa uživatele je platnou e-mailovou adresou}}.
    \item [Kontextová pravidla] jsou pravidla, která musej\'{\i} b\'yt zohledněna v daném kontextu
    byznysové operace, např\'{\i}klad \uv{\textit{Při přidán\'{\i} produktu do koš\'{\i}ku nesm\'{\i} součet položek
    v koš\'{\i}ku přesahovat částku milion korun}}
    \item [Průřezová pravidla] jsou parametrizována stavem systému nebo uživatelského účtu a maj\'{\i}
    dopad na velkou část byznysov\'ych operac\'{\i}. Uvažme pravidlo \uv{\textit{V systému nesm\'{\i} prob\'{\i}hat
    žádné změny po dobu účetn\'{\i} uzávěrky}}.
\end{description}

Dále také rozlišujeme dva typy byznysov\'ych pravidel, a těmi jsou \textit{preconditions}
a \textit{post-conditions}~\cite{cemus2015automated}.

\subsection{Precondition}

Aby mohla b\'yt byznysová operace vykonána, musej\'{\i}
b\'yt splněny předem definované podm\'{\i}nky, neboli předpoklady,
které naz\'yváme \textit{preconditions}. Pokud alespoň jedna z podm\'{\i}nek
nen\'{\i} splněna, byznysová operace nemůže proběhnout.

Pro lepš\'{\i} ilustraci uveďme př\'{\i}klad: aby mohla b\'yt provedena
registrace uživatele s danou emailovou adresu, mus\'{\i} b\'yt splněna
podm\'{\i}nka, že uživatel vyplnil svoj\'{\i} emailovou adresu, a zároveň
dosud v systému neexistuje žádn\'y uživatel se stejnou emailovou adresou.

\subsection{Post-condition}

Na byznysovou operaci mohou b\'yt kladeny požadavky, které
musej\'{\i} b\'yt splněny po jej\'{\i}m úspěšném vykonán\'{\i}. Př\'{\i}kladem
může b\'yt anonymizace uživatelů při vytvářen\'{\i} statistického
reportu e-commerce společnosti – po vygenerován\'{\i} reportu
post-condition zajist\'{\i}, že z něj budou smazány veškeré citlivé údaje.
Dalš\'{\i}m př\'{\i}padem může b\'yt filtrován\'{\i} v\'ystupu byznysové operace.
Např\'{\i}klad při v\'ypisu objednávek pro zákazn\'{\i}ka se chceme ujistit, že
všechny vypsané objednávky patř\'{\i} danému zákazn\'{\i}kovi.

\subsection{Reprezentace byznysového pravidla}

Existuje několik možnost\'{\i}, jak zachytit a reprezentovat byznysová pravidla~\cite{cemus2015automated}.
Nejběžnějš\'{\i} a nejpouž\'{\i}vanějš\'{\i} metodou je jejich zachycen\'{\i} v programovac\'{\i}m
jazyce. Tato metoda je snadná, protože programátor může použ\'{\i}t stejn\'y jazyk
pro popis pravidel stejně jako pro popis celého systému. Bohužel, tato metoda
nám nedává př\'{\i}liš možnost\'{\i} jak provést inspekci a extrakci pravidel.
Dalš\'{\i}, pokročilejš\'{\i} metodou, je zápis pravidel pomoc\'{\i} meta-instrukc\'{\i}, např\'{\i}klad anotac\'{\i},
nebo tzv. \textit{Expression Language} (\gls{EL}). Tato metoda poskytuje dobrou možnost inspekce,
ale zpravidla nen\'{\i} typově bezpečná a může snáze způsobovat chyby v programu.
Posledn\'{\i}, nejpokročilejš\'{\i} metodou, je zápis pomoc\'{\i} doménově specifick\'ych jazyků.
Ty jsou snadno srozumitelné nejen pro programátory, ale i pro doménové experty.
Nevyžaduj\'{\i} inspekci a mohou b\'yt typově bezpečné. Mezi jejich nev\'yhody ale patř\'{\i} vysoká
počátečn\'{\i} investice v podobě návrhu takového jazyka a nutnost jeho kompilace nebo
interpretace.

\subsection{Byznysov\'y kontext}

Informačn\'{\i} systém zpravidla implementuje v\'{\i}ce byznysov\'ych procesů, které se vážou
na jeden či v\'{\i}ce uživatelsk\'ych scénářů. Uživatelsk\'y scénář se pak děl\'{\i} na jednotlivé
kroky, např\'{\i}klad zaslán\'{\i} potvrzovac\'{\i}ho e-mailu k objednávce, či uložen\'{\i} objednávky
do databáze. Tyto kroky naz\'yváme \textit{byznysové operace} – tedy operace, které maj\'{\i}
byznysovou hodnotu. Ke každé byznysové operaci př\'{\i}sluš\'{\i} množina byznysov\'ych pravidel,
konkrétně preconditions a post-conditions.

Při běhu informačn\'{\i}ho systému je v paměti držen tzv. \textit{exekučn\'{\i} kontext} (z anglického \textit{execution context}),
kter\'y se skládá z několika d\'{\i}lč\'{\i}ch kontextů~\cite{cemus2017separation}. Prvn\'{\i}m
je \textit{aplikačn\'{\i} kontext} (z anglického \textit{application context}), ve kterém je uložen stav globáln\'{\i}ch proměnn\'ych systému,
jako například nastaven\'{\i} produkčn\'{\i}ho režimu, nebo př\'{\i}znak o tom, zda právě prob\'{\i}há obchodn\'{\i}
uzávěrka. Dalš\'{\i}m je \textit{uživatelsk\'y kontext}, kter\'y obsahuje informace o aktuálně
přihlášeném uživateli. \textit{Kontext požadavku} (z anglického \textit{Request context}) obsahuje
informace o aktuáln\'{\i}m požadavku, jako IP adresa uživatele či jeho geolokace,
a vztahuje se zejména k webov\'ym službám. Posledn\'{\i}m je \textit{byznysov\'y kontext}. Ten
chápeme jako množinu preconditions a post-conditions s byznysovou hodnotou, která se
váže na konkrétn\'{\i} byzynsovou operaci~\cite{cemus2015automated}.
Abychom mohli efektivně definovat co nejširš\'{\i} škálu byzynsov\'ych pravidel,
musej\'{\i} při jejich vyhodnocován\'{\i} b\'yt dostupné proměnné exekučn\'{\i}ho kontextu,

\section{Architektura orientovaná na služby}\label{sec:soa}

\textit{Architektura orientovaná na služby} (\gls{SOA}) je odpovědí na stále se zvyšující
nároky na informační systémy a jejich rostoucí velikost. Na rozd\'{\i}l od \textit{monolitické architektury},
děl\'{\i} \gls{SOA} systém na samostatné nezávislé celky, zvané \textit{služby}, které jsou
poskytují d\'{\i}lč\'{\i} části požadované funkcionality systému.
Historicky byl term\'{\i}n \gls{SOA} vykládán několika způsoby a představoval
několik rozd\'{\i}ln\'ych, nekompatibiln\'{\i}ch konceptů~\cite{fowler2005serviceorientedambiguity}.
Absence kvalitn\'{\i}ch definic služby a obecně \gls{SOA} vedla k v posledn\'{\i} době i ke snahám
o opuštěn\'{\i} tohoto konceptu~\cite{cerny2017disambiguation}.
Pro lepší porozumění se tato kapitola věnuje stručnému historickému přehledu \gls{SOA}
a shrnuje výhody a nevýhody jednotlivých přístupů.

\subsection{Common Object Request Broker Architecture}\label{sec:corba}

Prvn\'{\i}m historick\'ym předchůdcem architektury orientované na služby
byla tzv. \textit{Common Object Request Broker Architecture}
(\gls{CORBA})~\cite{siegel2000corba}. Ta umožňuje vzájemnou komunikaci aplikací
implementovan\'ych v různ\'ych technologi\'{\i}ch. Její základní komponentou je
je \textit{Object Request Broker} (\gls{ORB}), kter\'y emuluje objekty,
na kter\'ych může klient volat jejich metody. Při zavolán\'{\i} metody
na objektu, kter\'y se fyzicky nacház\'{\i} na vzdáleném stroji,
zprostředkovává \gls{ORB} veškerou komunikaci a poskytuje kompletn\'{\i} rozhran\'{\i}
volaného objektu. Komunikace se vzdálen\'ym objektem s sebou však nese celou řadu problémů,
např\'{\i}klad vyšš\'{\i} latenci při komunikaci nebo v\'yjimečné stavy, které je potřeba
ošetřit, či obížnou optimalizaci kódu využívající \gls{ORB}.

\subsection{Web Services}

Nedostatky architektury \gls{CORBA} vedly k vývoji jednodušš\'{\i}ho
a kvalitnějšího formátu pro popis komunikace služeb. Volán\'{\i} metod na vzdálen\'ych objektech
bylo nahrazeno explicitn\'{\i}m pos\'{\i}lán\'{\i}m zpráv mezi službami pomocí protokolu \gls{HTTP}.
Pro popis schématu zpráv vznikl formát \textit{Simple Object Access
Protocol} (\gls{SOAP})~\cite{box2000simple}, kter\'y v kombinaci s
\textit{Web Service Description Language} (\gls{WSDL})~\cite{christensen2001web}
umožňuje kompletn\'{\i} definici rozhran\'{\i} pro komunikaci mezi službami.

\subsection{Message Queue}

\begin{figure}
    \centering
    \includegraphics[keepaspectratio=true, width=0.5\linewidth]{figures/message-queue.pdf}
    \caption{Komunikace služeb pomoc\'{\i} Message Queue}
    \label{fig:message-queue}
\end{figure}

Dalš\'{\i}m z konceptů, kter\'y v rámci \gls{SOA} vznikl, je tzv. \textit{Message Queue} (\gls{MQ}).
Základn\'{\i} myšlenkou \gls{MQ}, znázorněnou na obrázku~\ref{fig:message-queue},
je asynchronn\'{\i} komunikace služeb pomoc\'{\i} zpráv nezávisl\'ych
na platformě. Komunikaci zprostředkovává fronta, která přij\'{\i}má a rozes\'{\i}lá
zprávy mezi službami. To přináš\'{\i} vyšš\'{\i} škálovatelnost a menš\'{\i} provázanost
mezi službami. Všechny služby ale mus\'{\i} použ\'{\i}vat jednotn\'y formát zpráv.

\subsection{Enterprise Service Bus}

Ačkoliv zm\'{\i}něné modely usnadňuj\'{\i} komunikaci služeb a zvyšuj\'{\i} jejich
spolehlivost, integrace služeb může b\'yt obt\'{\i}žná, pokud služby použ\'{\i}vaj\'{\i}
navzájem různé komunikačn\'{\i} protokoly a formáty. Tento problém řeší \textit{Enterprise Service
Bus} (\gls{ESB})~\cite{chappell2004enterprise}, znázorněn\'y na obrázku~\ref{fig:enterprise-service-bus},
kter\'y má za úkol propojit heterogenn\'{\i} služby a sestavit mezi nimi komunikačn\'{\i} kanály.
T\'{\i}m na sebe \gls{ESB} přeb\'{\i}rá zodpovědnost za překlad jednotliv\'ych zpráv a centralizuje
veškerou komunikaci v systému.

\begin{figure}
    \centering
    \includegraphics[keepaspectratio=true, width=0.7\linewidth]{figures/enterprise-service-bus.pdf}
    \caption{Komunikace služeb skrz Enterprise Service Bus}
    \label{fig:enterprise-service-bus}
\end{figure}

\subsection{Microservices}\label{sec:microservices}

%\goal{Microservices a budoucnost SOA}
Microservices je moderní architekturou, která podobně jako \gls{SOA} přináší řešení
problémů pramenících z vysoké komplexity současných \gls{IS}.
Tato architektura se dá chápat jako podmnožina \gls{SOA}, ačkoliv existují i názory,
že jde o odlišné architektury~\cite{richards2015microservices}\cite{cerny2017disambiguation}.
Základn\'{\i} myšlenkou je v\'yvoj informačn\'{\i}ho systému jako množiny mal\'ych oddělen\'ych služeb,
které jsou spouštěny v samostatn\'ych procesech a komunikuj\'{\i} spolu pomoc\'{\i} jednoduch\'ych
protokolů nezávislých na platformě~\cite{lewis2014microservices}. Microservices preferuje decentralizaci a samostatnost služeb
a zaměřují se na jejich organizaci kolem byznysov\'ych schopnost\'{\i} systému, nam\'{\i}sto horizontáln\'{\i}ho
dělen\'{\i} systému podle jeho vrstev\footnote{Zde předpokládáme klasickou tř\'{\i}vrstvou architekturu~\cite{fowler2002patterns},
rozděluj\'{\i}c\'{\i} systém na \textit{datovou vrstvu}, \textit{aplikačn\'{\i} vrstvu}
a \textit{prezentačn\'{\i} vrstvu}. Tyto vrstvy maj\'{\i} oddělené zodpovědnosti a komunikuj\'{\i}
spolu pomoc\'{\i} jasně definovan\'ych společn\'ych rozhran\'{\i}.}.

\subsection{Orchestrace a choreografie služeb}

Základní podmínkou pro správnou funkci systému stavějícímu na \gls{SOA} je správná
komunikace a spolupráce jednotlivých služeb. K tomu slouží dva odlišné přístupy \textendash\xspace
\textit{orchestrace služeb} a \textit{choreografie služeb}.

\paragraph{Orchestrace služeb}
Orchestrace služeb má za úkol zajistit, že komunikace mezi službami
proběhne úspěšně a ve správném časovém sledu~\cite{orchestration},
za použití centráln\'{\i} komponenty \textendash\xspace tzv. \textit{dirigenta}.
Typicky je jako dirigent využ\'{\i}ván \gls{ESB}, kter\'y je pro tuto roli vhodn\'y,
protože má informace o lokaci jednotliv\'ych služeb a zprostředkovává mezi nimi
komunikačn\'{\i} kanály.

\paragraph{Choreografie služeb}
Př\'{\i}m\'ym opakem orchestrace je tzv. \textit{choreografie služeb} a znamená
vykonáván\'{\i} byznysov\'ych operac\'{\i} autonomně a asynchronně, bez centráln\'{\i}
autority. Tento př\'{\i}stup je preferován zejména v rámci microservices~\cite{dragoni2017microservices},
protože orchestrace vede k vyšš\'{\i}mu provázán\'{\i} služeb a nerovnoměrnému rozložen\'{\i}
zodpovědnost\'{\i} v systémů. Porovnán\'{\i} obou př\'{\i}stupů je graficky
znázorněno na obrázku~\ref{fig:choreography-orchestration}.

\begin{figure}
    \centering
    \includegraphics[keepaspectratio=true, width=0.8\linewidth]{figures/choreography-orchestration.pdf}
    \caption{Porovnán\'{\i} orchestrace a choreografie služeb~\cite{cerny2017disambiguation}}
    \label{fig:choreography-orchestration}
\end{figure}

\subsection{Shrnutí}

Z předchozího textu vyplývá, že přístupy k realizaci \gls{SOA} vychází ze společné
myšlenky členění systémů do dílčích izolovaných služeb poskytujících byznysovou funkcionalitu.
Přístupy se liší zejména v přístupu ke komunikaci a spolupráci služeb a k centralizaci jejich správy.
Moderní přístupy od centralizace upouštějí, což ale přináší

\section{Nedostatky současného př\'{\i}stupu}\label{sec:shortcomings}

Některá složitější byznysová funkcionalita vyžaduje kompozici více služeb
najednou~\cite{papazoglou2003service}. Kompozitní služby by měly zohlednit
byznysová pravidla služeb, které ke své funkci využívají, aby zabránily
nekonzistentním stavům v systému a zbytečným spouštěním byznysových operací,
jejichž preconditions nejsou splněny~\cite{cerny2017disambiguation}.
To je však s přímým rozporem s požadavkem \gls{SOA} na nízkou provázanost služeb.
Služby by neměly mít potřebu vzájemně znát svoji interní strukturu. Tato skutečnost
vede k nutnosti duplikace byznysových pravidel v kompozitních službách~\cite{cerny2016survey}.

%\goal{Nast\'{\i}něn\'{\i} konkrétn\'{\i}ho př\'{\i}kladu}
Pro lepší představu tohoto problému uvažme e-commerce systém
skládaj\'{\i}c\'{\i} se z několika služeb naprogramovan\'ych v různ\'ych technologi\'{\i}ch,
a procesy vytváření faktury a vytváření objednávky, každý z nich implementovaný jinou službou.
Systém navíc obsahuje službu poskytující webové uživatelské rozhraní.
Při vytvářen\'{\i} faktury za objednávku mus\'{\i} b\'yt nejprve zvalidována fakturačn\'{\i} adresa.
Protože by mohla nastat situace, kdy by v př\'{\i}padě nevalidn\'{\i} adresy museli zaměstnanci
společnosti kontaktovat zákazn\'{\i}ka \textendash\xspace pokud vůbec takovou možnost maj\'{\i}
\textendash\xspace musí být adresa validována již při vytvářen\'{\i} objednávky.
V ideáln\'{\i}m př\'{\i}padě by navíc měl zákazn\'{\i}k být upozorněn na nevalidn\'{\i} fakturačn\'{\i}
adresu co nejdříve, ještě před odeslán\'{\i}m objednávkového formuláře př\'{\i}mo v uživatelském
rozhran\'{\i}~\cite{cemus2017separation}. Pro ilustraci je problém znázorněn na obrázku~\ref{fig:service-cutting},

%\goal{Náročná údržba a reakce na změnu požadavku}
Na příkladu lze pozorovat, že stejná funkcionalita se prom\'{\i}tá
do tř\'{\i} služeb, z nichž každá má zodpovědnost za jiné byznysové operace.
Stejn\'y kód, kter\'y realizuje validaci fakturačn\'{\i} adresy,
mus\'{\i} b\'yt implementován v každé ze zmiňovaných služeb, navíc v různých technologiích.
Pokud by vzešel změnový požadavek na validaci fakturačn\'{\i} adresy, změnu by bylo nutno
provést konzistentně na třech různ\'ych m\'{\i}stech, všechny tři služby znovu
sestavit a nasadit ve správném pořad\'{\i} tak, aby nedošlo k nekonzistení validaci adresy
při provádění jednotlivých byznysových operací. Změny byznysových pravidel se dějí častěji,
než změny kódu a struktury samotných služeb v \gls{SOA}~\cite{rosenberg2005business}.
Pokud je potřeba s každou změnou byznysového pravidla sestavit a nasadit jednu či více služeb,
dramaticky se zvyšuje náročnost na údržbu takového systému.

\begin{figure}
    \centering
    \includegraphics[keepaspectratio=true, width=0.8\linewidth]{figures/service-cutting.pdf}
    \caption{Př\'{\i}klad zásahu jedné funkcionality do v\'{\i}ce služeb}
    \label{fig:service-cutting}
\end{figure}

\section{Identifikace požadavků na implementaci frameworku}\label{sec:implementation-requirements}

Pro usnadnění vývoje a údržby systému stavějícího na \gls{SOA}, který
obsahuje kompozitní služby, je nutné umožnit sdílení byznysových pravidel.
Ta by měla být zachycena mimo samotnou implementaci služby, ideálně ve formátu, který bude nezávislý na
konkrétní platformě, bude poskytovat možnost automatické inspekce a bude srozumitelný doménovým expertům.
Úprava pravidel by navíc neměla vyžadovat změnu kódu služby a její opětovné nasazování.
Administrátoři systému by měly mít možnost byznysová pravidla spravovat centrálně a bez
přerušení provozu systému, aby mohli co nejrychleji a flexibilně reagovat na změnové požadavky.

Framework, který bude výstupem této práce, by tedy měl splňovat následující vlastnosti:

\begin{itemize}
    \item{Možnost definovat byznysová pravidla pomoc\'{\i} platformově nezávislého \gls{DSL} srozumitelného pro doménové experty}
    \item{Možnost centrálně spravovat byznysová pravidla, včetně úpravy stávaj\'{\i}c\'{\i}ch a vytvářen\'{\i} nov\'ych dynamicky za běhu systému}
    \item{Automatická distribuce byzynsových pravidel včetně vyhodnocován\'{\i} preconditions a aplikace post-conditions}
    \item{Možnost využ\'{\i}vat framework na v\'{\i}ce plaformách}
\end{itemize}

\section{Shrnut\'{\i}}

V této kapitole byla provedena analýza byznysov\'ych pravidel a byznysov\'ych kontextů a architektury orientované na služby.
Dále byly popsány nedostatky \gls{SOA} při kompozici služeb a sdílení byznysových pravidel.
Nakonec byly identifikovány požadavky, které by měl splňovat framework, který bude v\'ystupem této práce.
