\usepackage[T1]{fontenc}
\usepackage[utf8]{inputenc}

%!TEX ROOT=../diploma-thesis.tex

\chapter{Rešerše}\label{ch:reserse}

\section{Architektura orientovaná na služby}

\section{Modelem řízená architektura}

\section{Aspektově orientované programování}

% Co je paradigma
Programování je komplexní disciplína s teoreticky
neomezeným počtem možností, jakým programátor může
řešit zadaný problém. Ačkoliv každá úloha má své specifické
požadavky, za relativně krátkou historii programování se
stihlo ustálit několik ideologií, tzv. programovacích
paradigmat, které programátorovi poskytují sadu abstrakcí
a základních principů~\cite{van2009programming}.
Díky znalosti paradigmatu může programátor nejen zlepšit
svou produktivitu, ale zároveň může snáze pochopit myšlenky
jiného programátora a tím zlepšit kvalitu týmové spolupráce.

% OOP a jeho popis
Jedním z nejpopulárnějších paradigmat používaných k
vývoji moderních enterprise systémů je nepochybně
objektově orientované programování (OOP). To vnímá daný problém
jako množinu objektu, které spolu intereagují. Program
člení na malé funkční celky odpovídající struktuře
reálného světa~\cite{rentsch1982object}. Je vhodné zmínit,
že objekty se rozumí jak konkrétní koncepty, například
auto nebo člověk, tak i abstraktní koncepty,
namátkou bankovní transakce nebo objednávka v obchodě.
Objekty se pak promítají do kódu programu i do
reprezentace struktur v paměti počítače.
Tento přístup je velmi snadný pro pochopení,
vede k lepšímu návrhu a organizaci programu a snižuje
tak náklady na jeho vývoj a údržbu.

% Nedostatky OOP
Ačkoliv je OOP velmi silným a všestraným nástrojem,
existují problémy, které nelze jeho pomocí efektivně řešit.
Jedním takovým problémem jsou obecné požadavky na systém,
které musejí být konzistentně dodržovány na více místech,
které spolu zdánlivě nesouvisí. Příkladem
může být logování systémových akcí, optimalizace správy paměti
nebo uniformní zpracování výjimek~\cite{kiczales1997aspect}.
Takové požadavky nazýváme \textit{cross-cutting concerns}.
V rámci OOP je programátor nucen v ojektech manuálně opakovat
kód, který zodpovídá za jejich realizaci. Duplikace kódu
vede k větší náchylnosti na lidskou chybu a k vyšším nárokům na vývoj
a údržbu daného softwarového systému~\cite{fowler1999refactoring}.

% AOP jako odpověď na nedostatky OOP
Aspektově orientované programování (AOP) přináší řešení na
výše zmiňované problémy. Extrahuje obecné požadavky,
tzv. \textit{aspekty} do jednoho místa a pomocí procesu zvaného
\textit{weaving} je poté automaticky distribuuje do systému.
Weaving může proběhnout staticky při kompilaci programu nebo dynamicky
při jeho běhu. V obou případech ale programátorovi ulehčuje práci,
protože k definici i změně aspektu dochází centrálně a tím je eliminována
potřeba manuální duplikace kódu. Je nutno poznamenat, že AOP není
paradigmatem poskytujícím kompletní framework pro návrh programu.
V ideálním případě je tedy k návrhu systému využita kombinace
AOP s jiným paradigmatem. Pro účely této práce se zaměříme na
kombinaci AOP a OOP.

% TODO: diagram cross cutting concerns
% TODO: rozáhnout do více odstavců
% TODO: diagram weaveru

\begin{description}
    \item [Aspekt]
    \item [Join-point]
    \item [Pointcut]
    \item [Advice]
    \item [Weaving]
\end{description}

% TODO: \section{BPEL}

\section{Aspect-driven Design Approach}

Aspect-driven Design Approach (ADDA)

% Vhodnost AOP pro náš úkol
Vzhledem k požadavkům na implementaci našeho frameworku stanoveným
v předchozí kapitole~\ref{ch:analyza} se AOP a na něm stavějící ADDA
jeví jako vhodný přístup, který nám pomůže dosáhnout cíle.

\section{Stávající řešení reprezentace byznys pravidel}

\subsection{Drools DSL}
