%!TEX ROOT=../diploma-thesis.tex

\chapter{Rešerše}\label{ch:reserse}

\section{Modelem ř\'{\i}zená architektura}

\todo{
\begin{itemize}
    \item Co to je MDA
    \item V\'yhody
    \item Nev\'yhody
    \item Shrnut\'{\i} a proč se nám nehod\'{\i}
\end{itemize}
}

\section{Architektura klient-server}\label{sec:client-server}

\todo{
\begin{itemize}
    \item Co to je
    \item V\'yhody
    \item Nev\'yhody
    \item Shrnut\'{\i} a proč se nám hod\'{\i}
    \item Citovat ~\cite{berson1992client}
\end{itemize}
}

\section{Aspektově orientované programován\'{\i}}

\goal{Co je paradigma}
Programován\'{\i} je komplexn\'{\i} discipl\'{\i}na s teoreticky
neomezen\'ym počtem možnost\'{\i}, jak\'ym programátor může
řešit zadan\'y problém. Ačkoliv každá úloha má své specifické
požadavky, za relativně krátkou historii programován\'{\i} se
stihlo ustálit několik ideologi\'{\i}, tzv. programovac\'{\i}ch
paradigmat, které programátorovi poskytuj\'{\i} sadu abstrakc\'{\i}
a základn\'{\i}ch principů~\cite{van2009programming}.
D\'{\i}ky znalosti paradigmatu může programátor nejen zlepšit
svou produktivitu, ale zároveň může snáze pochopit myšlenky
jiného programátora a t\'{\i}m zlepšit kvalitu t\'ymové spolupráce.

\goal{OOP a jeho popis}
Jedn\'{\i}m z nejpopulárnějš\'{\i}ch paradigmat použ\'{\i}van\'ych k
v\'yvoji modern\'{\i}ch enterprise systémů je nepochybně
objektově orientované programován\'{\i} (\gls{OOP}). To vn\'{\i}má dan\'y problém
jako množinu objektu, které spolu intereaguj\'{\i}. Program
člen\'{\i} na malé funkčn\'{\i} celky odpov\'{\i}daj\'{\i}c\'{\i} struktuře
reálného světa~\cite{rentsch1982object}. Je vhodné zm\'{\i}nit,
že objekty se rozum\'{\i} jak konkrétn\'{\i} koncepty, např\'{\i}klad
auto nebo člověk, tak i abstraktn\'{\i} koncepty,
namátkou bankovn\'{\i} transakce nebo objednávka v obchodě.
Objekty se pak prom\'{\i}taj\'{\i} do kódu programu i do
reprezentace struktur v paměti poč\'{\i}tače.
Tento př\'{\i}stup je velmi snadn\'y pro pochopen\'{\i},
vede k lepš\'{\i}mu návrhu a organizaci programu a snižuje
tak náklady na jeho v\'yvoj a údržbu.

\goal{Nedostatky OOP}
Ačkoliv je \gls{OOP} velmi siln\'ym a všestran\'ym nástrojem,
existuj\'{\i} problémy, které nelze jeho pomoc\'{\i} efektivně řešit.
Jedn\'{\i}m takov\'ym problémem jsou obecné požadavky na systém,
které musej\'{\i} b\'yt konzistentně dodržovány na v\'{\i}ce m\'{\i}stech,
které spolu zdánlivě nesouvis\'{\i}. Př\'{\i}kladem
může b\'yt logován\'{\i} systémov\'ych akc\'{\i}, optimalizace správy paměti
nebo uniformn\'{\i} zpracován\'{\i} v\'yjimek~\cite{kiczales1997aspect}.
Takové požadavky naz\'yváme \textit{průřezové problémy}
(z anglického \textit{cross-cutting concerns}).
V rámci \gls{OOP} je programátor nucen v ojektech manuálně opakovat
kód, kter\'y zodpov\'{\i}dá za jejich realizaci. Duplikace kódu
vede k větš\'{\i} náchylnosti na lidskou chybu a k vyšš\'{\i}m nárokům na v\'yvoj
a údržbu daného softwarového systému~\cite{fowler1999refactoring}.

\goal{AOP jako odpověď na nedostatky OOP}
Aspektově orientované programován\'{\i} (\gls{AOP}) přináš\'{\i} řešen\'{\i} na
v\'yše zmiňované problémy. Extrahuje obecné požadavky,
tzv. \textit{aspekty} do jednoho m\'{\i}sta a pomoc\'{\i} procesu zvaného
\textit{weaving} je poté automaticky distribuuje do systému.
Weaving může proběhnout staticky při kompilaci programu nebo dynamicky
při jeho běhu. V obou př\'{\i}padech ale programátorovi ulehčuje práci,
protože k definici i změně aspektu docház\'{\i} centrálně a t\'{\i}m je eliminována
potřeba manuáln\'{\i} duplikace kódu. Je nutno poznamenat, že \gls{AOP} nen\'{\i}
paradigmatem poskytuj\'{\i}c\'{\i}m kompletn\'{\i} framework pro návrh programu.
V ideáln\'{\i}m př\'{\i}padě je tedy k návrhu systému využita kombinace
\gls{AOP} s jin\'ym paradigmatem. Pro účely této práce se zaměř\'{\i}me na
kombinaci \gls{AOP} a \gls{OOP}.

\todo{
\begin{itemize}
    \item diagram cross cutting concerns
    \item roztáhnout do v\'{\i}ce odstavců
    \item diagram weaveru
    \item section BPEL
\end{itemize}
}

\begin{description}
    \item [Aspekt]
    \item [Join-point]
    \item [Pointcut]
    \item [Advice]
    \item [Weaving]
\end{description}

\section{Aspect-driven Design Approach}

\todo{
\begin{itemize}
    \item co to je
    \item jak nám to pomůže
    \item využ\'{\i}t\'{\i} jeho konceptů
    \item shrnut\'{\i}
\end{itemize}
}

Aspect-driven Design Approach (\gls{ADDA})

\goal{Vhodnost AOP pro náš úkol}
Vzhledem k požadavkům na implementaci našeho frameworku stanoven\'ym
v předchoz\'{\i} kapitole~\ref{ch:analyza} se \gls{AOP} a na něm stavěj\'{\i}c\'{\i} \gls{ADDA}
jev\'{\i} jako vhodn\'y př\'{\i}stup, kter\'y nám pomůže dosáhnout c\'{\i}le.

\section{Stávaj\'{\i}c\'{\i} řešen\'{\i} reprezentace business pravidel}

\subsection{Drools DSL}

\todo{
\begin{itemize}
    \item co to je
    \item jak to funguje
    \item RETE algoritmus
    \item v\'yhody
    \item nev\'yhody
    \item shrnut\'{\i} a proč se nám nehod\'{\i}
\end{itemize}
}

\goal{Drools se nám nehod\'{\i}, protože je jen pro platformu Java}
...

\subsection{JetBrains MPS}

\todo{
\begin{itemize}
    \item co to je
    \item jak to funguje
    \item v\'yhody
    \item nev\'yhody
    \item shrnut\'{\i} a proč se nám nehod\'{\i}
\end{itemize}
}

\goal{MPS je super, ale nevyhovuje nám kvůli dynamick\'ym změnám}
...

\section{Shrnut\'{\i}}

V této kapitole jsme provedli rešerši \textit{architektury orientované
na služby}, jej\'{\i}ch v\'yhod, nev\'yhod a znám\'ych nedostatků. Dále jsme
prozkoumali, jak\'y způsobem funguje s\'{\i}ťová \textit{architektura klient-server},
jaké jsou v\'yhody a nev\'yhody \textit{modelem ř\'{\i}zeného v\'yvoje} pro náš př\'{\i}pad
a shrnuli jsme paradigma \textit{aspektově orientovaného programován\'{\i}} a
z něch vycházej\'{\i}c\'{\i} př\'{\i}stup k návrhu softwarov\'ych systémů \textit{ADDA}.
Nakonec jsme provedli rešerši stávaj\'{\i}c\'{\i}ch řešen\'{\i} reprezentace byznys pravidel
včetně komplexn\'{\i}ho frameworku \textit{Drools} a zhodnotili jsme jeho vhodnost
k řešen\'{\i} našeho problému.
