%\usepackage[T1]{fontenc}
%\usepackage[utf8]{inputenc}

%!TEX ROOT=../diploma-thesis.tex

\chapter{Implementace prototypů knihoven}\label{ch:implementace}

\goal{Uveden\'{\i} kapitoly a nast\'{\i}něn\'{\i} obsahu}
Součást\'{\i} zadán\'{\i} této práce je implementace prototypů
knihoven pro framework navržen\'y v kapitole~\ref{ch:navrh}
pro tři rozd\'{\i}lné platformy, z nichž jedna mus\'{\i} b\'yt \textit{Java}.
Tato kapitola popisuje výběr plaforem a konkrétní implementace
knihoven pro tyto platformy. Jelikož jednotlivé implementace
vycházejí ze stejného návrhu, kompletní implementace
je opsána pouze pro platformu Java a ostatní jsou shrnuty
komparativní metodou. Součástí kapitoly je i stručný popis
použitých technologií pro lepší kontext.

\section{V\'yběr použit\'ych platforem}

%\goal{Jaké jsme vybrali dalš\'{\i} platformy a proč}
Mimo jazyk Java, kter\'y byl určen zadán\'{\i}m, byla pro
implementaci vybrána platforma jazyka \textit{Python} a platforma \textit{Node.js},
která slouž\'{\i} jako běhové prostřed\'{\i} pro jazyk \textit{JavaScript}.
V\'yběr byl proveden na základě aktuáln\'{\i}ch trendů ve světě softwarového
inžen\'yrstv\'{\i}~\cite{githut}\cite{octoverse}\cite{stackoverflowsurvey}.
Tyto jazyky se v posledních letech stabilně umísťují na prvních příčkách
nejpopulárnějš\'{\i}ch programovacích jazyků pro obecné použit\'{\i}.

\section{Sd\'{\i}len\'{\i} byznys kontextů mezi službami}

%\goal{Formát pro přenos pravidel po s\'{\i}ti a jeho v\'yhody}
Pro sd\'{\i}lení byznysových kontextů a jejich pravidel
mezi jednotliv\'ymi službami bude využita s\'{\i}ťová komunikace
ve formátu nezávislém na platformě, s vysokou efektivitou přenosu.

\subsection{Protocol Buffers}

%\goal{Proč jsme použili Protobuf}
Pro přenos byznysov\'ych kontextů byl zvolen open-source formát
\textit{Protocol Buffers}\footnote{https://developers.google.com/protocol-buffers/}\cite{varda2008protocol}
vyvinut\'y společnost\'{\i} Google\footnote{https://www.google.com/}.
Umožňuje explicitně definovat a vynucovat schéma dat,
která jsou přenášena po s\'{\i}ti, bez vazby na konkrétn\'{\i} programovac\'{\i}
jazyk. Zároveň poskytuje obslužné knihovny pro vybrané platformy.
D\'{\i}ky binárn\'{\i} reprezentaci dat v přenosu velmi efektivn\'{\i},
oproti formátům jako je \gls{JSON} nebo \gls{XML}~\cite{maeda2012performance}.
Oproti protokolům \textit{Apache Thrift}\footnote{https://thrift.apache.org/}
a \textit{Apache Avro}\footnote{https://avro.apache.org/}, které poskytuj\'{\i}
velmi srovnatelnou funkcionalitu, maj\'{\i} Protocol Buffers
kvalitnějš\'{\i} a lépe srozumitelnou dokumentaci.

\lstinputlisting[
caption={Část definice schématu zpráv byznys kontextů v jazyce Protobuffer},
label={lst:protobuf-example},
language=protobuf2,
style=protobuf,
%frame=single,
%float,
%floatplacement=H
]{code/protobuffer_example.proto}

Zdrojov\'y kód~\ref{lst:protobuf-example} znázorňuje zápis schématu
síťových zpráv pro distribucu byznys kontexty ve formátu Protobuffer.
Schéma zpráv pro v\'yměnu kontextů dodržuje strukturu metamodelu navrženého
v sekci~\ref{sec:metamodel}.

\begin{description}
    \item [ExpressionMessage] obsahuje jméno, atributy a argumenty \code{Expression}
    \item [ExpressionPropertyMessage] je enumerace obsahuj\'{\i}c\'{\i} typy atributu \code{Expression}
    \item [PreconditionMessage] obsahuje název a podm\'{\i}nku precondition pravidla
    \item [PostConditionMessage] obsahuje název, typ, název odkazovaného pole a podm\'{\i}nku post-condition pravidla
    \item [PostConditionTypeMessage] je enumerace obsahuj\'{\i}c\'{\i} typy post-condition pravidla
    \item [BusinessContextMessage] obsahuje identifikátor, seznam rožš\'{\i}řen\'ych kontextů, seznam preconditions a post-conditions byznys kontextu
    \item [BusinessContextsMessage] obaluje v\'{\i}ce byznys kontextů
\end{description}

\subsection{gRPC}

%\goal{Proč jsme použili gRPC}
Pro realizaci architektury klient-server byl zvolen
open-source framework gRPC\footnote{https://grpc.io/}, kter\'y stav\'{\i}
na technologii Protocol Buffers a poskytuje v\'yvojáři
možnost definovat detailn\'{\i} schéma komunikace pomoc\'{\i}
protokolu \textit{\gls{RPC}}~\cite{nelson1981remote}.
Zdrojov\'y kód~\ref{lst:grpc-example} znázorňuje zápis serveru,
kter\'y umožňuje volat metody \code{FetchContexts},
\code{FetchAllContexts} a \code{UpdateOrSaveContext}.

\lstinputlisting[
caption={Definice služby pro komunikaci byznys kontextů pro gRPC},
label={lst:grpc-example},
language=protobuf2,
style=protobuf,
%frame=single,
%float,
%floatplacement=H
]
{code/grpc_example.proto}

\paragraph{FetchContexts()} je metoda, která umožňuje klientovi
z\'{\i}skat kontexty, jejichž identifikátory zašle jako argument
typu \code{BusinessContextRequestMessage}.
V odpovědi pak obdrž\'{\i} dotazované kontexty a nebo chybovou hlášku,
pokud kontexty s dan\'ymi identifikátory nemá server k dispozici.

\paragraph{FetchAllContexts()} dovoluje klientovi z\'{\i}skat všechny
dostupné kontexty serveru. Tato metoda je využ\'{\i}vána pro administraci
kontextů, kdy je potřeba z\'{\i}skat všechny kontexty všech služeb, aby
nad nimi mohly prob\'{\i}hat úpravy a anal\'yzy.

\paragraph{UpdateOrSaveContext()} slouž\'{\i} pro uložen\'{\i} nového či
editovaného pravidla, které je zasláno v serializované podobě
jako jedin\'y argument typu \code{BusinessContextUpdateRequestMessage}.

\section{Doménově specifick\'y jazyk pro popis byznys kontextů}\label{sec:dsl-impl}

%\goal{Popsat proč a jak jsme tvořili DSL}
Ačkoliv nen\'{\i} specifikace a implementace \gls{DSL} pro popis byznysových kontextů
úkolem této práce, pro ověřen\'{\i} konceptu je nutné nadefinovat alespoň jeho zjednodušenou
verzi a implementovat část knihovny, která bude umět jazyk zpracovat a sestavit metamodel popsaného
kontextu. Tento jazyk však bude možno v produkční verzi knihovny nahradit komplexnejším.

%\goal{Důvody pro v\'yběr XML}
Pro popis kontextů byl jako kompromis mezi jednoduchost\'{\i} implementace
a př\'{\i}větivost\'{\i} pro koncového uživatele zvolen univerzáln\'{\i} formát Extensible
Markup Language (\gls{XML})~\cite{bray1997extensible} doplněný o definici schematu dat
pomocí \textit{XML Schema Definition} (\gls{XSD})~\cite{lee2000comparative}.
D\'{\i}ky formálně definovanému schématu lze popis byznys kontextu
automaticky validovat a vyvarovat se tak př\'{\i}padn\'ych chyb.

%\goal{Popis formátu}
Ve zdrojovém kódu~\ref{lst:business-context-xml} je znázorněn
př\'{\i}klad zápisu jednoduchého byznys kontextu s jednou precondition.
Samotn\'y zápis byznys kontextu je obsažen v kořenovém elementu
\code{<businessContext>} a jeho název je popsán atributy
\code{prefix} a \code{name}. Identifikátory rozš\'{\i}řených kontextů jsou
vypsány v entitě \code{<includedContexts>}. Preconditions jsou
definovány uvnitř entity \code{<preconditions>} a podobně
jsou definovány \code{<postconditions>}. Obsažená data odpov\'{\i}daj\'{\i}
navrženému metamodelu byznysového kontextu v sekci~\ref{sec:metamodel}.
Pro zápis podm\'{\i}nek jednotliv\'ych preconditions a post-conditions byl zvolen
opis derivačního stromu. Toto rozhodnut\'{\i} vycház\'{\i} z předpokladu,
že lze vzhledem k povaze prototypu relaxovat podm\'{\i}nku
na čitelnost zápisu pravidel ve prospěch jednodušš\'{\i}ho zpracován\'{\i}.

\lstinputlisting[
caption={Př\'{\i}klad zápisu byznys kontextu v jazyce \gls{XML}},
label={lst:business-context-xml},
language=XML,
%frame=single,
%float,
%floatplacement=H
]
{code/business_context.xml}

\section{Knihovna pro platformu Java}

\todo{
\begin{itemize}
    \item Popis business context registry
    \item Popis expression AST
    \item Popis tř\'{\i}d kolem business kontextu
    \item Popis XML parseru a generátoru
    \item Popis server a klient tř\'{\i}d pro obsluhu GRPC
    \item Popis weaveru
    \item Popis anotac\'{\i} pro AOP
\end{itemize}
}

\subsection{Popis implementace}

\paragraph{BusinessContextRegistry}

\lstinputlisting[
caption={Označen\'{\i} operačn\'{\i}ho kontextu a jeho parametrů pomoc\'{\i} anotac\'{\i} Java knihovny},
label={lst:business-operation-aspectj},
language=Java,
%frame=single,
%float,
%floatplacement=H
]
{code/business_operation_aspectj.java}

\subsection{Použité technologie}

\paragraph{Apache Maven}

%\goal{Správa závislost\'{\i} a buildu projektu}
Pro správu závislost\'{\i} a automatickou kompilaci a sestavován\'{\i}
knihovny napsané v jazyce java byl zvolen projekt
\textit{Maven}\footnote{https://maven.apache.org/}.
Tento nástroj umožňuje v\'yvojáři komfortně a centrálně
spravovat závislosti jeho projektu včetně detailn\'{\i}ho
popisu jejich verze. Dále také umožňuje definovat jak\'ym
způsobem bude projekt kompilován.

\paragraph{AspectJ}

%\goal{Proč AspectJ a co to um\'{\i}}
Knihovna AspectJ přináš\'{\i} pro jazyk Java sadu nástrojů,
d\'{\i}ky kter\'ym lze snadno implementovat koncepty aspektově orientovaného
programován\'{\i}, zejména pak snadn\'y zápis pointcuts a kompletn\'{\i}
engine pro weaving aspektů.

\todo{
\begin{itemize}
    \item Ukázka kódu knihovny
\end{itemize}
}

\paragraph{JDOM 2}

%\goal{Proč jdom2 a co to um\'{\i}}
Knihovna JDOM 2\footnote{http://www.jdom.org/} poskytuje
kompletn\'{\i} sadu nástroju pro čten\'{\i} a zápis \gls{XML} dokumentů.
Implementuje specifikaci \textit{Document Object Model} (\gls{DOM})~\cite{wood2004document},
pomoc\'{\i} které lze automaticky sestavovat a č\'{\i}st \gls{XML} dokumenty.

\section{Knihovna pro platformu Python}

Knihovna pro platformu jazyka Python využ\'{\i}vá jeho
verzi 3.6. Pomoc\'{\i} nástroje \textit{pip}\footnote{https://pip.pypa.io/en/stable/}
lze knihovnu nainstalovat a využ\'{\i}vat jako python modul.
Implementace odpov\'{\i}dá navržené specifikaci.

\todo{
\begin{itemize}
    \item Srovnán\'{\i} řešen\'{\i} s knihovnou Java
    \item Problémy pythonu a jak byly vyřešeny
    \item Ukázka kódu knihovny
    \item Použité technologie
    \item Ukázka AOP v pythonu pomoc\'{\i} vestavěn\'ych dekorátorů
    \item Knihovna pro GRPC
    \item Popis weaveru
\end{itemize}
}

\subsection{Srovnán\'{\i} s knihovnou pro platformu Java}

\paragraph{Weaving} Největš\'{\i}m rozd\'{\i}lem oproti knihovně pro
jazyk Java je implementace weavingu byznys kontextů.
Jazyk Python totiž d\'{\i}ky své dynamické povaze a vestavěnému
systému dekorátorů umožňuje aplikovat principy aspektově orientovaného
programován\'{\i} bez potřeby dodatečn\'ych knihoven či technologi\'{\i}.
Zdrojov\'y kód~\ref{lst:python-weaving-example} znázorňuje
definici a použit\'{\i} dekorátoru \code{business\textunderscore operation}.
Dekorátoru je potřeba předat samotn\'y weaver, narozd\'{\i}l
od implementace v Javě, kdy se o předán\'{\i} weaveru postará dependency
injection container.

\lstinputlisting[
caption={Př\'{\i}klad použit\'{\i} dekorátorů pro weaving v jazyce Python},
label={lst:python-weaving-example},
language=Python,
%frame=single,
%float,
%floatplacement=H
]
{code/python_weaving.py}

\paragraph{}

\subsection{Použité technologie}

\section{Knihovna pro platformu Node.js}

Knihovna pro platformu \textit{Node.js} byla implementována
v jazyce JavaScript, konkrétně jeho verzi
ECMAScript 6.0~\cite{ecma60}.
Implementace odpov\'{\i}dá specifikaci návrhu, umožňuje
instalaci pomoc\'{\i} bal\'{\i}čkovac\'{\i}ho nástroje a snadnou
integraci do kódu v\'ysledné služby.

\subsection{Srovnán\'{\i} s knihovnou pro platformu Java}

\paragraph{Weaving} Podobně jako v knihovně pro jazyk Python,
i v knihovně pro Node.js byl oproti knihovně pro jazyk Java
největš\'{\i} rozd\'{\i}l v implementaci weavingu. Platforma Node.js
totiž nedisponuje žádnou kvalitn\'{\i} knihovnou, která by ulehčila
využit\'{\i} konceptů aspektově orientovaného programován\'{\i}.
Jazyk JavaScript je ale velmi flexibiln\'{\i} a lze
tedy pro dosažen\'{\i} požadované funkcionality využ\'{\i}t
podobně jako pro jazyk Python princip dekorátoru jako funkce.
Ukázka je ve zdrojovém kódu~\ref{lst:nodejs-weaving}.
Funkce \code{register()} obsahuje logiku pro registraci uživatele,
která může obsahovat např\'{\i}klad uložen\'{\i} entity do databáze a odeslán\'{\i}
registračn\'{\i}ho e-mailu. Při exportován\'{\i} funkce z Node.js modulu
využijeme \code{wrapCall()}, která má za úkol dekorovat
předanou funkci \code{func}, před jej\'{\i}m zavolán\'{\i} vyhodnotit
preconditions a po zavolán\'{\i} aplikovat post-conditions.
D\'{\i}ky tomu bude každ\'y kód, kter\'y využije modul definuj\'{\i}c\'{\i} funkci
pro registraci uživatele, pracovat s dekorovanou funkc\'{\i}.

\paragraph{Využit\'{\i} gRPC} Narozd\'{\i}l od implementac\'{\i} knihovny
v jazyc\'{\i}ch Java a Python um\'{\i} knihovna obsluhuj\'{\i}c\'{\i} gRPC
fungovat i bez předgenerovaného kódu. To poněkud usnadnilo
práci při serializaci byznys kontextů do přenosového formátu
i při deserializaci a ukládán\'{\i} kontextů do paměti. Úspora
kódu je ale na úkor typové kontroly a tak může b\'yt kód náchylnějš\'{\i}
na lidskou chybu.

\lstinputlisting[
caption={Př\'{\i}klad dekorace funkce v JavaScriptu pro aplikaci weavingu},
label={lst:nodejs-weaving},
language=JavaScript,
%frame=single,
%float,
%floatplacement=H
]
{code/nodejs_weaving.js}

\subsection{Použité technologie}

%\goal{Použité technologie pro v\'yvoj knihovny}
Podobně jako byl použit nástroj Maven pro knihovnu v jazyce Java byl
využit bal\'{\i}čkovac\'{\i} nástroj \textit{NPM}, kter\'y je předinstalován
v běhovém prostřed\'{\i} \textit{Node.js}. Tento nástroj ale nedisponuje
př\'{\i}liš silnou podporou pro správu automatick\'ych sestaven\'{\i} knihovny
a v základn\'{\i}m nastaven\'{\i} nen\'{\i} ani př\'{\i}liš efektivn\'{\i} pro správu závislost\'{\i}.
Proto bylo nutné využ\'{\i}t dodatečné knihovny, jmenovitě
\textit{Yarn}\footnote{https://yarnpkg.com/en/}, \textit{Babel}\footnote{https://babeljs.io/} a
\textit{Rimraf}\footnote{https://github.com/isaacs/rimraf}.

\section{Systém pro centráln\'{\i} správu byznys pravidel}\label{sec:central-administration}

Součástí této práce je i implementace nástroje, který umožní centrální správu byznysových
pravidel
\subsection{Popis implementace}

Systém pro centrální správu ...

Pro komfortn\'{\i} obsluhu centráln\'{\i} administrace bylo naprogramováno
uživatelské rozhran\'{\i} pomoc\'{\i} technologi\'{\i} Hypertext Markup Language~\cite{berners1995hypertext}
(HTML) a Cascading Style Sheets~\cite{bos1998cascading} (\gls{CSS}).
Detail byznysového kontextu v uživatelském rozhran\'{\i} je zobrazen
na sn\'{\i}mku~\ref{fig:screenshot-context-detail} a formulář pro úpravu kontextu
na sn\'{\i}mku~\ref{fig:screenshot-context-edit}.

\todo{
\begin{itemize}
    \item Uživatelské rozhran\'{\i} v HTML + CSS
    \item Jak jsme použili Spring Boot a jeho MVC k nastaven\'{\i} základn\'{\i} webové aplikace
    \item Dependency Injection Container
    \item Využit\'{\i} knihovny pro platformu Java
\end{itemize}
}

\subsection{Detekce a prevence potenciáln\'{\i}ch problémů}

%\goal{Problémy způsobené rozšiřován\'{\i}m kontextů}
Sekce~\ref{sec:context-inheritance} identifikuje problémy, které
mohou nastat při úpravě nebo vytvářen\'{\i} nového byznysového kontextu.
Kromě syntaktick\'ych chyb, které jsou detekovány automaticky pomoc\'{\i} definovaného schematu,
je potřeba detekovat následující sémantické chyby, které mohou b\'yt způsobeny rozšiřován\'{\i}m kontextů.
\begin{enumerate}[label=\alph*)]
    \item Neunikátní identifikátory byznysových pravidel
    \item Závislosti na neexistuj\'{\i}c\'{\i}ch kontextech
    \item Cyklus v grafu závislost\'{\i} kontextů
\end{enumerate}

%\goal{Chápán\'{\i} kontextů jako grafu}
Unikátnost byznysových pravidel lze zajistit postupným ukládáním jejich identifikátorů
do vhodné datové struktury a kontrolovat, zda již nejsou ve struktuře obsaženy. Vhodnou
strukturou je například \code{Set}~\cite{hopcroft1983data}.
Kontexty a jejich vzájemné závislosti lze vn\'{\i}mat jako
orientovan\'y graf, kde uzel grafu reprezentuje kontext
a orientovaná hrana reprezentuje závislost mezi kontexty.
Směr závislosti lze pro tento účel zvolit libovolně.
Pro detekci závislosti na neexistuj\'{\i}c\'{\i}ch kontextech je nejprve
sestaven seznam existuj\'{\i}c\'{\i}ch kontextů a následně jsou navštíveny
jednotlivé hrany grafu kontextů a je ověřeno, zda existuj\'{\i} oba kontexty
nálež\'{\i}c\'{\i} dané hraně. Při zvolen\'{\i} vhodn\'ych datov\'ych struktur
lze dosáhnout lineárn\'{\i} složitosti v závislosti na počtu hran grafu.
Pro detekci cyklů v grafu závislosti pravidel popsanou v sekci~\ref{sec:context-inheritance} byl
zvolen Tarjanův algoritmus~\cite{tarjan1971depth}. Ten umožňuje detekci souvisl\'ych
komponent a má lineárn\'{\i} složitost závislou na součtu počtu hran a
počtu uzlů grafu. V př\'{\i}padě, že zápis nového či upraveného kontextu obsahuje syntaktické
nebo sémantické chyby, administrace nedovol\'{\i} uživateli změnu provést a vyp\'{\i}še informativn\'{\i}
chybovou hlášku.

\section{Shrnutí}

%\goal{Dosáhli jsme vytyčen\'ych c\'{\i}lů implementace}
Na základě navrženého frameworku byly implementovány prototypy
knihoven pro platformy jazyka Java, jazyka Python a
Node.js. Knihovny umožňuj\'{\i} centráln\'{\i} správu a automatickou distribuci a integraci
byznysov\'ych kontextů, včetně vyhodnocován\'{\i} jejich pravidel, za
použit\'{\i} aspektově orientovaného př\'{\i}stupu.
V rámci této kapitoly byl specifikován \gls{DSL}, kter\'ym lze popsat
byznys kontext nezávisle na platformě.
Implementované protoypy knihoven lze využ\'{\i}t k implementaci služeb a k sestaven\'{\i}
funkčn\'{\i}ho systému, jak je popisáno v následuj\'{\i}c\'{\i} kapitole.

%\goal{Hostován\'{\i} na GitHubu + licence}
Vešker\'y kód implementace je hostován v centráln\'{\i}m repozitáři
ve službě GitHub\footnote{https://github.com/klimesf/diploma-thesis}
a je zpř\'{\i}stupněn pod open-source licenc\'{\i} MIT~\cite{mitlicense}.
Knihovny pro jednotlivé platformy tedy lze libovolně
využ\'{\i}vat, modifikovat a š\'{\i}řit.
