\usepackage[T1]{fontenc}
\usepackage[utf8]{inputenc}

%!TEX ROOT=../diploma-thesis.tex

\chapter{Implementace}\label{ch:implementace}

V této kapitole si popíšeme jakým způsobem byl implementován
navržený framework pro platformy Java, Python a Node.js.

\section{Sdílení byznys kontextů mezi službami}

% Přenášení business contextů
Pro sdílení byznys kontextů mezi jednotlivými službami
je potřeba je přenášet po síti ve formátu, který bude
nezávislý na platformách jednotlivých služeb.

\subsection{Schéma síťové komunikace}

\subsection{Použité technologie}

\subsubsection{Protocol Buffers}

Pro přenos pravidel byl zvolen open-source formát
\textit{Protocol Buffers}\footnote{
https://developers.google.com/protocol-buffers/
}
vyvinutý společností Google\footnote{
https://www.google.com/
}. Umožňuje explicitně definovat a vynucovat schéma dat,
která jsou přenášena po síti, bez vazby na konkrétní programovací
jazyk. Zároveň je díky binární reprezentaci v přenosu velmi efektivní,
oproti formátům jako je JSON nebo XML~\cite{varda2008protocol}.

% TODO: snippet z protobufferu

\subsubsection{gRPC}

Pro komunikaci byznys kontextů nám nestačí pouze přenosový formát,
je potřeba také popsat schéma samotné komunikace. K tomu byl zvolen
open-source framework gRPC\footnote{
https://grpc.io/
}, který staví na technologii Protocol Buffers a poskytuje vývojáři
možnost definovat komunikaci pomocí protokolu RPC.

\section{Knihovna pro platformu Java}

\subsection{Použité technologie}

\subsubsection{Apache Maven}

\subsubsection{AspectJ}

\section{Knihovna pro platformu Python}

\subsection{Použité technologie}

\subsection{}

\section{Knihovna pro platformu Node.js}


\subsection{Použité technologie}

\subsection{NPM}

Node.js package manager

\section{Systém pro centrální správu byznys pravidel}

\subsection{Použité technologie}

\subsection{Detekce cyklů}

\section{Shrnutí}

Veškerý kód je hostován v centrálním repozitáři
ve službě GitHub\footnote{
https://github.com/klimesf/diploma-thesis
} a je zpřístupněn pod open-source licencí MIT\footnote{
http://www.linfo.org/mitlicense.html
}. Knihovny pro jednotlivé platformy tedy lze libovolně
využívat, modifikovat a šířit.
