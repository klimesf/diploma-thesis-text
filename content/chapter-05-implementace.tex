\usepackage[T1]{fontenc}
\usepackage[utf8]{inputenc}

%!TEX ROOT=../diploma-thesis.tex

\chapter{Implementace prototypů knihoven}\label{ch:implementace}

Součástí zadání této práce je implementace prototypů
knihoven pro framework navržený v kapitole~\ref{ch:navrh}
pro tři rozdílné platformy, z nichž jedna musí být \textit{Java}.
V této kapitole si popíšeme jaké plaformy jsme vybraly a jakým
způsobem byly prototypy knihoven implementovány. Součástí
kapitoly je i stručná rešerše technologií, které byly použity
pro dosažení vytyčených cílů.

% Technické implementační problémy
Pro splnění cílů bylo potřeba vyřešit také několik technických otázek.
Těmi je přenos byznys kontextů mezi jednotlivými službami, podpora
aspektově orientovaného programování v daném programovacím jazyce
a využití principu \textit{runtime weavingu} a integrace knihoven
do služeb, které je budou využívat.

\section{Výběr použitých platforem}

Mimo jazyk Java, který byl určen zadání, byl pro
implementaci vybrán jazyk \textit{Python}
a ekosystém \textit{Node.js}, který slouží jako
běhové prostředí pro jazyk \textit{JavaScript}.
Výběr byl proveden na základě aktuálních trendů
ve světě softwarového inženýrství. Projekt GitHut\footnote{http://githut.info/}
z roku 2014, který shrnuje statistiky repozitářů
populární služby pro hosting a sdílení kódu
GitHub\footnote{https://github.com/}, určil
jazyky JavaScript, Java a Python jako tři nejaktivnější.
Služba GitHub následně sama zveřejnila statistiky za rok 2017
v rámci projektu Octoverse\footnote{https://octoverse.github.com/}
a dospěla ke stejnému závěru, ačkoliv Python se umístil na druhé
pozici na úkor jazyka Java. Podle průzkumu oblíbeného
programátorského webového portálu Stack
Overflow\footnote{https://insights.stackoverflow.com/survey/2017\#technology}
se umístily tyto jazyky v první čtveřici nejpopulárnějších jazyků pro obecné použití.

\section{Sdílení byznys kontextů mezi službami}

% Přenášení business contextů

Pro sdílení byznys kontextů mezi jednotlivými službami
je potřeba je přenášet po síti ve formátu, který bude
nezávislý na platformách jednotlivých služeb.

\subsection{Síťová komunikace}

% Formát pro přenos pravidel po síti a jeho výhody
Abychom mohli přenášet byznysové kontexty a jejich pravidla
po síti, musíme zvolit protokol a jednotný formát, ve kterém
spolu budou jednotlivé služby komunikovat.
Tento formát tedy musí být nezávislý na platformě a ideálně
by měl být co nejefektivnější v rychlosti přenosu.

\subsection{Použité technologie}

\subsubsection{Protocol Buffers}


Pro přenos byznysových kontextů byl zvolen open-source formát
\textit{Protocol Buffers}\footnote{
https://developers.google.com/protocol-buffers/
}
vyvinutý společností Google\footnote{
https://www.google.com/
}. Umožňuje explicitně definovat a vynucovat schéma dat,
která jsou přenášena po síti, bez vazby na konkrétní programovací
jazyk. Zároveň je díky binární reprezentaci v přenosu velmi efektivní,
oproti formátům jako je JSON nebo XML~\cite{varda2008protocol}.
Zrojový kód~\ref{lst:protobuf-example} znázorňuje zápis schématu
zasílaných zpráv ve formátu Protobuffer.

\lstinputlisting[caption={Příklad definice schématu zpráv v jazyce Protobuffer},label={lst:protobuf-example}language=Proto]{code/protobuffer_example.proto}


\subsubsection{gRPC}

Pro komunikaci byznys kontextů nám nestačí pouze přenosový formát,
je potřeba také popsat schéma samotné komunikace. K tomu byl zvolen
open-source framework gRPC\footnote{
https://grpc.io/
}, který staví na technologii Protocol Buffers a poskytuje vývojáři
možnost definovat komunikaci pomocí protokolu RPC.

\lstinputlisting[caption={gRPC},label={lst:grpc-example}language=Protobuf]{code/grpc_example.proto}

\section{Knihovna pro platformu Java}

\subsection{Použité technologie}

\subsubsection{Apache Maven}

\subsubsection{AspectJ}

\section{Knihovna pro platformu Python}

\subsection{Použité technologie}

\subsection{}

\section{Knihovna pro platformu Node.js}


\subsection{Použité technologie}

\subsection{NPM}

Node.js package manager

\section{Systém pro centrální správu byznys pravidel}

\subsection{Použité technologie}

\subsection{Detekce cyklů}

\section{Shrnutí}

% Hostování na GitHubu + licence
Veškerý kód je hostován v centrálním repozitáři
ve službě GitHub\footnote{
https://github.com/klimesf/diploma-thesis
} a je zpřístupněn pod open-source licencí MIT\footnote{
http://www.linfo.org/mitlicense.html
}. Knihovny pro jednotlivé platformy tedy lze libovolně
využívat, modifikovat a šířit.
