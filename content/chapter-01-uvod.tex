%\usepackage[T1]{fontenc}
%\usepackage[utf8]{inputenc}

%!TEX ROOT=../diploma-thesis.tex

\chapter{Úvod}\label{ch:uvod}

\todo{
\begin{itemize}
    \item popis tématu a jeho důležitost
    \item motivace práce
    \item co bude c\'{\i}lem a požadovan\'ym v\'ystupem této práce
    \item popis struktury DP
\end{itemize}
}

\section{Jak č\'{\i}st tuto práci}

\goal{Popis struktury DP a obsah kapitol}
Kapitola~\ref{ch:analyza} se věnuje detailn\'{\i} anal\'yze problematiky bynysov\'ych pravidel a
architektury orientované na služby, včetně jej\'{\i}ho historického v\'yvoje až po nejnovějš\'{\i} trendy
a v závěru identifikuje požadavky kladené na implementaci knihovny pro centráln\'{\i} správu a
automatickou distribuci byznysov\'ych pravidel v této architektuře. Kapitola~\ref{ch:reserse}
se zab\'yvá rešerš\'{\i} stávaj\'{\i}c\'{\i}ch pr\'{\i}stupu k v\'yvoji informacn\'{\i}ch systému a speciálně se zaměřuje
na koncepty aspektově orientovaného programován\'{\i} a modern\'{\i}ho aspekty ř\'{\i}zeného př\'{\i}stupu k návrhu
systémů. Dále se kapitola věnuje průzkumu existuj\'{\i}c\'{\i}ch nástrojů pro správu byznysov\'ych pravidel.
Kapitola~\ref{ch:navrh} formalizuje prostřed\'{\i} architektury orientované na služby do terminologie
aspektově orientovaného programován\'{\i} a na základě této formalizace navrhuje koncept frameworku,
kter\'y realizuje centráln\'{\i} správu a automatickou distribuci byznysov\'ych pravidel.
V kapitole~\ref{ch:implementace} je detailně probrána implementace knihoven pro navržen\'y framework
pro platformy jazyků Java a Python a frameworku Node.js. Následuj\'{\i}c\'{\i} kapitola~\ref{ch:verifikace}
popisuje, jak\'ym způsobem byly tyto knihovny otestovány a jak byla prokázána jejich funkčnost. Zároveň
je zde popsána validace a vyhodnocen\'{\i} konceptu frameworku jeho nasazen\'{\i}m při v\'yvoji
jednoduchého ukázkového e-commerce systému. V posledn\'{\i} kapitole~\ref{ch:zaver} je shrnuto, jak\'ych
c\'{\i}lů bylo v práci dosáhnuto a jak\'ym dalš\'{\i}m směrem se může v\'yzkum v této oblasti ub\'{\i}rat.

\goal{Zkratky}
Zkratky použ\'{\i}vané v textu jsou vždy nejdř\'{\i}ve zavedeny a vysvětleny. Pro přehlednost byl ale zároveň
souhrn použ\'{\i}van\'ych zkratek přidán jako př\'{\i}loha práce~\ref{ch:shortcuts}.
