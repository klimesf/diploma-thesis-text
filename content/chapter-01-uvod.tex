%!TEX ROOT=../diploma-thesis.tex

\chapter{Úvod}\label{ch:uvod}

%\goal{Informační systémy a jejich důležitost}
Informační systémy se ve 21. století staly neodmyslitelnou součástí našich každodenních životů.
Do styku s nimi přicházíme jak při výkonu našich povolání, tak ve volném čase. Usnadňují
řadu našich činností, od vzdělání a vědy, kde umožňují snadný přístup ke studijním materiálům,
přes zdravotnictví, kde pomáhají zvyšovat efektivitu a úroveň péče o pacienty~\cite{fichman2011editorial},
až po sociální sítě, kde umožňují lidem globálně komunikovat a sdílet své myšlenky, pocity a zážitky.
Očekávání na kvalitu a množství funkcí informačních systémů se zvyšuje každým rokem.
Jedním z úkolů výzkumu v oblasti softwarového inženýrství je zjednodušení a zefektivnění
vývoje informačních systémů. Díky tomu budou tyto systémy moci splňovat stále rostoucí množství požadavků.

%\goal{SOA}
Náročnost vývoje některých informačních systémů překračuje možnosti jednotlivců, ale
i celých týmů či skupin. Tyto systémy často využívají větší počet různorodých technologií kvůli
širokému spektru funkcionality, kterou nabízejí. Jedním z přístupů, který tyto problémy řeší,
je použití architektury orientované na služby. Ta se zaměřuje na sestavení systému z menších, vzájemně
nezávislých celků, tzv. služeb. Každá služba pak zastřešuje pouze část funkcionality systému.
Tím je umožněno využívat teoreticky neomezené množství technologií a rozdělit práci na systému mezi více nezávislých
vývojářských týmu.

%\goal{Byznysová pravidla}
Tato architektura bohužel nepřináší odpověď na všechny problémy, které je potřeba v informačních
systémech řešit. Jak je popsáno v následujících kapitolách, jedním z těchto problémů jsou tzv. byznysová
pravidla. Ta slouží k zajištění správné funkcionality systému a konzistenci uložených dat.
Některá tato pravidla zasahují do celého systému, tedy i do více služeb.
To při použití konvenčního přístupu přináší nutnost manuální duplikace zdrojového
kódu a tím jsou zvýšeny náklady na vývoj systému a riziko lidské chyby.

%\goal{Motivace a cíle}
Cílem této práce je prozkoumat myšlenku inovativního přístupu k centrální správě a automatické
integraci byznysových pravidel v systémech využívajících architekturu orientovanou na služby
a navrhnout framework, který by umožnil tento přístup uplatnit v praxi.
Tento koncept by měl díky využití aspektově orientovaného programování usnadnit práci vývojářů
a doménových expertů. Díky tomu by mohl přinést snížení nákladů na vývoj a údržbu informačních systémů
a tím zvýšit jejich kvalitu a snížit náklady na jejich vývoj.

%\goal{Popis struktury DP a obsah kapitol}
Kapitola~\ref{ch:analyza} se věnuje detailn\'{\i} anal\'yze problematiky bynysov\'ych pravidel a
architektury orientované na služby, včetně jej\'{\i}ho historického v\'yvoje až po nejnovějš\'{\i} trendy,
a v závěru identifikuje požadavky kladené na framework pro centráln\'{\i} správu a
automatickou integraci byznysov\'ych pravidel v této architektuře. Kapitola~\ref{ch:reserse}
se zab\'yvá rešerš\'{\i} stávaj\'{\i}c\'{\i}ch pr\'{\i}stupu k v\'yvoji informacn\'{\i}ch systému a speciálně se zaměřuje
na koncepty aspektově orientovaného programován\'{\i} a modern\'{\i}ho aspekty ř\'{\i}zeného př\'{\i}stupu k návrhu
systémů. Dále se kapitola věnuje průzkumu existuj\'{\i}c\'{\i}ch nástrojů pro správu byznysov\'ych pravidel a existujícím
síťovým architekturám, které budou sloužit pro distribuci byznysových pravidel mezi službami.
Kapitola~\ref{ch:navrh} formalizuje prostřed\'{\i} architektury orientované na služby do terminologie
aspektově orientovaného programován\'{\i} a na základě této formalizace navrhuje koncept frameworku,
kter\'y realizuje centráln\'{\i} správu a automatickou integraci byznysov\'ych pravidel.
V kapitole~\ref{ch:implementace} je detailně probrána implementace knihoven pro navržen\'y framework
na platformách jazyků Java a Python a frameworku Node.js. Následuj\'{\i}c\'{\i} kapitola~\ref{ch:verifikace}
popisuje, jak\'ym způsobem byly tyto knihovny otestovány a jak byla prokázána jejich funkčnost. Zároveň
je zde popsána validace a vyhodnocen\'{\i} konceptu frameworku jeho nasazen\'{\i}m při v\'yvoji
jednoduchého ukázkového e-commerce systému. V posledn\'{\i} kapitole~\ref{ch:zaver} je shrnuto, jak\'ych
c\'{\i}lů bylo v práci dosáhnuto a jak\'ym dalš\'{\i}m směrem se může v\'yzkum v této oblasti ub\'{\i}rat.
