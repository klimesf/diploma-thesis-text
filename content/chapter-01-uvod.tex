%\usepackage[T1]{fontenc}
%\usepackage[utf8]{inputenc}

%!TEX ROOT=../diploma-thesis.tex

\chapter{Úvod}\label{ch:uvod}

\todo{
\begin{itemize}
    \item popis tématu a jeho důležitost
    \item motivace práce
    \item co bude cílem a požadovaným výstupem této práce
    \item popis struktury DP
\end{itemize}
}

\section{Jak číst tuto práci}

\goal{Popis struktury DP a obsah kapitol}
Kapitola~\ref{ch:analyza} se venuje detailní analýze problematiky bynysových pravidel a
architektury orientované na služby, včetně jejího historického vývoje až po nejnovější trendy
a v závěru identifikuje požadavky kladené na implementaci knihovny pro centrální správu a
automatickou distribuci byznysových pravidel v této architektuře. Kapitola~\ref{ch:reserse}
se zabývá rešerší stávajících prístupu k vývoji informacních systému a speciálně se zaměřuje
na koncepty aspektově orientovaného programování a moderního aspekty řízeného přístupu k návrhu
systémů. Dále se kapitola věnuje průzkumu existujících nástrojů pro správu byznysových pravidel.
Kapitola~\ref{ch:navrh} formalizuje prostředí architektury orientované na služby do terminologie
aspektově orientovaného programování a na základě této formalizace navrhuje koncept frameworku,
který realizuje centrální správu a automatickou distribuci byznysových pravidel.
V kapitole~\ref{ch:implementace} je detailně probrána implementace knihoven pro navržený framework
pro platformy jazyků Java a Python a frameworku Node.js. Následující kapitola~\ref{ch:verifikace}
popisuje, jakým způsobem byly tyto knihovny otestovány a jak byla prokázána jejich funkčnost. Zároveň
je zde popsána validace a vyhodnocení konceptu frameworku jeho nasazením při vývoji
jednoduchého ukázkového e-commerce systému. V poslední kapitole~\ref{ch:zaver} je shrnuto, jakých
cílů bylo v práci dosáhnuto a jakým dalším směrem se může výzkum v této oblasti ubírat.
