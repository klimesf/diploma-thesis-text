%\usepackage[T1]{fontenc}
%\usepackage[utf8]{inputenc}

%!TEX ROOT=../diploma-thesis.tex

\chapter{Závěr}\label{ch:zaver}

\section{Anal\'yza dopadu použit\'{\i} frameworku}

\section{Budouc\'{\i} rozšiřitelnost frameworku}

\subsection{Kvalitní doménově specifický jazyk}
Zadáním této práce nebylo zkonstruovat vlastní \gls{DSL}
k účelům automatické distribuce a centrální správy byznys pravidel,
nicméně v sekci~\ref{sec:implementation-requirements} jsme potřebu takového
jazyka jasně identifikovali a následně v kapitole~\ref{ch:reserse}
jsme došli k závěru, že momentálně neexistuje vhodné \gls{DSL},
které by splňovalo všechny naše požadavky a mohli bychom ho
využít pro naše účely. V rámci implementace prototypu knihoven
jsme navrhli a implementovali vlastní \gls{DSL} v jazyce \gls{XML},
jak jsme popsali v sekci~\ref{sec:dsl-impl}. Tento jazyk je však
velmi omezený a snaží se vyhovět co nejnižším nárokům na implementaci.
Sestavení kvalitního jazyka pro naše účely je tématem nejméně pro
bakalářskou práci. Nicméně, námi navržený framework je schopen toto
rozšíření pojmout, stačí doimplementovat plug-in, který se bude starat
o převod z daného \gls{DSL} do paměťové reprezentace byznysového kontextu.

Kvalitní jazyk by měl kromě výše zmíněných požadavků pro zachycení
pravidla poskytovat co nejpřehlednější zápis, aby ho mohl snadno číst a zapisovat
nejen vývojář, ale i doménový expert či administrátor systému. Tím by se
ještě zvýšil přínos centrální administrace byznysových pravidel, kterou jsme v rámci
této práce implementovali a popsali v sekci~\ref{sec:central-administration}.
Můžeme také diskutovat, že by jazyk pro popis byznysových kontextů sloužil pouze
jako platforma a samotná pravidla by byla popsána v \gls{DSL} vytvořeném na míru
byznysové doméně, pro kterou by byl implementován systém využívající našeho frameworku.


\section{Možnost\'{\i} uplatněn\'{\i} navrženého frameworku}

\section{Dalš\'{\i} možnosti uplatněn\'{\i} \gls{AOP} v \gls{SOA}}

\todo{
\begin{itemize}
    \item{Extrakce dokumentace}
    \item{Extrakce byznysového modelu}
    \item{Konfigurace prostřed\'{\i}}
\end{itemize}
}

\section{Shrnut\'{\i}}

\todo{
\begin{itemize}
    \item{Dosáhli jsme c\'{\i}lů práce}
    \item{Stručné shrnut\'{\i} co všechno a jak jsme udělali}
\end{itemize}
}
