%\usepackage[T1]{fontenc}
%\usepackage[utf8]{inputenc}

%!TEX ROOT=../diploma-thesis.tex

\chapter{Závěr}\label{ch:zaver}

Architektura orientovaná na služby přináší snažší způsob vývoje komplexních
informačních systému pomocí členění jejich funkcionality do dílčích služeb. Díky
tomu lze snáze oddělit zodpovědnost a zvýšit znovupoužitelnost jednotlivých
komponent systému. Existuje však funkcionalita, která zasahuje do více služeb
najednou a je potřeba ji všude vykonávat konzistentně. Zástupcem této funkcionality jsou
byznysová pravidla, která zajišťují validní vykonávání byznysových procesů
a konzistenci dat uložených v systému. Při využití stávajících přístupů je
potřeba tato pravidla ve službách manuálně duplikovat, což zvyšuje náklady
spojené s vývojem a údržbou takových systémů.

Tato práce se věnuje problematice byznysových pravidel v architektuře orientované
na služby a navrhuje způsob, jakým lze usnadnit práci vývojářů a administrátorů
pomocí centrální správy pravidel a jejich automatické distribuce a integrace.
K tomu využívá aspektově orientovaného programování a na něm založeného
přístupu \gls{ADDA}.

\section{Přínos a možnosti použití frameworku}

Framework navržený v této práci přináší způsob, kterým mohou vývojáři systémů
stavějících na \gls{SOA} výrazně ušetřit náklady spojené s manuální duplikací
byznysových pravidel v jednotlivých službách. Tento framework je nezávislý na
platformě a díky tomu může být využíván i v technologicky heterogenních systémech,
jejichž služby využívají více programovacích jazyků. Ačkoliv se text této práce odkazuje
zejména na třívrstvou architekturu, framework neklade nároky na architekturu jednotlivých
služeb.

Pro využití frameworku musejí vývojáři zachytit byznysová pravidla ve speciálním
\gls{DSL} a navrhnout mapování a hirearchii byznysových kontextů a byznysových operací.
Díky tomu ale zvyšují znovupoužitelnost pravidel, ulehčují jejich centrální administraci
a realizaci změnových požadavků. Tato počáteční investice se tedy vyplatí od určité velikosti systému,
kdy náklady na manuální duplikaci pravidel přesáhnou cenu za návrh a využití \gls{DSL}.

\section{Budouc\'{\i} rozšiřitelnost frameworku}

Navržený framework podporuje budoucí rozšíření o další moduly a funkce, které
dále zvýší jeho schopnost znovupoužití byzynsových pravidel v \gls{SOA}.
Díky svobodné licenci \gls{MIT} může framework kdokoliv volně používat a modifikovat.

\subsection{Univerzální doménově specifický jazyk}
Zadáním této práce nebylo zkonstruovat vlastní \gls{DSL}
k účelům automatické integrace a centrální správy byznys pravidel,
nicméně v sekci~\ref{sec:implementation-requirements} byla potřeba takového
jazyka identifikována. Kapitola~\ref{ch:reserse}
došla k závěru, že momentálně neexistuje vhodné \gls{DSL},
které by splňovalo všechny požadavky kladené na navržený framework.
V rámci implementace prototypu knihoven
bylo navrženo a implementováno \gls{DSL} v jazyce \gls{XML},
popsané v sekci~\ref{sec:dsl-impl}. Tento jazyk je však
omezený a slouží pouze jako nástroj pro demonstraci navrženého řešení.
Sestavení komplexního jazyka pro popis byznysových kontextů je proto vhodným
rozšířením frameworku. Ten je k tomu navíc plně připraven.

\subsection{Integrace frameworku s uživatelským rozhraním}

V sekci~\ref{sec:architecture} je popsána architektura frameworku, která umožňuje využití
ve všech třech standardních vrstvách \gls{EIS}. Autoři přístupu \gls{ADDA} již
vyvinuli způsob, kterým lze integrovat vyhodnocování byznysových pravidel do uživatelského
rozhraní. Propojení s navrženým frameworkem by znamenalo implementovat adaptér, který by převáděl
reprezentaci byznysového pravidla do podoby, kterou je schopen využívat aspect weaver v \gls{UI}.
Díky tomu by bylo umožněno další snížení nákladů na vývoj a údržbu systému využívající framework.
Zároveň by došlo ke zvýšení uživatelského komfortu díky dynamické validaci vstupních hodnot formulářů
na straně klienta.

\subsection{Integrace frameworku s datovou vrstvou}

Integrace do datové vrstvy \gls{EIS} je další z možností budoucí rozšiřitelnosti navrženého frameworku.
Podobně jako v případě \gls{UI}, autoři přístupu \gls{ADDA} navrhují způsob, kterým lze automaticky distribuovat
post-conditions do datové vrstvy transformováním jejich podmínek do výrazů v jazyce \gls{SQL}.
Implementací a napojením příslušného aspect weaveru na navržený framework by byla pokryta další oblast,
ve které může docházet k manuální duplikaci byznysových pravidel.

\section{Dalš\'{\i} možnosti uplatněn\'{\i} \gls{AOP} v \gls{SOA}}

Byznysová pravidla nejsou jediným průřezovým problémem, se kterým se systémy stavějící na \gls{SOA} musejí
vypořádat. Jak bylo popsáno v sekci~\ref{sec:adda}, autoři přístupu \gls{ADDA} identifikují strukturu
doménového modelu jako průřezový problém zasahující do všech standardních vrstev \gls{EIS},
zejména pak do \gls{UI}. V \gls{SOA} může být struktura doménového modelu navíc sdílena mezi jednotlivými
službami, což se promítá zejména do rozhraní, pomocí kterých spolu komunikují.
Díky \gls{AOP} by bylo možné automaticky integrovat tuto strukturu do kódu, který komunikaci obsluhuje.

Dalším průřezovým problémem, kterému se autoři \gls{ADDA} ve svém výzkumu věnují, je extrakce dokumentace.
V rámci \gls{SOA} by se pak \gls{AOP} dalo využít k extrakci informací o doménovém modelu, implementovaných use-cases
a byznysových pravidlech a následnému automatickému generování dokumentace.

Posledním průřezovým problémem v \gls{SOA} je společná konfigurace parametrů služeb, zejména pak v
decentralizovaném systému využívajícím architekturu Microservices. Příkladem takového parametru může
být v ukázkovém e-commerce systému výše DPH, která se využívá v objednávkové i
fakturační službě. Pomocí centrální, automaticky distribuované konfigurace by se dalo zamezit nekonzistencím
a zabránit tak poškozením dat v systému, či jiným závažným chybám.

\section{Shrnut\'{\i}}

%\goal{Naplnění požadavků ze zadání}
V rámci této práce byly dosaženy cíle stanovené v úvodu práce a byly splněny všechny
požadavky zadání. Nejprve bylo analyzováno využití, vyjádření a znovupoužití
byzynsových pravidel v \gls{SOA} a byly identifikovány potenciální problémy a z nich vyplývající požadavky
na framework navržený v této práci. Dále byla provedena rešerše vhodných architektur,
paradigmat a nástrojů, které by mohly být použity pro řešení těchto problémů, včetně stávajících
frameworků pro reprezentaci byznysových pravidel a jejich výhod a nevýhod. Na základě analýzy a
rešerše byl navržen framework pro správu a automatickou integraci byznysových pravidel
v architektuře orientované na služby. Tento framework využívá koncepty aspektově orientovaného programování
a na něm založeného přístupu \gls{ADDA}. Funkčnost navrženého řešení byla demonstrována implementací a
otestováním prototypů knihoven pro platformy jazyků Java a Python a platformu Node.js. Tyto knihovny byly
použity pro vývoj jednoduché ukázkové e-commerce aplikace, která je tvořena šesti službami naprogramovanými
ve třech různých jazycích. Na této aplikaci byla ukázána schopnost frameworku centrálně administrovat byznysová
pravidla a automaticky je sdílet a integrovat do služeb aplikace. Zároveň byl změřen a analyzován
dopad použití frameworku na počet duplikací byznysových pravidel a byl diskutován jeho vliv na
údržbu systému. V závěru práce byly navíc analyzovány další oblasti \gls{SOA}, kde by bylo možné aplikovat
aspektově orientované programování.
