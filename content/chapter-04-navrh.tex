%!TEX ROOT=../diploma-thesis.tex

\chapter{Návrh frameworku}\label{ch:navrh}

V této kapitole je diskutován návrh frameworku pro centrální správu a automatickou integraci
business pravidel v prostředí architektury orientované na služby. Tento návrh staví na znalostech
získaných v předchozí kapitole~\ref{ch:reserse}, zejména na paradigmatu \gls{AOP} a přístupu \gls{ADDA},
a vyhovuje požadavkům identifikovaným v sekci~\ref{sec:implementation-requirements}

\section{Vize frameworku}

Navrhovaný framework vychází z přístupu \gls{ADDA} představeného v sekci~\ref{sec:adda}
a jeho cílem je umožnit zápis byznysových pravidel v \textit{single-focal point}, ze kterého
budou pravidla automaticky integrována do jednotlivých služeb systému. Pro zápis pravidel
bude framework využívat speciální \gls{DSL}, které umožní jejich snadný zápis a inspekci.
Díky tomu bude možné redukovat manuální duplikaci pravidel, která jsou sdílena mezi jednotlivými
službami a tím snížit náklady na vývoj a údržbu systému a zmírnit riziko lidské chyby,
které je spojeno s manuální synchronizací duplikované logiky.

\section{Formalizace architektury orientované na služby}

Využítí přístupu \gls{ADDA} vyžaduje formalizaci problematiky byznysových pravidel v \gls{SOA}
do termínů \gls{AOP}. Za aspekt bude framework považovat byznysové pravidlo. Dále je nutno
identifikovat \textit{join-points}, určit podobu \textit{advices}, popsat způsob jakým budou
zachyceny \textit{pointcuts} a nakonec navrhnout proces \textit{weavingu} pravidel.

\subsection{Join-points}

Identifikace join-points vychází ze životního cyklu služby, který je znázorněn
na obrázku~\ref{fig:join-points}. První fází v životním cyklu služby je její inicializace,
konkrétně načtení exekučního kontextu. V tomto bodě je potřeba získat veškerá pravidla, která
bude služba potřebovat ke své funkci.
Po inicializaci vstupuje služba do fáze, ve které může přijímat požadavky
na vykonání byznysových operací. Při přijmutí požadavku je nejprve nutno určit
byznysový kontext a poté vyhodnotit veškeré \textit{preconditions}. Pokud jsou všechny předpoklady
pro spuštění operace splněny, může být vykonána. Po dokončení operace je nutno aplikovat relevantní
post-conditions.

\begin{figure}
    \centering
    \includegraphics[keepaspectratio=true, width=0.6\linewidth]{figures/join-points.pdf}
    \caption{Diagram životn\'{\i}ho cyklu služby a identifikovan\'ych join-pointů}
    \label{fig:join-points}
\end{figure}

Identifikované join-points tedy jsou:

\benum[label=\circledarabic]
\item\label{itm:initialization} Inicializace instance služby
\item\label{itm:before} Volání byznysové operace
\item\label{itm:after} Dokončení byznysové operace
\eenum

\subsection{Pointcuts}

\lstinputlisting[
caption={Ukázka zápisu validačních pravidel pomocí anotací v jazyku Java},
label={lst:jsr303},
language=Java,
%frame=single,
%float,
%floatplacement=t
]
{code/jsr303.java}

V join-pointech~\ref{itm:before}~a~\ref{itm:after} musejí být aplikována všechna byznysová pravidla
vztahující se k prováděné byznysové operaci.
Pro zápis poincutu byznysového pravidla pro tyto join-pointy se lze inspirovat standardem \gls{JSR}
303~\cite{bernard2009jsr}, který umožňuje validovat data byznysových objektů vstupujících do
byznysových operací pomocí anotací atributů těchto objektů. Příklad validačních anotací je znázorněn
ve zdrojovém kódu~\ref{lst:jsr303}, kde je pomocí anotace \code{$@$NotBlank} zajištěno, že fakturační
adresa bude mít vyplněno pole \code{country} -- v kontextu navrhovaného frameworku se jedná o paralelu preconditions.
Podobným způsobem by každá byznysová operace mohla pomocí meta-instrukcí specifikovat, která byznysová
pravidla bude využívat. Toto řešení však neposkytuje možnost dynamicky při běhu programu změnit sadu
byznysových pravidel, ani sdílení pravidel mezi jednotlivými operacemi. Tento problém lze řešit využitím
konceptu byznysového kontextu, který zapouzdřuje byznysová pravidla, a byznysová operace se na něj může
explicitně odkázat. Obsah byznysového kontextu přitom může být dynamicky změněn za běhu programu. Příklad provázání
byznysové operace a byznysového kontextu v jazyce Java je znázorněn ve zdrojovém kódu~\ref{lst:business-operation-context}.
Implementace byznysové operace pomocí anotace \code{$@$BusinessContext} definuje, který byznysový kontext má být
při jejím vykonávání využit. Kontext je definován separátně pomocí \gls{DSL} a lze ho měnit bez zásahu do implementace
operace.

\lstinputlisting[
caption={Příklad provázání byznysové operace a byznysového kontextu pomocí anotací v jazyce Java},
label={lst:business-operation-context},
language=Java,
%frame=single,
%float,
%floatplacement=t
]
{code/business_operation_context.java}

Sdílení pravidel mezi byznysovými kontexty, potažmo byznysovými operacemi a mezi jednotlivými službami,
lze realizovat pomocí dědičnosti kontextů. Každý kontext, který by potřeboval validovat fakturační
adresu, by tak mohl dědit od kontextu vytváření faktury. Uvažme příklad uvedený v sekci~\ref{sec:shortcomings}.
Vytváření objednávky sdílí byznysová pravidla operace vytváření faktury. Byznysové operace se mohou odkazovat
na své byznysové kontexty, které mají být při jejich vykonávání použity. Pokud by kontext vytváření objednávky
dědil od kontextu vytváření faktury, mohl by sdílet jeho pravidla, včetně validace fakturační adresy. Pokud by bylo
potřeba upravit, přidat či sdílet další pravidla, nebylo by nutné zasahovat do kódu služeb.

V join-pointu~\ref{itm:initialization} načte služba všechna byznysová pravidla, která
bude potřebovat ke své činnosti. V případě kompozitní služby je nutno načíst i sdílená pravidla
definovaná v jiných službách. Kompozitní služba tedy musí zjistit, která pravidla je potřeba získat,
a následně si je vyžádat od ostatních služeb. Pointcut byznysových pravidel v tomto join-pointu vychází
z dědičnosti byznysových kontextů, která určuje, jaká pravidla budou mezi službami přenesena.
V příkladu ze sekce~\ref{sec:shortcomings} je pointcut pravidla validujícího fakturační adresu
určen dědičností kontextu vytváření objednávky od kontextu vytváření faktury.


\subsection{Advices}

V případě join-pointu~\ref{itm:initialization} je advice samotná reprezentace byznysového
kontextu přenášeného mezi službami. Naopak v join-pointech~\ref{itm:before}~a~\ref{itm:after}
je přidanou funkcionalitou vyhodnocování preconditions nad aplikačním kontextem, resp. aplikování
post-conditions na návratovou hodnotu operace.

\subsection{Weaving}

\begin{figure}
    \centering
    \includegraphics[keepaspectratio=true, width=0.9\linewidth]{figures/business-rules-weaver.pdf}
    \caption{Proces weavingu byznysov\'ych pravidel}
    \label{fig:business-rules-weaver}
\end{figure}

Proces weavingu je zachycen na obrázku~\ref{fig:business-rules-weaver}. Weaving v případě
join-pointu~\ref{itm:initialization} bude provádět komponenta frameworku, která
analyzuje lokálně dostupná pravidla služby, vyhodnotí, která pravidla je potřeba stáhnout,
a vyžádá tato pravidla od příslušných služeb. V případě join-pointů~\ref{itm:before}~a~\ref{itm:after}
je k weavingu potřeba využít speciální aspect weaver. Ten zachytí volání byznysové operace a získá
informace o aktuálním stavu aplikačního kontextu. Následně zjistí, který byznysový kontext má být aplikován,
shromáždí všechny preconditions a každou z nich vyhodnotí. Pokud některá precondition není splněna, byznysová
operace je zastavena a je vyhozena výjimka, kterou služba zpracuje. V opačném případě je kontrola vrácena zpět
službě, která vykoná byznysovou operaci. Po dokončení operace aspect weaver zachytí výstup byznysové
operace a aplikuje post-conditions daného byznysového kontextu.

\section{Popis byznysových kontextů}

Př\'{\i}stup \gls{ADDA} doporučuje popsat byznysová pravidla pomoc\'{\i}
vlastn\'{\i}ho, na m\'{\i}ru šitého, doménově specifického jazyka~\cite{cemus2015automated}.
Pro účely frameworku bude popsán pomocí \gls{DSL} celý byznysový kontext.
Jak bylo popsáno v sekci~\ref{sec:business-rule-dsl}, vlastnosti nástrojů Drools a JetBrains MPS
nejsou pro použití v této práci optimální. Pro účely frameworku budou specifikovány vlastnosti,
kterými by \gls{DSL} mělo disponovat, a jeho konkrétní podoba bude přenechána na implementaci
frameworku.

\subsection{Dědičnost byznysových kontextů}\label{sec:context-inheritance}

V předchozím textu byl představen koncept dědičnosti byznysových kontextů.
Každý kontext díky němu může dědit od libovolného množství jiných kontextů,
a sdílet tak jejich byznysová pravidla. Byznysové operace pak mohou samy určit,
který byznysový kontext se k nim váže. \gls{DSL} pro popis pravidel musí umožňovat
dědičnost zachytit. Tento koncept však přináší několik problémů, které jsou rozebrány
v následujících odstavcích.

Může nastat situace, kdy je potřeba sdílet pouze některá byznysová pravidla
daného kontextu. Při mapování kontextů jedna ku jedné s operacemi by to ale
nebylo možné. Řešením je využití tzv. \textit{abstraktních kontextů},
které přímo nevyužívá žádná byznysová operace. Příklad znázorněný na
obrázku~\ref{fig:abstract-context} popisuje situaci, kdy je nežádoucí,
aby kontext \code{user.register} zdědil pravidlo vyžadující přihlášení uživatele.

\begin{figure}
    \centering
    \includegraphics[keepaspectratio=true, width=1\linewidth]{figures/abstract-context.pdf}
    \caption{Znázornění abstraktního byznysového kontextu}
    \label{fig:abstract-context}
\end{figure}

Kvůli dědičnosti může vzniknout v grafu závislostí kontextů cyklus, který by způsobil zacyklení
procesu inicializace v~\ref{itm:initialization}. Řešením je validátor vestavěný do nástroje pro
správu byznysových kontextů, který bude případné cykly detekovat a reportovat uživateli.

Vícenásobná dědičnosti může přinést problém, kdy jeden kontext zdědí více stejných pravidel z
různých zdrojů, tzv. \textit{diamond problem}~\cite{boyen1994generalized}. Tomu lze předejít tak, že
každé pravidlo bude mít unikátní identifikátor v rámci celého systému a při dědění budou zohledněna
pouze unikátní pravidla. Zajištění unikátního identifikátoru lze zajistit díky nástroji pro centrální
administraci byznysových pravidel, který má přehled o všech pravidlech. Díky tomu může administrátora
upozornit na případnou duplikaci.

\subsection{Logické podmínky byznysových pravidel}\label{sec:expressions-design}

Sekce~\ref{sec:business-rules} uvádí, že pravidla obsahují logické podmínky. V případě preconditions je to
ověření podmínky, která musí být platná před spuštěním byznysové operace v~\ref{itm:before}. V případě
post-condition může filtrování návratové hodnoty podléhat splnění určité podmínky, která musí být
vyhodnocena v~\ref{itm:after}. Zápis logických podmínek bude závislý na konkrétní implementaci \gls{DSL}.
Nicméně návrh frameworku bude předpokládat, že po načtení z \gls{DSL} budou jednotlivé výrazy podmínek v paměti
uloženy jako strom samostatných objektů podle návrhové vzoru Composite~\cite{fowler2002patterns}. Díky tomu
bude framework moci využít k jejich vyhodnocování návrhový vzor Interpreter~\cite{fowler2002patterns}.
\gls{DSL} tedy musí umožňovat takový zápis logických podmínek, který půjde převést do této podoby.

\subsection{Filtrování návratových hodnot byznysové operace}

Při aplikování post-conditions je filtrována návratová hodnota byznysové operace. Tou může být proměnná
obsahující číslo, text, objekt, či jejich kolekce. Filtrování jednoduchých hodnot nemá pro byznysová pravidla reálný přínos.
V případě objektu lze filtrovat jeho atributy, například skrýt e-mailovou adresu uživatele.
V případě kolekce lze filtrovat jejich prvky, například skrýt objednávky, které uživateli nepatří.
Pokud se v kolekci nachází objekty, lze požadovat, aby byly zakryty atributy jednotlivých objektů,
například filtrování e-mailových adres v kolekci více uživatelů. Identifikovanými typy post-conditions jsou:

\begin{itemize}
    \item \code{FILTER\_OBJECT\_FIELD} filtruje atribut objektu, který je výstupem operace.
    \item \code{FILTER\_LIST\_OF\_OBJECTS} filtruje objekty v kolekci, která je výstupem operace.
    \item \code{FILTER\_LIST\_OF\_OBJECTS\_FIELDS} filtruje atributy objektů v kolekci, která je výstupem operace.
\end{itemize}

\subsection{Shrnutí}

Na základě textu této kapitoly lze vyvodit specifikaci, kterou musí implementace \gls{DSL} pro zápis
byznysových kontextů splňovat. Požadované vlastnosti jsou:

\begin{itemize}
    \item Označení byznysových kontextů unikátními identifikátory, které se skládají z prefixu a jména
    \item Podpora dědičnosti byznysových kontextů včetně abstraktních konceptů
    \item Zápis preconditions a jejich označení unikátním identifikátorem
    \item Zápis post-conditions, jejich označení unikátním identifikátorem a zápis jejich typu
    \item Zápis logických výrazů byznysových pravidel a možnost jejich transformace do návrhového vzoru Composite
\end{itemize}

\section{Metamodel byznysového kontextu}\label{sec:metamodel}

\begin{figure}
    \centering
    \includegraphics[keepaspectratio=true, width=\linewidth]{figures/business-context-metamodel.pdf}
    \caption{Metamodel byznysového kontextu}
    \label{fig:business-context-metamodel}
\end{figure}

Z předchozího textu vyplývá podoba metamodelu byznysových kontextů, pomocí které
budou reprezentovány v paměti počítače. Kromě samotných logických výrazů musí pravidlo nést
informace o tom, zda se jedná o precondition nebo post-condition, a také jeho identifikátor.
Post-condition navíc potřebuje uložit informaci o jejím typu a názvu. Pravidla jsou uskupována
do byznysových kontextů, z nichž každý má svůj unikátní identifikátor skládající se z prefixu
a samotného jména a seznam kontextů, od kterých dědí. Diagram tříd navrženého kontextu je znázorněn
na obrázku~\ref{fig:business-context-metamodel}. Abstraktní třída \code{BusinessRule} reprezentuje
byznysové pravidlo a je rozšířena třidami \code{Precondition} a \code{PostCondition}. Typ
postcondition je reprezentován enumerací \code{PostConditionType}. Logické podmínky
byznysových pravidel budou rozšiřovat rozhraní \code{Expression}, které podporuje návrhový vzor
Interpreter. Byznysová pravidla jsou uskupena do byznysových kontextů, které jsou reprezentovány
třídou \code{BusinessContext}. Identifikátory kontextů jsou reprezentovány třídou \code{BusinessContextIdentifier}.

\section{Organizace byznysových kontextů}

Každá služba bude mít lokálně uložen popis byznysových kontextů, které se sémanticky vztahují
k její doméně. Pro snazší přidělení byznysových kontextů ke službám bude v identifikátoru kontextu sloužit
tzv. \textit{prefix}. Kontexty se stejným prefixem pak budou spravovány výhradně jednou službou. Například kontexty
služby spravující objednávky budou označeny prefixem \code{order}, zatímco kontexty služby zajišťující fakturaci budou
označeny prefixem \code{billing}.

\subsection{Registr byznysových kontextů}\label{sec:registry-design}

Cílem frameworku je popsat byznysové kontexty v jednom místě, ze kterého budou
automaticky distribuovány. Pro tento účel bude každá služba disponovat registrem byznysových pravidel
(třída \code{BusinessContextRegistry}), který bude mít za úkol kontexty načítat z \gls{DSL} do metamodelu,
stahovat lokálně nedostupné kontexty z ostatních služeb a načtené kontexty uchovávat pro použití při weavingu.
Při inicializaci kontextů spolu budou registry jednotlivých služeb komunikovat a vyměňovat si sdílené kontexty.

Postavení byznysových registrů vůči jednotlivým službám je ilustrováno na obrázku~\ref{fig:layers-vs-framework}.
Registry kompozitních služeb budou využívat rozhraní registrů ostatních služeb, jejichž byznysová
pravidla sdílejí.

\begin{figure}
    \centering
    \includegraphics[keepaspectratio=true, width=0.7\linewidth]{figures/layers-vs-framework.pdf}
    \caption{Postavení registrů byznysových kontextů vůči jednotlivým službám}
    \label{fig:layers-vs-framework}
\end{figure}

\subsection{Uložení kontextů}

Byznysové kontexty popsané pomocí \gls{DSL} mohou být v příslušné službě uloženy v souborech na disku či v
databázi. Navrhovaný framework by na způsobu uložení neměl být závislý a o potřebné kroky
pro načtení či případně uložení kontextu se postará konkrétní implementace. Pro tento účel
je tedy vhodné, aby registr pracoval s nekonkrétními rozhraními, na jejichž implementaci
nebude nijak záviset.

\section{Distribuce sdílených byznysových kontextů mezi službami}

Navržený framework má za úkol automaticky integrovat byznysová pravidla, potažmo byznysové kontexty,
do jednotlivých služeb systému. V případě, že je byznysový kontext sdílen mezi více službami, je
nutné ho distribuovat ze služby, ve které je definován. K tomu framework využije síťovou komunikaci.

\subsection{Formát pro síťový přenos}

Pro přenos kontextů po síti lze využít přímo \gls{DSL}, ve kterém jsou pravidla uložena. Takové
řešení by ale mohlo být neefektivní kvůli velikosti přenášených dat. Implementace proto může zvolit
specializovaný síťový formát. Framework umožňuje využít oba tyto způsoby.

\subsection{Architektura síťové komunikace}

Architektura pro síťový přenos musí stavět na modelu klient-server. Framework předpokládá, že kompozitní
služba bude zastávat roli klienta, a služba, která sdílí svá pravidla, bude zastávat roli serveru.
Konkrétní podoba je stejně jako přenosový formát přenechána na implementaci. Lze využít přístup \gls{RPC},
který umožní volání procedur vracejících požadované byznysové kontexty. Stejně tak může implementace
využít architekturu \gls{REST}, kdy budou byznysové kontexty vnímány jako jednotlivé resources a budou
dostupné na unikátní \gls{URI}.

\subsection{Service discovery}

Aby framework mohl distribuovat byznysové kontexty mezi službami, musí služba vyžadující kontext
znát adresu služby, od které ho vyžaduje. Adresy služeb mohou podléhat různým konfiguracím,
které se mohou lišit systém od systému. Framework proto nesmí být závislý na způsobu,
jakým se adresování služeb provádí. Nejlepším řešením je přenechat na uživateli frameworku,
aby sám získal a předal adresy služeb ve chvíli, kdy je framework potřebuje -- tedy
ve chvíli, kdy je potřeba načíst lokálně nedostupné kontexty.

\section{Inicializace byznysových kontextů}

\begin{figure}
    \centering
    \includegraphics[keepaspectratio=true, width=\linewidth]{figures/business-context-loading.pdf}
    \caption{Proces inicializace byznysov\'ych kontextů}
    \label{fig:business-context-loading}
\end{figure}

Jak bylo uvedeno v předchozím textu, v join-pointu~\ref{itm:initialization} je potřeba inicializovat byznysové
kontexty, které bude služba ke své funkci potřebovat.
Na obrázku~\ref{fig:business-context-loading} je znázorněn navržený proces inicializace.
Na obrázku~\ref{fig:business-context-loading} je znázorněn navržený proces inicializace.
Při něm jsou nejprve načteny lokálně dostupné kontexty popsané pomocí \gls{DSL}.
Po převedení kontextů z \gls{DSL} do metamodelu je shromážděn seznam kontextů, od kterých ostatní kontexty dědí,
a z nich jsou vybrány ty, které nejsou lokálně dostupné. Následně jsou tyto vzdálené kontexty vyžádány od příslušných služeb
a po obdržení jsou převedeny ze síťového formátu do metamodelu. Nakonec jsou sdílená pravidla kontextů začleněna
do kontextů, které od nich dědí. Celou inicializaci bude zastřešovat komponenta \code{BusinessContextRegistry}, která
má znalost o všech subsystémech, které jsou k tomuto procesu potřeba. Tato komponenta implementuje
návrhový vzor \textit{Facade}~\cite{fowler2002patterns}.

\section{Centráln\'{\i} správa byznysových kontextů}\label{sec:central-administration-design}

\begin{figure}
    \centering
    \includegraphics[keepaspectratio=true, width=0.8\linewidth]{figures/business-context-management.pdf}
    \caption{Proces centráln\'{\i} správy byznysov\'ych kontextů}
    \label{fig:business-context-management}
\end{figure}

Jak bylo popsáno v předchozím textu, byznysové kontexty budou podle prefixu přiděleny službám,
které budou spravovat jejich aktuální a jediný stav a poskytovat je jiným službám. Aby bylo
možno pravidla centrálně spravovat, musí framework poskytovat proces, kterým je bude moci ze
služeb získat a při úpravě je do nich znovu uložit.

\subsection{Uložen\'{\i} sdíleného kontextu}\label{sec:saving-context}

Při ukládání byznysového kontextu je potřeba jeho změnu propagovat také do všech kontextů, které od něj dědí.
Při změně sdíleného kontextu budou všechny služby, které ho využívají, informovány pomocí nástroje pro
centrální správu byznysových pravidel. Ten má informaci o všech závislostech v systému a zároveň zná i adresu všech
služeb. Nevýhodou tohoto přístupu je zvýšená komunikační zátěž kvůli většímu objemu přenesených informací, stejný kontext
je totiž potřeba rozeslat mezi více služeb. Při implementaci je nutno zvážit, zda je tato zátěž vůči absolutnímu objemu
přenášených dat v systému významná. Při implementaci by bylo vhodné vybrat vhodný přenosový formát, který minimalizuje dopad
veškeré síťové komunikace týkající se distribuce byznysových pravidel.

\subsection{Proces úpravy kontextu}

Na obrázku~\ref{fig:business-context-management} je znázorněn proces úpravy kontextu. Nástroj pro
centrální administraci nejprve načte všechny byznysové kontexty všech služeb v systému. Administrátor
si vybere konkrétní pravidlo, a nástroj mu následně zobrazí formulář pro úpravu pravidla.
Pravidlo je pro účely formuláře převedeno z metamodelu do \gls{DSL}. Po odeslání formuláře je pravidlo
převedeno zpět do metamodelu. Nástroj pro administraci poté analyzuje, na které služby bude mít změna pravidla
dopad. Následně zašle upravené pravidlo službě, která ho spravuje, a ostatním službám, jejichž kontexty upravené
pravidlo rozšiřují, rozešle požadavek na jejich aktualizaci. Proces uložení nového kontextu je analogický, až na
rozeslání požadavku na aktualizaci závislých kontextů, která není u nového pravidla potřeba.

\section{Architektura frameworku}\label{sec:architecture}

V této sekci je popsána obecná architektura navrženého frameworku v rámci služby využívající
klasickou třívrstvou architekturu~\cite{fowler2002patterns}, která se skládá z prezentační,
aplikační a datové vrstvy. Každá z těchto vrstev může framework využívat pro weaving byznysových pravidel
-- prezentační vrstva při validování vstupních polí formuláře, aplikační vrstva při
aplikaci byznysových pravidel v byznysových operacích a datová vrstva při aplikaci post-conditions pro
filtrování dat při jejich získávání z databáze.

\begin{figure}
    \centering
    \includegraphics[keepaspectratio=true, width=\linewidth]{figures/business-context-registry.pdf}
    \caption{Architektura navrženého frameworku}
    \label{fig:business-context-registry}
\end{figure}

Architektura frameworku je zachycena na obrázku~\ref{fig:business-context-registry}.
Základem frameworku je komponenta \code{BusinessContextRegistry}, tedy registr
byznysových kontextů, který je zodpovědný za inicializaci a uchovávání byznysových kontextů.
Načítání kontextů lze rozdělit na lokální a vzdálené. Při načítání lokálně dostupných kontextů
je potřeba získat \gls{DSL} kontextu ze souboru či databáze a převést ho do metamodelu.
K tomu bude využito rozhraní \code{LocalBusinessContextLoader}. Implementace rozhraní může být libovolná
a záviset na použitém \gls{DSL} či místu uložení pravidel. Naopak při načítání vzdálených
kontextů je potřeba vyžádat kontexty od vzdálené služby. Třída \code{RemoteBusinessContextLoader} požadované
kontexty zorganizuje podle prefixu a poté pomocí rozhraní \code{RemoteLoaderClient} načte
pravidla od jednotlivých služeb. Implementace tohoto rozhraní bude záviset na použité
technologii a zajistí síťovou komunikaci a převod do a z formátu pro síťový přenos.
Aby mohl framework poskytovat lokální byznysové kontexty dané služby ke stažení, musí zastřešit
i serverovou funkcionalitu. K tomu slouží rozhraní \code{BusinessContextServer}. To bude využívat
\code{BusinessContextRegistry}, ze kterého načte byznysové kontexty, které si vyžádá \code{RemoteLoaderClient}.
Implementace serveru bude opět závislá na konkrétní technologii.
Nakonec bude framework obsahovat sadu aspect weaverů, které umožní weaving byznysových pravidel do
jednotlivých vrstev systému. Pro účely této práce bude framework poskytovat weavery pro využití v
aplikační vrstvě pro weaving preconditions a post-conditions do byznysových operací.

\section{Shrnutí}

V této kapitole byl popsán návrh frameworku pro centrální správu a automatickou distribuci
byznysových pravidel v \gls{SOA} na základě přístupu \gls{ADDA}. Nejprve byla formalizována
doména byznysových pravidel v \gls{SOA} do názvosloví \gls{AOP}. Dále byla diskutována podobu byznysových pravidel,
jejich logických výrazů a jakým způsobem je lze zachytit v metamodelu a v \gls{DSL}.
Kapitola dále popisuje organizaci kontextů a procesy, kterými budou distribuovány a spravovány.
Nakonec byla shrnuta architektura frameworku.
