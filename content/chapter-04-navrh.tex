\usepackage[T1]{fontenc}
\usepackage[utf8]{inputenc}

%!TEX ROOT=../diploma-thesis.tex

\chapter{Návrh}\label{ch:navrh}

\section{Formalizace architektury orientované na služby}

\subsection{Join-points}

\subsection{Advices}

\subsection{Pointcuts}

\subsection{Weaving}

\section{Architektura frameworku}

\section{Zachycení byznysového konextu}

Přístup ADDA doporučuje popsat byznysová pravidla pomocí
vlastního, na míru šitého, doménově specifického jazyka~\cite{cemus2015automated}.
V našem případě můžeme jazykem DSL popsat kompletně i celý
byznysový kontext.

\section{Metamodel byznys kontextu}\label{sec:metamodel}

\section{Expression}

\section{Registr byznys kontextů}

\section{Byznys kontext weaver}

\section{Centrální správa byznys kontextů}

\subsection{Uložení rozšířeného pravidla}

\goal{Diskutovat chaining vs. direct update}
% TODO: napiš mě

\section{Service discovery}

\goal{Popsat nezávislost na service discovery}
